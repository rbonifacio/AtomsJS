\section{Threats to Validity}
\label{threat}

Since our research was conducted online with unknown participants, and thus we had no way of confirming their
levels of education and experience. Also,  we did not have a way to prevent respondents from cheating,
such as running the code on an interpreter, or consulting other people. This presents a threat to the validity to the
conclusions of our work, even though we leverage some means to try to mitigate this threat (as discussed
in Section~\ref{method}).

Another threat to our results lies in the fact that some constructs
might not be frequent, or present at all, in JavaScript code bases.
The atom candidates Arithmetic as Logic and
the Comma Operator were rarely found in the
open source JavaScript code we mined in our
third study.

A potential problem with our method is that
there little incentive for subjects to think
thoroughly about the questions. We observed a lack of
engagement when we ran the survey with undergraduate students during
a pilot study. Although our final subjects were voluntarily
partaking in the survey, we could not be sure that, after some
time taking the survey, respondents would become tired and stop thinking
clearly about the code.

Finally, in one of the atoms, namely the Omitted Curly Braces,
we intentionally removed indentation from the original code,
which is highly unusual, given that many programmers use automatic
formatting in their code editors. This can introduce some level
of artificiality to this atom's question. Nonetheless,
we discovered that JavaScript does allow the programmer
to omit the curly braces after \textit{if} statements,
and insert multiple statements in the following line.
This fact itself might constitute
a source of confusion, which we leave to analyse
in our future endeavors. 
