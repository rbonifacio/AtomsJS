\section{Threats to Validity}
\label{threat}

Since our research was conducted online with unknown participants, we had no way of confirming their levels of education and experience. Also, as we briefly discussed in Section 4.3, we did not have a way to prevent respondents from cheating, such as running the code on an interpreter, or consulting other people. This presents a threat to the validity of the histograms presented in Figures \ref{fig:degree} and \ref{fig:degree}.

Another threat to our results lies in the fact that some constructs might not be frequent, or present at all, in JavaScript code bases. The Automatic Semicolon Insertion and the Comma Operator atoms were not found in the code samples we consulted when formulating the questions, and we had to create blocks of code containing the atoms. The work done in Chapter 5 could serve as a starting point for detecting these atoms in a large data set of JavaScript files.

Regarding hypotheses H.2 \todo[inline]{Linkar as Hipoteses} and research question RQ.2, a potential problem with our method is that there was no strong incentive for subjects to think thoroughly about the questions. We observed a lack of engagement when we ran the survey with undergraduate students. Although our final subjects were voluntarily partaking in the survey, we could not be sure that, after some time taking the survey, respondents would become tired and stop thinking clearly about the code. As questions were assigned randomly, and contained different levels of difficulty, comparing a subject's answer with their other questions would not provide a reliable metric.

In one of the atoms, namely the Omitted Curly Braces, we intentionally removed indentation from the code, which is highly unusual, given that many programmers use automatic formatting in their code editors. This can introduce some level of artificiality to this atom's question. On the other hand, while describing JavaScript's grammar to develop the parser we used for the work in Chapter 5, we discovered that the language allows the programmer to omit the curly braces after \textit{if} statements, and insert multiple statements in the following line. This fact itself might constitute a source of confusion, which we leave to analyse in our future endeavors. 
