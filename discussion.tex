\section{Discussion}
\label{sec:discussion}

Our work leads to several implications.
First, we generalize the
work of Gopstein et al.~\cite{DBLP:conf/sigsoft/GopsteinIYDZYC17},
which investigates
the impact of atom candidates on
understanding \clang and \cpplang code. That is,
similarly to their work, the atom candidates
Comma Operator, Post/Pre Increment, Omitted Curly Braces,
Assignment as Value, Implicit Predicate, Logic as
Control Flow, and Ternary Operator make the 
JavaScript code hard to understand. Our results also 
refute the hypothesis that Arithmetic as Logic is an atom of
confusion (i.e., a source of misunderstanding).
We just find some differences in the effect size
of the atom candidates, in comparison to the original
work of Gopstein et al.~\cite{DBLP:conf/sigsoft/GopsteinIYDZYC17}.


The results of the interview study complement the understanding of atoms of confusion because the participants make clear the existence of a tradeoff between code comprehension and other quality attributes. For instance, most of the participants prefer the versions of the code with the Ternary Operator, even though they agree that its uses might contribute to the misunderstanding of JavaScript code, particularly when novices are maintaining the codebase. The participants of the interview study also mentioned other possible sources of misunderstanding in JavaScript,
including the use of prototype-inheritance and nested call-backs (as discussed in Section~\ref{sec:interview-results}). Other JavaScript atom candidates include
Object Destructuring, Array Spread, Object Spread, and Type Conversion. We conducted a non-exact replication of the three studies (survey, interview, and mining software repositories) considering these more specific JavaScript atom candidates. We confirmed that they truly correspond to sources of misunderstanding. Due to lack of space,
we cannot present all the results here, and we postpone the presentation of these results as future work.



