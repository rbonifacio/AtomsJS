\section{Discussion}
\label{sec:discussion}

Our work has several implications.
First, it adds external validity to the
work of Gopstein et al.~\cite{DBLP:conf/sigsoft/GopsteinIYDZYC17},
which investigates
the impact of atom candidates on
understanding \clang,\cpplang code. That is,
similarly to their work, the atom candidates
Comma Operator, Post/Pre Increment, Omitted Curly Braces,
Assignment as Value, Implicit Predicate, and Logic as
Control Flow seem to make 
JavaScript code harder to understand. For five of them, the difference is statistically significant, with a large effect size for four atoms. Our results also 
refute the hypothesis that Arithmetic as Logic is an atom of confusion (i.e., a source of misunderstanding).
In comparison to the original
work of Gopstein et al.~\cite{DBLP:conf/sigsoft/GopsteinIYDZYC17}, 
our study
led to some differences in the effect size
of the atom candidates.
Altogether, we answer our first research question
\emph{\rqa}:

\begin{mh}
  {\bf Answer to RQ1:} The first study (survey) gives evidence that some of the atom candidates in \clang and \cpplang programs that also exist in JavaScript correspond to a source of misunderstanding in
  JavaScript code. 
\end{mh}

The results of the interview study complement the understanding of atoms of confusion because the participants make clear the existence of a trade-off between code comprehension and other quality attributes. For instance, most of the participants prefer the version of the code with the Conditional Operator, even though they agree that its use might contribute to the misunderstanding of JavaScript code, particularly when novices are maintaining the codebase. The participants of the interview study also
mentioned other possible sources of misunderstanding in JavaScript,
including the use of prototype-based inheritance and nested call-backs (as discussed in Section~\ref{sec:interview-results}). Other JavaScript atom candidates include
Object Destructuring, Array Spread, Object Spread, and Type Conversion.
In summary, the results of the second study (interviews) allow
us to answer the second (\rqb) research question.

\begin{mh}
  {\bf Answer to RQ2:} The qualitative analysis of the
  interviews supports the results of the first study,
  since JavaScript developers most often agree that atoms of confusion compromise
  source code understanding. 
\end{mh}

%% \begin{mh}
%%   {\bf Answer to RQ3:} The qualitative analysis of the
%%   interviews suggests that specific JavaScript constructs might also correspond to atoms of
%%   confusion, including prototype inheritance and
%%   nested callbacks. 
%% \end{mh}

The results of the third study (mining open source
JavaScript repositories) provides evidence that,
although atoms of confusion compromise program
comprehension, they frequently appear in open
source JavaScript projects. In particular,
seven, out of 10 atoms considered
  in our study, appear in more than 50\% of
the 72 projects we analyzed. Furthermore, at least two of them are used intensively, more than once for every 200 lines of code. In summary, the third study
allows us to answer the fourth research
question (\emph{\rqd}).


\begin{mh}
  {\bf Answer to RQ3:} The MSR study reveals that
  the several atom candidates explored in our research
  appear frequently in practice, and cleaning up the use of 
  Post-Increment/Decrement and the Automatic Semicolon Insertion
  might improve the readabiliy of JavaScript code substantially. 
\end{mh}



%% We conducted a non-exact replication of the three
%% studies (survey, interview, and mining software repositories)
%% considering these more specific JavaScript atom candidates.
%% We confirmed that they truly correspond to sources of misunderstanding.
%% Due to lack of space,
%% we cannot present all the results here, and we postpone the presentation of these results as future work.



