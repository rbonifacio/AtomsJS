\section{Discussion}
\label{sec:discussion}

Our work leads to several implications.
First, we generalize the
work of Gopstein et al.~\cite{DBLP:conf/sigsoft/GopsteinIYDZYC17},
which investigates
the impact of atom candidates on
understanding \clang and \cpplang code. That is,
similarly to their work, the atom candidates
Comma Operator, Post/Pre Increment, Omitted Curly Braces,
Assignment as Value, Implicit Predicate, Logic as
Control Flow, and Ternary Operator make the 
JavaScript code hard to understand. Our results also 
refute the hypothesis that Arithmetic as Logic is an atom of
confusion (i.e., a source of misunderstanding).
In comparison to the original
work of Gopstein et al.~\cite{DBLP:conf/sigsoft/GopsteinIYDZYC17}, 
our study
led to some differences in the effect size
of the atom candidates.
Altogether, we answer our first research question
\emph{\rqa}:

\begin{mh}
  {\bf Answer to RQ1:} The first study (survey) gives evidence that
  the atom candidates for \clang and \cpplang
  that also exist for JavaScript correspond to
  a source of misunderstanding in
  JavaScript code. 
\end{mh}

The results of the interview study complement the understanding
of atoms of confusion because the participants make
clear the existence of a trade-off between code comprehension
and other quality attributes. For instance, most of the participants
prefer the version of the code with the Ternary Operator,
even though they agree that its use might contribute to the
misunderstanding of JavaScript code, particularly when novices are maintaining
the codebase. The participants of the interview study also
mentioned other possible sources of misunderstanding in JavaScript,
including the use of prototype-inheritance and nested call-backs (as discussed in Section~\ref{sec:interview-results}). Other JavaScript atom candidates include
Object Destructuring, Array Spread, Object Spread, and Type Conversion.
In summary, the results of the second study (interviews) allow
us to answer the second (\rqb) and third research questions
(\emph{\rqc}).

\begin{mh}
  {\bf Answer to RQ2:} The qualitative analysis of the
  interviews supports the results of the first study,
  indicating that atoms of confusion compromise
  source code understanding. 
\end{mh}

\begin{mh}
  {\bf Answer to RQ3:} The qualitative analysis of the
  interviews suggest that specific constructs of
  JavaScript might also correspond to atoms of
  confusion, including prototype inheritance and
  nested callbacks. 
\end{mh}

The results of the third study (mining open source
Java script repositories) give evidence that,
although atoms of confusion compromise program
comprehension, they frequently appear in open
source JavaScript projects. In particular,
seven, out of 10 atoms considered
  in our study, appear in more than 50\% of
the projects. 
In summary, the third study
allows us to answer the fourth research
question (\emph{\rqd}).


\begin{mh}
  {\bf Answer to RQ4:} The MSR study reveals that
  the atoms of confusion explored in our research
  appear frequently in practice. 
\end{mh}



%% We conducted a non-exact replication of the three
%% studies (survey, interview, and mining software repositories)
%% considering these more specific JavaScript atom candidates.
%% We confirmed that they truly correspond to sources of misunderstanding.
%% Due to lack of space,
%% we cannot present all the results here, and we postpone the presentation of these results as future work.



