\section{Prevalence Study}\label{sec:s04}

%To understand how often the analyzed atom candidates appear in open source JavaScript projects, and thus answer our third research question (\emph{What is the frequency of occurrence of atoms of confusion in practice?}), we mine a set of GitHub open source repositories and compute the prevalence of atom candidates.

We mine a set of GitHub open source repositories and compute how often these atom candidates occur. We do this with the goal of quantifying the prevalence of the investigated atom candidates. \revised{If they rarely occur in real-world systems, this would suggest that cannot be a significant source of confusion for developers. This study addresses our first research question (\emph{What is the frequency of occurrence of atom candidates in practice?}).}

\subsection{Study Setting}

We first collect the most popular GitHub repositories that are primarily written in JavaScript. We measure popularity using the project's stargazers. This metric, available through the GitHub API, represents the number of stars a project received from users of the platform. The same metric has been used in a number of previous studies as a proxy to estimate projects' popularity~\cite{gyimesi2019bugsjs,canedo:esem2020}. We then select the top 100 most popular repositories and remove projects that do not reach the first quartile of the distribution of lines of code (excluding from our dataset small projects). 

After filtering out JavaScript project candidates, we build a curated dataset comprising the top \minedprojects repositories. Examples of projects in this dataset include \textsc{React}, \textsc{Node JS}, and \textsc{AngularJS}. Table~\ref{tab:projects-statistics} presents some statistics about the projects we consider in our research. The size of the projects range from small ones (5543 lines of code) to complex systems with more than 1 MLOC. All projects in our dataset have at least \num[group-separator = {,}]{1244} forks and at least \num[group-separator = {,}]{23672} stars. We automate all the steps to filter, clone, and collect the statistics from the repositories using Python scripts.

We mine the occurrence of atom candidates from the repositories in our curated dataset using source code queries that we specified using the Semmle's CodeQL language~\cite{moor:gttse2007}. CodeQL is an object-oriented variant of the Datalog language~\cite{rodriguez2020efficient}, and currently supports researchers and practitioners to query the source code of systems written in different languages (such as \cpplang, Java, and JavaScript). All CodeQL queries deal with a specific atom candidate, and collect the exact location in the source code where the atoms appear. Figure~\ref{lst:codeql} shows an example of a CodeQL query that mine occurrences of the conditional operator.

Although most of our queries are simple, we have to circumvent a few corner cases to avoid false positives. For instance, we do not consider atom candidates the pre-increment occurrences that appear within
\texttt{for statement} definitions.  We automate the process of running the queries and exporting the results to a format that simplifies our analysis (and also the reproduction of this study). Finally, we compute some descriptive statistics to measure the prevalence of atoms of confusion in practice. 

\begin{table}[ht]
  \centering
   \caption{Some descriptive statistics about the projects used in the MSR study}
%  \begin{scriptsize}
 \begin{tabular}{rrrrr}
   \toprule
                       & Min.             & Median        & Mean             & Max. \\ \midrule
 Lines of Code         & \num[group-separator = {,}]{5543}       & \num[group-separator = {,}]{36161.5} & \num[group-separator = {,}]{111432.33}  & \num[group-separator = {,}]{1278405} \\
 Num. of Forks         & \num[group-separator = {,}]{1244}       & \num[group-separator = {,}]{6078}    & \num[group-separator = {,}]{8906.24}    & \num[group-separator = {,}]{68849} \\
 Num. of Contributors  & \num[group-separator = {,}]{6}          & \num[group-separator = {,}]{285.5}   & \num[group-separator = {,}]{533.44}     & \num[group-separator = {,}]{4047} \\
 Num. of Stars         & \num[group-separator = {,}]{23672}      & \num[group-separator = {,}]{34990}   & \num[group-separator = {,}]{46919.51}   & \num[group-separator = {,}]{310935} \\
 
    \bottomrule
 \end{tabular}
 %\end{scriptsize}
 \label{tab:projects-statistics} 
\end{table}

\begin{figure}
\begin{lstlisting}[language=CodeQL]
import javascript

from ConditionalExpr e
where not e.getTopLevel().isMinified() 
select e, e.getLocation().getStartLine() as Location,e.getLocation().getFile()
\end{lstlisting}
\caption{CodeQL query that searches for Conditional Expressions.}
\label{lst:codeql}
\end{figure}

%% \begin{lstlisting}[language=CodeQL]
%% import javascript

%% from ControlStmt c
%% where not c.getTopLevel().isMinified() and
%%  (c instanceof IfStmt and not c.(IfStmt).getCondition().mayHaveBooleanValue(true)) or
%%  (c instanceof LoopStmt and  not c.(LoopStmt).getTest().mayHaveBooleanValue(true) ) or 
%%  (c instanceof WithStmt and not  c.(WithStmt).getExpr().mayHaveBooleanValue(true))  
%% select c, c.getLocation().getStartLine() as Location,c.getLocation().getFile()
%% \end{lstlisting}  

\subsection{Results}
\label{sec:msr-results} 

In this section we present the results of the  software repository
mining effort, whose goal is to reveal how prevalent atom candidates are in open source JavaScript projects. 
We mined \minedprojects open source JavaScript repositories to understand how
often atoms arise in real software. As in the interview study, this study focused on atoms that were investigated in both
experiments (detailed in Section~\ref{sec:s01} and Section~\ref{sec:s02}).  


Similarly to previous studies~\cite{DBLP:conf/msr/GopsteinZFC18,DBLP:journals/ese/MedeirosLAAKRG19} which investigate
the prevalence of atoms of confusion in open source \clang and \cpplang projects, we found that atoms of confusion
frequently arise in JavaScript open source systems. For instance, the six most frequently found atoms
occur in at least 80\% of the projects. Considering the extremes, atom candidates Conditional Operator and
Implicit Predicate were found in all repositories we mined (see examples in Figure~\ref{lst:conditional-operator-sample}
and Figure~\ref{lst:implicit-predicate-sample}), while Comma Operator occurred in only 15\% of them (as seen in Table~\ref{tab:occurrences-summary}).
Figure~\ref{lst:comma-operator-sample} shows an example of the use of the Comma Operator in the \texttt{styled-components}
project.

\begin{figure}[htb]
  \begin{lstlisting}[language=JavaScript]
function prependModifierMarker (symbol, name, dynamic) {
  return dynamic
  ? ("_p(" + name + ",\"" + symbol + "\")")
  : symbol + name // mark the event as captured
}

function concat (a, b) {
  return a ? b ? (a + ' ' + b) : a : (b || '')
}
  \end{lstlisting}
  \caption{Examples of the Conditional Operator atom present in the \texttt{vue} project (urls: shorturl.at/bJPS4 and shorturl.at/tJOS8)}
   \label{lst:conditional-operator-sample}
\end{figure}

\begin{figure}[htb]
  \begin{lstlisting}[language=JavaScript]
React.useEffect(() => {
  if (props.config.selectedLines) { // Implicit Predicate
    dispatch({
      type: 'MULTILINE',
      selectedLines: props.config.selectedLines,
    })
  }
}, [props.config.selectedLines])
  \end{lstlisting}
  \caption{Example of the Implicit Predicate atom present in the \texttt{carbon} project (url: shorturl.at/aloqU)}
  \label{lst:implicit-predicate-sample}
\end{figure}

\begin{figure}[htb]
  \begin{lstlisting}[language=JavaScript]
var props = ((_props = {}),
 (_props[SC_ATTR] = stringifyNames(names)),
 (_props[SC_VERSION_ATTR] = '4.3.1'),
 _props);
  \end{lstlisting}
  \caption{Example of the Comma Operator atom present in the \texttt{styled-components} project (url: shorturl.at/sCKM0)}
  \label{lst:comma-operator-sample}
\end{figure}

% Figure~\ref{fig:rate} summarizes this finding, showing that the occurrence of atoms of confusion in JavaScript systems range from 11.34\% (Comma Operator) to 89.69\% (Ternary Operator).


%Figure~\ref{fig:atoms-occurrence} shows that the occurrence of in JavaScript systems range from 

% latex table generated in R 4.0.3 by xtable 1.8-4 package
% Fri Nov 13 11:28:56 2020
%%\begin{table}[!htb]
%%\centering
%% \caption{Summary of atoms occurrences on our dataset}
%% 
%%\setlength\tabcolsep{2pt} % default value: 6pt
%%\label{tab:occurrences-summary}
%%\begin{tabular}{lrcc}
%%
%%  \hline
%%Atom & Projects & Occurrences/KLOC & Category \\ 
%% \hline
%%Ternary Operator & 100\% & 10.16 &  highly used \\ 
%%Omitted Curly Braces & 91.67\% & 6.61 &  highly used \\ 
%%Post Increment & 90.28\% & 5.62 &  highly used \\ 
%%Pre-Increment & 83.33\% & 1.04 & commonly used \\ 
%%Assignment as Value & 81.94\% & 1.02 & commonly used \\ 
%%Logic as Control Flow & 56.94\% & 0.95 & commonly used \\ 
%%Arithmetic as Logic & 23.61\% & 0.02 & little used \\ 
%%Comma Operator & 15.28\% & 0.03 & little used \\ 
%%   \hline
%%\end{tabular}
%%\end{table}

% latex table generated in R 4.0.3 by xtable 1.8-4 package
% Tue Apr 13 09:34:35 2021
\begin{table}[ht]
  \centering
\caption{Summary of atom candidate occurrences in our dataset.}
\label{tab:occurrences-summary}
\setlength\tabcolsep{2pt} % default value: 6pt
\begin{tabular}{lrr}%c}
  \toprule
Atom & Projects & Occurr./KLOC \\%& Frequency \\ 
  \midrule
Implicit Predicate & 100\% & 19.89 \\%&  intensively used \\ 
  Conditional Operator & 100\% & 10.16 \\%&  intensively used \\ 
  Omitted Curly Braces & 91.67\% & 6.61 \\%&  intensively used \\ 
  Post Increment & 90.28\% & 5.62 \\%&  intensively used \\ 
  Pre Increment & 83.33\% & 1.04 \\%& commonly used \\ 
  Assignment as Value & 81.94\% & 1.02 \\%& commonly used \\ 
  Logic as Control Flow & 56.94\% & 0.95 \\%& commonly used \\ 
  Automatic Semicolon Insertion & 33.33\% & 0.13 \\%& commonly used \\ 
  Arithmetic as Logic & 23.61\% & 0.02 \\%& rarely used \\ 
  Comma Operator & 15.28\% & 0.03 \\%& rarely used \\
   \bottomrule
\end{tabular}
  
\end{table}


Considering all JavaScript projects in our dataset, we found a total of \num[group-separator = {,}]{364873} atom candidates, though four atoms are responsible
for 92.97\% of this total: Implicit Predicate, Post Increment, Conditional Operator, and Omitted Curly Braces. 
%The first two, based on the results of Section~\ref{sec:repeated} and Section~\ref{sec:latin}, can be considered atoms of confusion. 
The remaining atom
candidates comprise \num[group-separator = {,}]{25621} occurrences combined, 7.03\% of the total number of occurrences.
The Comma Operator and Arithmetic as Logic atoms are the ones that arise less
frequently (0.001\% in total with 232 and 165 occurrences, respectively). 

% Regarding frequency, we proceeded with a classification of \emph{atom occurrence}
% based on thresholds used in a previous
% study~\cite{DBLP:journals/ese/MedeirosLAAKRG19}. 
% Atoms that occur 1000 times are classified as \textit{rarely used} while atoms with more than \num[group-separator = {,}]{10000} occurrences are classified as \textit{highly used}; \textit{commonly used} atoms occur in between.
We calculate the frequency of each atom per KLOC.
The four atoms mentioned in the previous paragraph,
Implicit Predicate, Post Increment, Conditional Operator, and Omitted Curly Braces, occur frequently in the analyzed projects. All of them have more than 5 occurrences per KLOC on average. In particular, two atom candidates occur very frequently, Implicit Predicate (19.89 occurrences per KLOC) and Conditional Operator (10.16 occurrences per KLOC). On the other hand, Arithmetic as Logic and Comma Operator occur less than once for every 50 KLOC. 

%In our dataset of JavaScript projects, 
%We found the Ternary Operator atom occurring more frequently than the Omitted Curly Braces, differently from what has been reported in a previous work~\cite{DBLP:conf/msr/GopsteinZFC18}. %




%%%%%%%%%%%%%%%%%%%%%% MAYBE the following can go to a discussion section. 
%%%%%%%%%%%%%%%%%%%%%% 
%The first two, based on the results of Section~\ref{sec:repeated} and Section~\ref{sec:latin}, can be considered atoms of confusion. 

%The results of our experiments and interviews suggest that Conditional Operator does not contribute significantly as a source of misunderstanding (increasing the number of wrong answers in 9\% of the cases, according to the Latin-square study). Nonetheless, the Post-Increment Expression and Automatic Semicolon Insertion atoms are listed in the top four sources of misunderstanding  (Table~\ref{tab:difference-correctness}), and also frequently appear in open source systems. As such, refactoring existing systems to avoid Post-Increment and Automatic Semicolon Insertion might improve the readability of large amounts of code in JavaScript projects. 




 
