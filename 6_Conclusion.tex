\section{Conclusions}
\label{conclusion}

This paper has reported the results of a mixed-method research effort to better understand the impact of atoms of confusion in JavaScript code. First, we conducted a survey with 140 JavaScript developers. We asked them to predict the output of some code snippets (with and without atoms). We provide evidences that seven atom candidates have a significant impact on program comprehension activities.

Afterwards, we interviewed 15 professional developers. In general, they considered code without atoms of confusion easier to understand and highlighted other JavaScript constructs and idioms that might also introduce misunderstanding. Last but not least, we conducted a mining software repositories study. In particular, we found four atoms of confusion highly used in JavaScript code, i.e., Implicit Predicate, Ternary Operator, Omitted Curly Braces, and Post Increment.

Our efforts have important implications for practice. First, we alert developers to avoid writing JavaScript code with certain atoms of confusion (e.g., Comma Operator, Automatic Semicolon Insertion, Post Increment, and Omitted Curly Braces). Second, our results might help tool developers to create program transformation tools to remove atoms of confusion automaticallu. They might firstly focus, for instance, on the atoms we found to be more confusing and at the same time more common in practice (e.g., Omitted Curly Braces and Post Increment).

As mentioned, the results of the interviews pointed out additional JavaScript constructs and idioms
that might introduce misunderstanding. These include not only nested callbacks and prototype-based
inheritance (which might not have a counterpart version), but also idioms such
as \emph{property access} and \emph{arrow functions}. As future work, we intend to
reproduce our study to validate whether or not these \emph{atom candidates} affect
the understanding of JavaScript code. 

