\section{Conclusions}
\label{conclusion}

This paper introduced a mixed-method research effort to better understand the impact of atoms of confusion in JavaScript code. First, we conducted a survey with 140 JavaScript developers. We asked them to predict the output of some code snippets (with and without atoms). We provide evidences that, even though the atoms of confusion we considered do not take a much higher amount of time to predict the output of the code, they might still have a significant impact on program comprehension activities in almost all of the cases we studied. Afterwards, we interviewed 15 professional developers. In general, they considered code without atoms of confusion easier to understand. Last but not least, we conducted a mining software repositories study. In particular, we found four atoms of confusion highly used in JavaScript code, i.e., Implicit Predicate, Ternary Operator, Omitted Curly Braces, and Post Increment.

Our efforts have important implications for practice. First, we alert developers to avoid writing JavaScript code with certain atoms of confusion ({\color{red} \textbf{EXEMPLO AQUI???}).} Second, we might help tool developers to create a program transformation tool to remove automatically atoms of confusion. They might firstly focus, for instance, on the atoms we found to be more confusing and at the same time more common in practice.

%\rb{draft. apenas movi esse texto que estava em settings para ca. pode servir de inspiracao para responder algumas das questoes de pesquisa.}

%{\color{blue}Based on these results, we provide evidence that, although the atoms of confusion considered in our study do not take a much higher amount of time to predict the output of the code, their impact on program comprehension is, in almost all of our cases, highly significant.}