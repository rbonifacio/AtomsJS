\section{Conclusions}
\label{conclusion}

This paper reports the results of a mixed-method research effort to better understand the impact of atoms of confusion in JavaScript code. First, we conducted a survey with 140 JavaScript developers, asking them to predict the output of some code snippets (with and without atoms). We provide evidence that five atom candidates significantly impact program comprehension activities and can be considered atoms of confusion: Comma Operator, Automatic Semicolon Insertion, Post Increment, Assignment as Value, and Implicit Predicate. 
We also interviewed 15 professional developers. In general, they considered code without atom candidates easier to understand and highlighted other JavaScript constructs and idioms that might also introduce misunderstanding. 
%Last, but not least,
Finally, we conducted a mining software repositories study. We found four atom candidates frequently used in JavaScript code, i.e., Implicit Predicate, Post Increment, Ternary Operator, and Omitted Curly Braces.  %The first two can be considered atoms of confusion. 

Our efforts have important implications for practice. First, we alert developers to avoid writing JavaScript code with certain atoms of confusion (e.g., Comma Operator, Automatic Semicolon Insertion, Post Increment, Assignment as Value, and Implicit Predicate). Second, our results might help tool developers to create program transformation tools to remove the atoms automatically. 
%They might firstly focus, for instance, on the atoms we found to be more confusing and at the same time more common in practice (e.g., Omitted Curly Braces and Post Increment).
%Promising candidates may be the atoms we found to be most confusing and, at the same time, common in practice (e.g., Implicit Predicate and Post Increment).

%As mentioned, the results of the interviews
The results of the interviews pointed out additional JavaScript constructs and idioms 
that might introduce misunderstanding. These include nested callbacks, % and prototype-based inheritance (which might not have a counterpart version), but also idioms such
%as 
\emph{property access}, and \emph{arrow functions}. As future work, we intend to
reproduce our survey to validate whether or not these additional
\emph{atom candidates}
%, which the interviewees pointed out, 
truly affect
the understanding of JavaScript code. 

