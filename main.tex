\documentclass[10pt,conference,blind]{IEEEtran}

\IEEEoverridecommandlockouts

% The preceding line is only needed to identify funding in the first footnote. If that is unneeded, please comment it out.




\usepackage{enumerate}
\usepackage{balance}
\usepackage{caption}
\usepackage{cite}
\usepackage{amsmath,amssymb,amsfonts}
\usepackage{algorithmic}
\usepackage{graphicx}
\usepackage{supertabular}
\usepackage{textcomp}
\usepackage{xcolor}
\usepackage{colortbl}
\def\BibTeX{{\rm B\kern-.05em{\sc i\kern-.025em b}\kern-.08em
    T\kern-.1667em\lower.7ex\hbox{E}\kern-.125emX}}
\usepackage{listings}
\usepackage{color}
\usepackage{scalefnt}
\usepackage{multirow}
\usepackage{pifont}
\usepackage{xspace}
\usepackage{todonotes}
\usepackage{booktabs}
\usepackage{longtable}
\usepackage{tabularx} 
\usepackage{multicol}
\usepackage{listings}
\usepackage{xspace}
\usepackage{booktabs}
\usepackage{siunitx}
\usepackage{amsmath,amssymb,amsfonts}
\usepackage{algorithmic}
\usepackage{graphicx}
\usepackage{graphics}
\usepackage{textcomp}
\usepackage{siunitx}
\usepackage{xcolor}
\usepackage{csquotes}
\usepackage{array}
\usepackage{listings}
\usepackage{color}
\usepackage{scalefnt}
\usepackage{multirow}
\usepackage{pifont}
\usepackage{xurl}
% \usepackage{appendix}
\usepackage[framemethod=TikZ]{mdframed}
\definecolor{oldlace}{rgb}{0.99, 0.96, 0.9}
\newmdenv [ %
 skipabove=\topsep,
 skipbelow=\topsep,
 leftmargin       = 2,
 rightmargin      = 2,
 splittopskip     = \topskip]{mh}

\newmdenv [ %
 skipabove=\topsep,
 skipbelow=\topsep,
 roundcorner = 5pt,
 leftmargin = 2,
 rightmargin = 2,
 % backgroundcolor = oldlace,
 innertopmargin = 3,
 splittopskip = 3]{mq}

\definecolor{javared}{rgb}{0.6,0,0} % for strings
\definecolor{javagreen}{rgb}{0.25,0.5,0.35} % comments
\definecolor{javapurple}{rgb}{0.5,0,0.35} % keywords
\definecolor{javadocblue}{rgb}{0.25,0.35,0.75} % javadoc
\lstset{
  basicstyle=\footnotesize\tt,        % the size of the fonts that are used for the code
  breakatwhitespace=false,         % sets if automatic breaks should only happen at whitespace
  breaklines=true,                 % sets automatic line breaking
  captionpos=b,                    % sets the caption-position to bottom
  extendedchars=true,              % lets you use non-ASCII characters; for 8-bits encodings only, does not work with UTF-8
  frame=single,                    % adds a frame around the code
  language=Java,                 % the language of the code
  keywordstyle=\bf,
  showspaces=false,                % show spaces everywhere adding particular underscores; it overrides 'showstringspaces'
  showstringspaces=false,          % underline spaces within strings only
  showtabs=false,                  % show tabs within strings adding particular underscores
  tabsize=2,                       % sets default tabsize to 2 spaces
  keywordstyle=\color{javapurple}\bfseries,
  stringstyle=\color{javared},
  commentstyle=\color{javagreen},
  morecomment=[s][\color{javadocblue}]{/**}{*/},
  numbers=left,
  numberstyle=\tiny\color{black},
  stepnumber=1,
  keywords={let},
}    
\AtBeginDocument{%
  \providecommand\BibTeX{{%
    \normalfont B\kern-0.5em{\scshape i\kern-0.25em b}\kern-0.8em\TeX}}}

\newcommand{\clang}{C\nolinebreak\xspace}

\newcommand{\cpplang}{C\nolinebreak\hspace{-.05em}\raisebox{.4ex}{\small\bf +}\nolinebreak\hspace{-.10em}\raisebox{.4ex}{\small\bf +}\xspace}

% TODO: Completar
\newcommand{\minedprojects}{72\xspace}

\newcommand{\rb}[1]{\todo[inline]{{\bf (rb note) }#1}}
\newcommand{\diego}[1]{\todo[inline,color=green]{{\bf (diego note) }#1}}
\newcommand{\adriano}[1]{\todo[inline,color=yellow]{{\bf (adriano note) }#1}}
\newcommand{\castor}[1]{\todo[inline,color=pink]{{\bf (castor note) }#1}}

\newcommand{\rqa}{What is the impact of atoms of confusion on the comprehension of
  JavaScript code?}

\newcommand{\rqb}{Do JavaScript developers identify atoms of confusion as contributing to program misunderstanding?}

\newcommand{\rqc}{What are the particular JavaScript idioms and language constructs that might cause source code misunderstanding?}

\newcommand{\rqd}{What is the frequency of occurrence of atoms of confusion in practice (i.e., in open-source JavaScript projects)?}


\lstset{language=C}
\lstset{
 morekeywords={printf}
}

\newcommand{\lhs}{left-hand side\xspace}
\newcommand{\rhs}{right-hand side\xspace}

\lstdefinelanguage{JavaScript}[]{Java}{
  morekeywords={let, console.log},
  moredelim=[is][\textcolor{darkgray}]{\%\%}{\%\%},
  moredelim=[il][\textcolor{darkgray}]{§§}
}

\newcommand{\na}{10\xspace}

\begin{document}
\newenvironment{atom}[1]
  {\mdfsetup{
    frametitle={\colorbox{white}{\space#1\space}},
    innertopmargin=10pt,
    frametitleaboveskip=-\ht\strutbox,
    frametitlealignment=\center
    }
  \begin{mdframed}%
  }
  {\end{mdframed}}

\title{On The Impact of Atoms of Confusion in JavaScript
Code}

\author{\IEEEauthorblockN{Anonymous}}


%% \author{\IEEEauthorblockN{1\textsuperscript{st} Given Name Surname}
%% \IEEEauthorblockA{\textit{dept. name of organization (of Aff.)} \\
%% \textit{name of organization (of Aff.)}\\
%% City, Country \\
%% email address or ORCID}
%% \and
%% \IEEEauthorblockN{2\textsuperscript{nd} Given Name Surname}
%% \IEEEauthorblockA{\textit{dept. name of organization (of Aff.)} \\
%% \textit{name of organization (of Aff.)}\\
%% City, Country \\
%% email address or ORCID}
%% \and
%% \IEEEauthorblockN{3\textsuperscript{rd} Given Name Surname}
%% \IEEEauthorblockA{\textit{dept. name of organization (of Aff.)} \\
%% \textit{name of organization (of Aff.)}\\
%% City, Country \\
%% email address or ORCID}
%% \and
%% \IEEEauthorblockN{4\textsuperscript{th} Given Name Surname}
%% \IEEEauthorblockA{\textit{dept. name of organization (of Aff.)} \\
%% \textit{name of organization (of Aff.)}\\
%% City, Country \\
%% email address or ORCID}
%% \and
%% \IEEEauthorblockN{5\textsuperscript{th} Given Name Surname}
%% \IEEEauthorblockA{\textit{dept. name of organization (of Aff.)} \\
%% \textit{name of organization (of Aff.)}\\
%% City, Country \\
%% email address or ORCID}
%% \and
%% \IEEEauthorblockN{6\textsuperscript{th} Given Name Surname}
%% \IEEEauthorblockA{\textit{dept. name of organization (of Aff.)} \\
%% \textit{name of organization (of Aff.)}\\
%% City, Country \\
%% email address or ORCID}
%% }

\maketitle

% Número das páginas. Remover para camera-ready
% útil para nós e Program Chair que pode ver o total de páginas facilmente
\thispagestyle{plain}
\pagestyle{plain}

\begin{abstract}
  Evolving legacy code is a challenging task, particularly when the code has been poorly written or uses confuse idioms and language constructs, which might increase maintenance efforts and impose a significant cognitive load on developers. For this reason, researchers have investigated possible sources of confusion in codebases, including the impact of small code patterns (hereafter atoms of confusion) that contribute to misunderstanding the source code written in statically typed languages such as \clang, \cpplang, and Java. In this work, we investigate whether atoms of confusion identified in statically typed languages also confuse developers of a dynamically typed language. We use JavaScript as a representative example of dynamic programming languages and collect evidence from a mixed-method research effort: a survey, a set of interviews with practitioners, and an activity of mining open source JavaScript repositories (MSR). Our survey and interviews confirm that atoms of confusion lead to code that is hard to understand in JavaScript. Considering ten atom candidates, developers correctly predict the outcome of at least 15\% more cases in code snippets where atom candidates are not present, compared to alternative versions that include the atom candidates. For five of these atoms, the difference is statistically significant (accounting for p-value correction). Effect size is large for four atoms and medium for one of them. In addition, our MSR effort reveals that atom candidates are frequent and used intensively in 72 popular open-source JavaScript systems. Four atom candidates appear in 90\% of the analyzed projects, and two of them occur more than once for every 100 lines of code in the dataset. Altogether, our findings might help practitioners: (1) better understand the implications of atoms of confusion on understanding JavaScript code, (2) avoid writing code that is unnecessarily difficult to maintain, and (3) design program transformation tools that remove potential sources of misunderstanding in JavaScript code.
\end{abstract}

\begin{IEEEkeywords}
code readability, program comprehension, program understanding, atoms of confusion, JavaScript code
%datasets, neural networks, gaze detection, text tagging
\end{IEEEkeywords}

\section{Introduction}
\label{intro}

% Source code misunderstandings
% Recent work... For instance, consider the pair...
% However, focus on C, C++ and Java.
% In this paper ==> Javascript!



%%%%%%%%%%%%%%%%%%%%%%%%%%%%%
% INICIO INTRODUCAO - MARCIO
%%%%%%%%%%%%%%%%%%%%%%%%%%%%%
%% Understanding code is an expensive and important task. In real-world software development, it is commonplace for developers to spend more time understanding code than actually writing it~\cite{CitationRequired}.Figuring out what parts of an existing system do by reading its source code is necessary for important software development tasks such as testing, debugging, and adding new functionality to a system. 

Developers are often confused while reading unfamiliar code~\cite{Ebert:2021:ESC}. According to Hermans~\cite{ProgrammersBrain}, that confusion can stem from three sources: (i) lack of knowledge, i.e., not knowing what an element in the program does; (ii) lack of information, i.e., not knowing how a program element works; and (iii) lack of processing power, i.e., the inability to combine all the elements in a program and make sense of its execution in one's head. Even small pieces of code can be confusing for developers~\cite{Ajami:2017:SPI,DBLP:conf/sigsoft/GopsteinIYDZYC17} at first glance.

For instance, consider the pair of JavaScript code snippets in Listing~\ref{fig:lst01} (adapted from \textsc{NervJS/taro}). The code snippet on the left-hand side is arguably harder to understand than the one on the right-hand side. The former uses the logical AND operator (\texttt{\&\&}) beyond its lexical meaning~\cite{castor2018} to determine the control flow of the program. Furthermore, the expression has a potential side effect, due to the use of the post-increment operator (\texttt{s++}). The snippet on the right-hand side uses an \texttt{if} statement for control flow and increments the value of variable \texttt{s} using an assignment, making it easier to understand. Notwithstanding, confusion is a direct consequence of the experience of the person who is reading the code. A developer who customarily employs the idiom on the left-hand side may find it easier to understand. 

\begin{figure*}[thb]
\noindent\begin{minipage}{.48\textwidth}  
\begin{lstlisting}[language=JavaScript]
res = ps.reduce((s, b) => {
  b && s++
  return s
}
\end{lstlisting}
\end{minipage}\hfill
\begin{minipage}{.48\textwidth}
\begin{lstlisting}[language=JavaScript]
res = ps.reduce((s, b) => {
  if(b) {
    s = s + 1
  }
  return s
}
\end{lstlisting}
\end{minipage}
\captionof{lstlisting}{Code snippet from \textsc{NervJS/taro} project (left-hand side) and an alternative implementation (right-hand size).}\label{fig:lst01}
% On the right-hand side, we present an alternative implementation without atoms of confusion. We could simplify the code on the left-hand side even more, by using \texttt{filter} and \texttt{length} instead of the \texttt{reduce} recursive pattern (e.g., \texttt{res = properties.filter(p => p.length})).}
\end{figure*}

%Line 2 on the left-hand side of the figure contains two atoms of confusion discussed in ~\cite{DBLP:conf/sigsoft/GopsteinIYDZYC17}: Logic as Control Flow and Post-increment. 

Recent work~\cite{DBLP:journals/ese/MedeirosLAAKRG19,DBLP:conf/sigsoft/GopsteinIYDZYC17,Langhout:2021:ACJ,TheEyesDoNotLie} has attempted to elicit and catalog simple code patterns and language constructs that tend to confuse developers when reading them and for which there are less confusing, functionally-equivalent alternatives. These confusing constructs and patterns are called ``atoms of confusion''~\cite{DBLP:conf/sigsoft/GopsteinIYDZYC17} when it is possible to experimentally ascertain that there is a less confusing alternative (otherwise they are called ``atom candidates''). The snippet on the left-hand side of Figure~\ref{fig:lst01} contains two atoms of confusion discussed in previous work~\cite{DBLP:conf/sigsoft/GopsteinIYDZYC17} focusing on the \clang language: Logic as Control Flow and Post-increment. Previous work has identified atoms of confusion in two languages, \clang and Java. According to the studies of Gopstein et al.~\cite{DBLP:conf/sigsoft/GopsteinIYDZYC17}, removing the atoms Logic as Control Flow and Post-increment from small \clang code snippets improved the ability of study participants to predict their outcomes by 41\% and 34\%, respectively. For small Java code snippets, Langhout and Aniche~\cite{Langhout:2021:ACJ} report improvements of 53\% and 46.27\%, respectively. These results show that atoms of confusion may be \textbf{challenging} for developers using these languages. In addition, a study~\cite{DBLP:conf/msr/GopsteinZFC18} with 14 large-scale open source projects written in \clang found that 4.38\% of the lines of code have an atom and their presence has a strong correlation with bug fixing commits and long code comments. These results show that atoms of confusion are \textbf{prevalent} in large systems. They are also \textbf{relevant}, even for experienced software developers. 
%According to the experiments of Gopstein et al.~\cite{DBLP:conf/sigsoft/GopsteinIYDZYC17}, removing the atoms Logic as Control Flow and Post-increment improved the accuracy of program understanding tasks in 41\% and 34\%, respectively.

%factors such as the wrong choice of names for variables and methods~\cite{avidan:icpc2017} and the use of uncommon language constructs~\cite{VEM2018} may confused developers might hinder developers from understanding a piece of code. Besides that, programming languages can also have specific idioms and constructs that may lead to misunderstandings. That is the case of atoms of confusion: small pieces of code that prevent developers from program comprehension, even though functionally equivalent alternatives exist and are easier to understand~\cite{DBLP:conf/sigsoft/GopsteinIYDZYC17, DBLP:conf/msr/GopsteinZFC18}.


The goal of this paper is twofold. First, it aims to investigate whether atoms of confusion that were identified in Java and C also cause confusion to JavaScript developers. The latter, in spite of the syntactic similarities to Java, is a dynamically-typed language that is arguably more related to Scheme than it is to Java~\cite{Eich:2018:BHJ}. 
% This way, there is a knowledge gap with respect to the impact of such atoms in dynamic languages such as JavaScript. In this sense, because 
JavaScript's programming culture differs from that of languages such as \clang and Java, particularly due to its dynamic capabilities and weaker type system. At the same time, the language is not so different that previously identified atoms cannot be represented in it. Second, the paper aims to verify whether other code patterns, not investigated in previous work, also tend to confuse JavaScript developers. Some of these code patterns are specific to JavaScript, e.g., object destructuring. 
% \castor{The previous sentence is really bad.} 
%Second, we want to analyze whether these atoms also cause confusion when we control for developer experience. As mentioned before, experience seems to play a significant role in whether a certain construct or code pattern can be considered an atom of confusion or not. Gopstein et al.~\cite{DBLP:conf/sigsoft/GopsteinIYDZYC17} report that, in their studies, experience is positively correlated with better results. In this paper, we aim to use an experimental design that controls for experience, so as to have greater confidence in our results. 
%, we may find that idioms and constructs that are atoms of confusion in those languages are not particularly difficult to understand for JavaScript programmers and vice-versa. 
%Also, besides being the most widely used language for web development and the most often adopted language on GitHub~\cite{BugsJS:ICST}, 
%JavaScript contains features that combined make the source code fertile ground for errors~\cite{BugsJS:ICST,ClientSideBugsJS:TSE} and confusion~\cite{JSGoodParts:book}.\castor{I think we should only say this if we explicitly point out examples of such features}. 
%, such as weakly typed language~\cite{DynamicBehaviorJS:PLDI}, object-oriented, functional,
% and imperative constructs~\cite{JS-CodeSmells:SCAM}, and features such as run-time
% evaluation~\cite{BugsJS:ICST,ClientSideBugsJS:TSE}, and asynchronous operations.

We present the results of a mixed-method research effort based on two experiments, a set of interviews, and a mining software repositories study. The two experiments ask the participants to predict the output of small code snippets, where some of them have atom candidates and others do not. These two experiments were designed and conducted independently, by different researchers, with different samples, and complementary methodologies; one uses a repeated measures design~\cite{Keselman:2001:ARM} whereas the other uses a Latin square design~\cite{Hunter-Experimenters}.  
Both our experiments suggest that two code patterns that have been previously observed to confuse C programmers also confuse JavaScript programmers (with statistical significance, after correction): the comma operator and assignments being used as values. In addition, pre- and post-increment operators, when used as values, omitted curly braces, changes of literal encoding, and type conversions have led to confusion in at least one of the two experiments. For all these cases effect sizes were either medium or high. In addition, we have discovered that object destructuring and automatic semicolon insertion are potential JavaScript-specific atoms that have not been discussed in previous work.

To learn more about the misunderstandings in the snippets used in the experiments, we also interviewed 15 experienced professional developers. Participants preferred the code snippets without the atoms in 70\% of the cases. The frequency with which the atoms appear \emph{in the wild} might reveal cases where developers should be more or less careful when writing JavaScript code. To measure the prevalence of the atoms, we mined popular JavaScript projects from GitHub looking for the atom candidates that were investigated in both experiments. We found out 
%We found out that atoms frequently appear in practice. Finally, 
%our MSR effort reveals 
that atom candidates are frequent and used intensively in 72 popular open-source JavaScript systems.

%% The number of correct answers for the \emph{clean version} of the code snippets for 7 out of \na atom candidates is at least 15\% higher than the corresponding \emph{confusing version} of the code snippet (i.e., with an atom candidate). For 5 of these atoms, the difference is statistically significant (accounting for p-value correction). Effect size is large for 4 atoms and medium for 1 of them. 
%% The presence of an atom candidate might also affect the time necessary for submitting a correct answer. For instance, the presence of the Comma Operator atom increases the average time to predict the output by 76.23\%, though the outputs of snippets containing the Pre-Increment atom were correctly predicted 38.19\% faster.
%% To learn more about the misunderstandings in the snippets used in the survey, we also interviewed 15 experienced professional developers. Participants preferred the code snippets without the atoms in 70\% of the cases, therefore corroborating the survey findings. The participants also discussed other idioms and constructs that might lead developers to misunderstand JavaScript code.

%% The frequency in which the atoms appear \emph{in the wild} might reveal cases where developers should be more or less careful when writing JavaScript code. To understand the incidence of atoms, we mined popular JavaScript projects from GitHub and found that atoms frequently appear in practice. 
%% Our MSR effort reveals that atom candidates are frequent and used intensively in 72 popular open-source JavaScript systems. Four atom candidates appear in 90\% of the analyzed projects and two of them, Implicit Predicate and Ternary Operator, occur more than once for every 100 lines of code in the dataset. 
%Indeed, four atom candidates are frequently used in JavaScript code---the Implicit Predicate atom, for instance, occurs 19.89 times per 1,000 lines of code.

%\castor{I would rather remove the following. We don't even talk about the "library and infrastructure".}
%Altogether, the main contributions of this paper are:

%\begin{itemize}
    
%\item A mixed-method research effort---based on two experiments, interviews, and a software
%  repository mining study---about the impact of atoms of confusion on JavaScript programs.
  % (a dynamic language), controlling for subject experience.

%  \item A comparison of the impact and prevalence of atoms of confusion in
%    JavaScript with what has been previously reported for other languages
%    (\clang, \cpplang, and Java).
%    \item A library and infrastructure to mine atoms of confusion using 
%    code queries (implemented in the .QL language). This library and 
%    infrastructure are publicly available and might help researchers and 
%    practitioners to automate program analysis at a large scale.

%\end{itemize}

%As implications, our results can help researchers and practitioners to better understand atoms of confusion in JavaScript. For example, our findings might alert the JavaScript community to avoid writing code with certain atoms of confusion. Moreover, our mixed study might help researchers and practitioners to design program transformation tools to remove atoms of confusion in JavaScript.

%They might prioritize atoms that cause a lot of confusion and are very common in the repositories, for instance.

%%%%%%%%%%%%%%%%%%%%%%%%%%%%%
% FINAL INTRODUCAO - MARCIO
%%%%%%%%%%%%%%%%%%%%%%%%%%%%%





%Here we investigate a group of atom candidates that stems from the study of Gopstein and colleagues and detail additional JavaScript constructs that might introduce misunderstanding---according to the opinions of JavaScript developers.

%%%%%%%%%%%%%%%%%%%%%%%%%%%%%%%%%%%%%

% The source code of a program can be naively understood as a collection of sentences in a language that would be interpreted by a computer (or compiled into a lower level representation and then executed by a computer). Nonetheless, this understanding is far from being complete. While it is not possible to execute programs without writing them in a way that the machine is able to interpret, it is also very difficult to develop software that performs complex tasks if developers do not care about how easily other programmers are going to understand its source code \cite{DBLP:journals/cj/Knuth84}.

% \rb{acho que esse proximo paragrafo pode ser excluido}
% \diego{também acho}

% {\color{blue}That means programming should, in various ways, be regarded as an act of communication. In this context, the programmer is often faced with a dilemma. When initially confronted with a problem they have to solve, the programmer will proceed to employ their cognition into developing the logic of the solution, whilst simultaneously having to worry about writing ``syntactically correct'' code for the machine to parse and translate. The problem is that, during the time spent solving the problem and communicating with the computer, developers often do not have enough cognitive resources to also make considerations about whether their code is understandable by other programmers. Therefore, it takes a change of track to make the code easier to understand \cite{DBLP:books/daglib/0019908}. Not proceeding with such change of track is one of the reasons why ``confusing'' language constructs and idioms arise and remain in a program.}

% ESSE PARAGRAFO ESTAVA NA VERSAO ANTIGA
%Source code misunderstandings can occur due to diverse reasons. For instance, a poorly understood requirement might lead to poorly implemented code. An intricate algorithm may require extra time to be fully comprehended. Besides that, programming languages can also be at fault and specific syntax and semantics of a language can cause misunderstandings. The same applies to some code idioms and stylistic choices~\cite{DBLP:conf/msr/GopsteinZFC18}.

% \diego{acho que o prox. parágrafo deve ser adaptado ou até removido}
% {\color{blue}A direct consequence of the presence of these idioms in the code base is that it becomes even more difficult to develop new features for the software. 
% According to Gopstein et al. \cite{DBLP:conf/sigsoft/GopsteinIYDZYC17} ``\emph{the ability to understand pre-existing source code is one of the most important elements of a continuously successful software project}'', it follows that, if one cannot properly understand the \emph{purpose} and \emph{meaning} of a piece of code, they are not going to be able to build upon its existing parts.
% Other consequences of the presence of confusing code include: 
% (a) increased probability of bug introduction, due to a lack of proper understanding of what the program does; 
% and (b) loss of internal quality measures, such as cohesion and coupling. 
% This phenomenon usually leads to an architectural erosion. 
% All these situations, which usually compound each other, expose the necessity of developing recommendations, tools, and techniques that help developers to avoid writing confusing code, identify potentially confusing code, and transform confusing code into clear code. 
% Therefore, it is necessary to understand what programming language constructs and idioms might lead to code that is hard to understand.}

%This research focuses on existing JavaScript idioms that a previous research work have proved to introduce misunderstanding in \clang.

% ESSE PARAGRAFO ESTAVA NA VERSAO ANTIGA
%JavaScript is the most widely used language for web development and the most often adopted language on GitHub~\cite{BugsJS:ICST}. At the same time, JavaScript is infamously known to be ``the worst most popular language in the world''.\footnote{\url{https://github.com/getify/You-Dont-Know-JS/blob/2nd-ed/preface.md}} It is a weakly typed language~\cite{DynamicBehaviorJS:PLDI} that provides object-oriented, functional, and imperative constructs~\cite{JS-CodeSmells:SCAM}, in addition to features such as run-time evaluation~\cite{BugsJS:ICST,ClientSideBugsJS:TSE}, and asynchronous operations. All these features combined make JavaScript source code fertile ground for errors~\cite{BugsJS:ICST,ClientSideBugsJS:TSE} and confusion~\cite{JSGoodParts:book}.

% ESSE PARAGRAFO ESTAVA NA VERSAO ANTIGA
%Our research aims to help practitioners identify potentially confusing snippets of code and alternative snippets that can make the code clearer. To achieve that, we investigate JavaScript idioms that might negatively impact program understanding. These idioms correspond to a set of candidates for \emph{atoms of confusion} in JavaScript---that is, small pieces of code that hinder developers' understanding of the source code of a program for which there are functionally equivalent alternatives that are easier to understand~\cite{DBLP:conf/sigsoft/GopsteinIYDZYC17}. For instance, consider the pair of code snippets in Listing~\ref{fig:lst01} (adapted from \textsc{NervJS/taro}). The code on the left-hand side of the figure contains two atoms of confusion discussed in previous work focusing on the \clang language~\cite{DBLP:conf/sigsoft/GopsteinIYDZYC17}: \emph{Logic as Control Flow} and \emph{Post-increment}. On the right-hand side, we present a clean version of the code, that is, without atoms of confusion. In the experiments of Gopstein et al.~\cite{DBLP:conf/sigsoft/GopsteinIYDZYC17}, removing the atoms \emph{Logic as Control Flow} and \emph{Post-increment} improved the accuracy of program understanding tasks in  41\% and 34\%, respectively.

% ESSE PARAGRAFO ESTAVA NA VERSAO ANTIGA
%Recent work~\cite{DBLP:journals/ese/MedeirosLAAKRG19,DBLP:conf/sigsoft/GopsteinIYDZYC17,Langhout:2021:ACJ} has explored the impact of atoms of confusion on program understanding, and confirmed they negatively impact source code readability. However, these previous studies focus exclusively on statically typed languages (e.g., \clang, \cpplang, and Java). However, JavaScript's programming culture differs from that of languages such as \clang and \cpplang, particularly due to the JavaScript's dynamic capabilities and weaker type system. Thus, we may find that idioms and constructs that are atoms of confusion in those languages are not particularly difficult to understand for JavaScript programmers and vice-versa. Here we investigate a group of atom candidates that stems from the study of Gopstein and colleagues and detail additional JavaScript constructs that might introduce misunderstanding---according to the opinions of JavaScript developers. 

% \begin{figure*}[thb]
% \noindent\begin{minipage}{.45\textwidth}  
% \begin{lstlisting}[language=JavaScript]
% res = properties.reduce((s, b) => {
%   b && s++
%   return s
% }
% \end{lstlisting}
% \end{minipage}\hfill
% \begin{minipage}{.45\textwidth}
% \begin{lstlisting}[language=JavaScript]
% res = properties.reduce((s, b) => {
%   if(b) {
%     s = s + 1
%   }
%   return s
% }
% \end{lstlisting}
% \end{minipage}
% \captionof{lstlisting}{Code snippet from \textsc{NervJS/taro} project. Line 2 on the left-hand side of the figure contains two
%   atoms of confusion discussed in ~\cite{DBLP:conf/sigsoft/GopsteinIYDZYC17}: \emph{Logic
%   as Control Flow} and \emph{Post-increment}. On the right-hand side, we present an alternative implementation without atoms of confusion. We could simplify the code on the left-hand side even more, by using \texttt{filter} and \texttt{length} instead of the \texttt{reduce} recursive pattern (e.g., \texttt{res = properties.filter(p => p).length}).}\label{fig:lst01}
% \end{figure*}

% ESSE PARAGRAFO ESTAVA NA VERSAO ANTIGA
%To understand whether an atom of confusion introduces misunderstandings, we surveyed 140 JavaScript developers. In the survey,  their task was to predict the output of short code snippets, with half of them containing atoms of confusion. {\color{red}The number of correct answers for the \emph{clean version} of the code snippets for 7 out of \na atom candidates is at least 15\% higher than the corresponding \emph{confusing version} of the code snippet (i.e., with an atom candidate). The presence of an atom candidate might also affect the time necessary for submitting a correct answer. For instance, the presence of the Comma Operator atom increases  the average time to predict the output by 76.23\%, though the outputs of snippets containing the Pre-Increment atom were correctly predicted 38.19\% faster.

% ESSE PARAGRAFO ESTAVA NA VERSAO ANTIGA
%To learn more about the misunderstandings in the snippets used in the survey, we interviewed 15 experienced professional developers. Participants considered the code snippets without the atom candidates easier to understand in 70\% of the cases, therefore corroborating the survey findings. The participants also discussed other idioms that might lead developers to misunderstand JavaScript code. The occurrence of atoms in open-source repositories might reveal cases where developers should be more or less careful when writing code---and open the possibility to implement new program transformations for cleaning up confusing code. To identify such cases, we mined popular JavaScript projects from GitHub and found four atom candidates that are highly used in JavaScript code. The Implicit Predicate atom, for instance, occurs 19.89 times per 1,000 lines of code.

% ESSE PARAGRAFO ESTAVA NA VERSAO ANTIGA
%Altogether, the main contributions of this paper are: \rb{revisar as contribuicoes quando o artigo estiver mais maduro. por enquanto, vamos ignorar esses bullets.}

% \begin{itemize}
%     \item A study about the impact of atoms of confusion on a
%     dynamic language.
%     \item A list of additional atoms of confusion addressing particular 
%     JavaScript constructs and idioms.
%     \item A library and infrastructure to mine atoms of confusion using 
%     code queries (implemented in the .QL language). This library and 
%     infrastructure are freely available and might help researchers and 
%     practitioners to automate program analysis at a large scale.
% \end{itemize}



% Our research aims to identify and characterize JavaScript \textit{atoms of confusion}: small pieces of code that hinder developers' understanding of the source code of a program~\cite{DBLP:conf/sigsoft/GopsteinIYDZYC17}.
% Previous research has confirmed the negative impact of atoms of confusion when considering statically typed languages (e.g., \clang and \cpplang)~\cite{DBLP:journals/ese/MedeirosLAAKRG19, DBLP:conf/msr/GopsteinZFC18}.

% Recent works have explored the impact of \emph{atoms of confusion}~\cite{DBLP:journals/ese/MedeirosLAAKRG19,DBLP:conf/sigsoft/He19,DBLP:conf/msr/GopsteinZFC18} on program understanding.
% Atoms of confusion correspond to small pieces of code that hinder developers' understanding of the source code of a program. 
% Previous research has confirmed the negative impact of atoms of confusion when considering statically typed languages (e.g., \clang and \cpplang)~\cite{DBLP:journals/ese/MedeirosLAAKRG19,DBLP:conf/msr/GopsteinZFC18}.

% In this work, to the best of our knowledge, we are the first to explore this issue in the context of a dynamically typed language (JavaScript), answering the following research questions:
% \emph{How do JavaScript developers  identify atoms of confusion as contributing to program misunderstanding?}
% \emph{What is the impact of atoms of confusion on misunderstanding JavaScript code?},
% and \emph{What is the frequency of occurrence of  atoms of confusion in practice?} 
% To answer these research questions, we conduct this investigation using a mixed-methods approach, involving a survey, a set of interviews, and an activity of mining source code repositories. Altogether, the main contributions of this paper are

% \rb{revisar as contribuicoes quando o artigo estiver mais maduro. por enquanto, vamos ignorar esses bullets.}

% \begin{itemize}
%     \item A first study about the impact of atoms of confusion on a
%     dynamic language. {\color{red}present some of the results here.}
%     \item A library and infrastructure to mine atoms of confusion using 
%     code queries (implemented in the .QL language). This library and 
%     infrastructure is freely available and might help other researcher and 
%     practitioners to automate program analysis at a large scale. 
%     \item A list of additional atoms of confusion addressing particular 
%     JavaScript constructs and idioms. {\color{red}present some insights 
%     about these additional atoms here}
% \end{itemize}

%\section{Background and Related Work}
%\label{back}
%\subsection{Program Comprehension}

% \castor{It would be nice to have a table with the atom candidates
% similar to the one that appears in Gopstein et al. 2017.}

% The concept of program comprehension is central to software maintenance and feature development \cite{DBLP:conf/iwpc/TilleySP96}. Regardless of whether they are performing maintenance on legacy software or adding new features, programmers have to understand what the source code does before making any improvements. A natural consequence of this is that a programmer must also have in mind that other programmers (or their future selves) 
% %are going to have to
% will
% spend time trying to understand the code being currently developed before doing any work themselves.

% As program comprehension is regarded less as a systematic process than as an objective \cite{DBLP:journals/ibmsj/OHareT94}, there is no single approach or framework that is capable of yielding easily understandable programs. Factors such as problem domain, size and complexity of the code base, and available tools all vary significantly between different settings. These aspects, combined with differences between individuals, which can include experience, working memory capacity, and other cognitive traits, make it clear that program comprehension should be approached in different ways in different contexts. 

% However, one of the ways in which programs become less comprehensible occurs when programmers favor complicated syntactic constructs instead of using semantically equivalent versions---when these exist---that are potentially less complex. %allowed by the language or framework they are working with. 
% This can be a matter of personal taste, since different individuals might hold different opinions on whether a given syntax is difficult to understand. Nevertheless, it is possible to empirically investigate if there exist programming language constructs or idioms that are considerably harder to understand, leading readers of the code to frequently make incorrect predictions of its behavior. 

%\subsection{Atoms of Confusion}\label{sec:aoc}

\section{Related Work}

\label{back}

The concept of program comprehension is central to software maintenance and feature development \cite{DBLP:conf/iwpc/TilleySP96, DBLP:journals/ibmsj/OHareT94}. Atoms of confusion is an important field of program comprehension introduced by Gopstein et al. \cite{DBLP:conf/msr/GopsteinZFC18}, who defined them as small code patterns that can verifiably lead to misunderstandings. In addition, for a code pattern to be considered an atom, there must be some alternative pattern or language construct that is functionally equivalent and less likely to cause confusion. One example that occurs in many programming languages is the Change of Literal Encoding. For instance, the \clang code statement \lstinline{printf("\%d", 013);} 
often leads programmers to predict the output to be \texttt{13}, even though the correct answer is \texttt{11}. This occurs because a leading \texttt{0} in a numerical literal indicates that the number is in base \texttt{8}, a fact that is not only unknown to less experienced programmers, but misleading even for seasoned developers. After formulating a list of 19 atom candidates, Gopstein et al. \cite{DBLP:conf/sigsoft/GopsteinIYDZYC17} were able to identify 15 code idioms and language constructs that present a statistically significant difference in comparative answer correctness when each of the 15 atoms was removed. The least confusing atom showed a 14\% boost in prediction accuracy when the atom was removed, whereas the most confusing one showed a 60\% accuracy increase. In a different work, Gopstein et al. \cite{DBLP:conf/msr/GopsteinZFC18} presented the results of a comprehensive research on the incidence of atoms of confusion in the wild, considering open-source projects written in \clang and \cpplang~programming languages.


%Their work laid an empirical framework to identify such atoms. In summary, the basic approach to determine whether a code pattern is an atom involves: (i) identifying a set of atom candidates; (ii) identifying alternative constructs to these atom candidates; (iii) building a set of small programs that each contain a single atom candidate; (iv) building different versions of these programs where the atom candidates were removed and which employ the aforementioned alternative constructs; (v) in an experiment or semi-experiment, presenting subjects with the two versions of the programs for each atom candidate and asking subjects to predict their outputs; (vi) statistically comparing the performance of the subjects.
%By having subjects evaluate code where in the expected output relied exclusively on evaluating the result of previously known confusing programs that had been written in \clang or \cpplang, and asking the participants to predict the output of a set of code snippets where the atoms had been removed, the authors were able to isolate and measure the impact of such confusing atoms. 

These previous studies~\cite{DBLP:conf/sigsoft/GopsteinIYDZYC17,DBLP:conf/msr/GopsteinZFC18}
have a great influence in our work. Even though JavaScript is a dynamic language that differs from \clang in a variety of aspects, it also has a number of constructs in common. In addition, the imperative aspects of the language are syntactically similar to \clang, e.g., assignments, conditional expressions, \texttt{if}-statements, pre and post increments and decrements, among others. On the one hand, this means that some of the atoms of confusion that exist in \clang programs may also occur in JavaScript code. On the other hand, differences between the languages and the surrounding programming cultures may lead to code patterns that are atoms of confusion in one language not being atoms in the other one.
%Although JavaScript allows us to program in different paradigms from the one C follows, most of the JavaScript code found in popular repositories are written under the imperative paradigm. Also, both languages share many syntactical rules, which allows us to replicate some of the C atoms in JavaScript.

%The first study of Gopstein et al.~\cite{DBLP:conf/sigsoft/GopsteinIYDZYC17} inspired other studies to further investigate the impact of atoms of confusion on code comprehension.

Oliveira et al.~\cite{TheEyesDoNotLie} conducted an experiment using an eye-tracker with 30 participants involving three atoms in \clang. They found that code with atoms of confusion requires more time from participants to predict the output and more visual effort to comprehend. Gopstein et al., in another paper~\cite{ThinkingAloud} further explored the original atoms by additionally conducting interviews and having programmers discuss among themselves the studied atoms. The authors argue that this method is important to complement a quantitative analysis. Indeed, their study revealed findings such as a \textit{``correct evaluation of an atom might not mean that a programmer understood its meaning''}. We follow a similar approach as we augment our quantitative analysis with interviews.

%Gopstein et al. also influenced researchers to look into atoms in languages other than \clang and \cpplang.
Langhout and Aniche~\cite{Langhout:2021:ACJ} derived a set of 14 atoms for Java and performed an experiment with 132 students. They found that participants are more likely to make mistakes (ranging from 2.7 up to 56 times) when atoms are present in half of their studied atoms. Castor~\cite{castor2018} presented a preliminary catalog of six atom candidates for the Swift programming language. Unlike JavaScript, in Swift, most of the atoms identified by Gopstein et al.~\cite{DBLP:conf/sigsoft/GopsteinIYDZYC17} are avoided by construction, e.g., it does not have assignments as values, increment operators, or macros. 
%The paper also presents a structured definition for what it means for a code pattern to be an atom of confusion and a set of general principles to help in the identification of new atoms.

Medeiros et al. \cite{DBLP:journals/ese/MedeirosLAAKRG19} analyzed 50 open source projects written in the \clang language with the goal of evaluating 12 code patterns called ``misunderstanding patterns'' by the authors. Many of these misunderstanding patterns are either atoms of confusion or atom candidates~\cite{DBLP:conf/sigsoft/GopsteinIYDZYC17}. This study shows that these patterns are prevalent; among these 50 projects, there are more than 109K occurrences of misunderstanding patterns. 
In order to gauge the relevance of these misunderstanding patterns, the authors sent 35 pull requests removing occurrences of these patterns to randomly-selected open source projects. The authors of the study received feedback for 21 of these pull requests and the maintainers of the projects accepted 8 of them (22.86\%). 
%The authors also analyzed 36 coding style guides from these projects, but found few guidelines specifically addressing misunderstanding patterns. 

% Para compreender a relevância desses misunderstanding patterns os autores enviaram 35 pull requests selecionados aleatoriamente, para remover as ocorrências desses padrões encontrados no código dos projetos avaliados. Os autores receberam feedback para 21 pull requests, e os desenvolvedores aceitaram 8 pull requests (38\%). Os autores também analisaram 36 Guidelines dos open-source projects provide for developers regarding misunderstanding code patterns.

% Castor~\cite{castor2018} presented a preliminary catalog of six atom candidates for the Swift programming language. Unlike JavaScript, in Swift most of the atoms identified by Gopstein et al.~\cite{DBLP:conf/sigsoft/GopsteinIYDZYC17} are avoided by construction, e.g., it does not have assignments as values, increment operators, or macros. The paper also presents a structured definition for what it means for a code pattern to be an atom of confusion and a set of general principles to help in the identification of new atoms. 

%Castor \cite{castor2018} apresentou um catalógo preliminar de átomos de confusão na linguagem de programação Swift. O autor identificou 6 estruturas candidatas a átomos de confusão em Swift. Foram omitidos os candidatos que tinham uma correspondência direta com os átomos previamente identificados por Gopstein et al. \cite{DBLP:conf/sigsoft/GopsteinIYDZYC17}.

%% According to Arnaoudova et al.~\cite{Arnaoudova:2016:LAW}, Linguistic Antipatterns (LA) are \textit{``recurring poor practices in the naming, documentation, and choice of identifiers in the implementation of an entity [..] that may impair program understanding''}. These antipatterns are design issues that go against developer's  intuition about a program, for example, a \texttt{set} method that returns a result. Similarly to atoms of confusion, LAs occur at the source code level and hinder code readability. Differently from atoms of confusion, LAs are more akin to design principles than code idioms. In addition, they are language-agnostic, unlike atoms of confusion.

Santos and Gerosa~\cite{Santos:2018:ICP} analyze the impact of eleven coding practices on program readability. However, most of the practices analyzed in their study relate to code formatting, instead of language constructs or idioms that might introduce misunderstanding. We borrow from their research the methodology for collecting the practitioners opinion about code snippets preference.


%% \castor{Provavelmente devemos reduzir o espaço tomado pelos quatro trabalhos acima, a depender da existência de outros. O de Linguistic Antipatterns talvez até devesse sair.}
%% \rb{Incluir os trabalhos do grupo do Marcio Ribeiro e do Fernando Castor. Quais outros trabalhos deveriamos incluir aqui?}

\section{Study Settings}
\label{method}

The main goal of this research is to investigate the impact of atoms of confusion in JavaScript code comprehension. As such, in this paper we answer the following general research questions: 

\begin{enumerate}[(RQ1)]
    \item Do JavaScript developers identify atoms of confusion as contributing to program misunderstanding? 
    \item What is the impact of atoms of confusion on developers' comprehension of JavaScript code? 
    \item What are the particular JavaScript idioms and language constructs that might cause source code misunderstanding?
    \item What is the frequency of occurrence of atoms of confusion in practice (i.e., in open-source JavaScript projects)?
\end{enumerate}

 
Answering these research questions has several implications. For instance, we can either generalize or refute the perceptions about atoms of confusion already discussed in the literature (goal of research questions (RQ1) and (RQ2)). In addition, answers to the third and fourth questions allow us to enrich existing catalogs about atoms of confusion and discuss how often they occur in practice. Our motivation to detect the presence and impact of atoms of confusion is also motivated by the possibility of laying the foundations for the implementation of libraries that can automatically transform code into its simpler versions---though we postpone these results to a future research work. This might lead to a new catalog of refactoring for removing atoms of confusion.  To answer these research questions we conduct a mixed-methods study, including a survey, a set of interviews, and an effort of mining JavaScript software repositories.  


 \subsection{Settings for the first study: a survey}\label{sec:survey-settings}\castor{Eu mudaria os títulos desta sub-seção e das duas próximas para "First study: Survey", "Second study: Mining open source repositories", etc. Vale a pena manter "First study", etc.?}
% \subsubsection{The Atoms Considered In This Work}

In the first study, our goal was to collect evidence to answer two specific questions: 
\begin{itemize}
    \item \emph{Do code snippets that contain atoms of confusion produce a higher misunderstanding rate when programmers try to predict their outcome?}
    \item \emph{Do code snippets with atoms of confusion require programmers to take longer to predict their output?}
\end{itemize}

To address these questions we conducted a survey to assess the impact of atoms of confusion on understanding JavaScript code. Since JavaScript and C have some constructs in common (Section~\ref{sec:aoc}), we first selected a set of atoms of confusion for the \clang language~\cite{DBLP:conf/sigsoft/GopsteinIYDZYC17} that may also exist in JavaScript programs. {\color{red}In addition, we consider a particular atom that is exclusive to JavaScript in this first study}. Appendix~\ref{sec:appendix-atoms} summarizes all atoms we consider in our research. 

\subsubsection{Survey design} 

The design of our first study aims to block two variables (developer experience and the code snippets) and considers two treatments: the presence or absence of 
atoms of confusion within the code snippets. 
To achieve such a design goal, controlling the effect of experience and individual code snippets, we resorted to the \textit{Latin Square Design} \cite{Hunter-Experimenters}. Using this design we create a $2 x 2$ matrix in which each row represents a subject and each column indicates the set of code snippets. The design of each square (a replica) is such that no treatment is repeated in the same row or column. For example, if a given subject (P1) is asked to predict the output of code snippets that contain atom candidates \texttt{\{1,2,3,4,5\}}, then, when answering questions about non-confusing code snippets, they will only be presented with non-confusing versions of atom candidates \texttt{\{6,7,8,9,10\}}. Furthermore, a given subject (P2), which constitutes the second row of our example square, will be asked questions about the non-confusing versions of \texttt{\{1,2,3,4,5\}}, and will answer questions about confusing snippets of \texttt{\{6,7,8,9,10\}}. By doing that, we guarantee that all of our 20 snippets are contained within each square, and that each configuration occurs only once within a square. Figure \ref{fig:latinsquare} offers a visual representation of the concept.

  \begin{figure}[htb!]
      \noindent
      \centering
      \includegraphics[scale=.50]{images/latin-square.pdf}
      \caption{Latin square design. Each ``square'' corresponds to 
      a replica in our study. Each replica comprises two students (square rows) 
      and two sets of code snippets (CS-1 and CS-2). We randomly apply the 
      treatments (atom or non-atom code) to the cells of the squares.} 
      \label{fig:latinsquare}
  \end{figure}


Having selected 10 atoms in our first study, we wrote small pieces of code that contained each atom. We also wrote their equivalent snippet without the confusing idioms and constructs, leading to a total of 20 code snippets. In order to reduce the cognitive effort, we decided that each subject would be asked to predict the output of a subset of 10 listings, wherein each subset contained 5 blocks that contained atoms of confusion, whilst the remaining 5 had the atoms removed. The order in which the questions were presented was randomized. By doing this, we were seeking to minimize the chances of subjects being aware that the current listing they were analyzing contained (or not) atoms of confusion.
That is, each respondent of the survey should indicate what would be the outcomes of the code snippets, some of them having atoms of confusion (while other code snippets did not). 
We measured answer correctness and total time each participant needed to answer reach question of their survey. 

\subsubsection{Survey Instrument} 

We implemented our survey as a web application. As part of this 
effort, we carried out an informal pilot whose main objectives were: to spot bugs in the application and in the data collection mechanism; to gain feedback from respondents about the user experience of the application; and to formulate an estimate about how long answering the survey would, on average, take. Fellow undergraduate students, professional colleagues, and friends took the pilot survey. Some users reported layout defects, and many reported that the landing page did not explain the survey well enough. We also spotted minor issues with our routines to create and populate the Latin Squares. 

We organized the survey in {\color{red}three} sections. The first section aims to characterize the subjects, asking for their age, education level, and programming experience. We also included a check button, whose checking meant users agreed that all collected data would be used solely for research purposes. In the second section, we presented to the participants a small set of instructions, where we explained how the survey worked and asked them to dedicate their attention to it. We stressed to participants the importance of not using any aids during the survey, such as online or console interpreters. For each question page, we kept track of whether or not the subjects switched windows. 

The next section of the survey presented a sequence of 10 questions, each containing a code snippet. For each question, there was a text box where the answer should be written. There was also an ``I do not know'' button, which, when clicked, led the subject to the next question. In our setting, ``I do not know'' was treated as a wrong answer. The code snippets were presented as images copied from a text editor, so as to demotivate respondents from resorting to external resources by copying and pasting the code into an interpreter. Upon submitting their answer for a particular question, the subject was automatically led to a similar page, containing another snippet.
    
We decided not to provide feedback about the time students took to answer each question. Nor did we tell them whether their answers were correct or not. Our main concern was to avoid introducing bias for future respondents. Since we posted the survey in a social media platform, if we gave respondents instant feedback, they might post comments on particular atoms, therefore interfering with future participants' thought process.

% \rb{nao acho esse paragrafo necessario} 
% \adriano{o de baixo ne? eu tambem acho}

% As we mentioned before, we first wrote the code listings in a text editor, and took pictures of it. In the case of an atom of confusion that was exclusive to JavaScript, which we called \textit{Automatic Semicolon Insertion} (see Appendix~\ref{sec:appendix-atoms}), it was necessary to remove the syntax highlighter. Even though semicolons at the end of statements are optional to programmers in JavaScript, the interpreter automatically inserts them into the code. Our text editor was incorrectly highlighting a line break after a return statement, even thought it was valid JavaScript syntax. We had thus to turn the highlighter off to take the picture of this atom. 

\subsubsection{Survey audience} 
\diego{Esses detalhes são possíveis candidatos para remoção, se precisarmos de espaço}
{\color{blue}We invited developers to answer our survey by sending invitations to communities of JavaScript programmers on the Internet. Our initial plan was to try to engage contributors for two major JavaScript projects, namely Node.js and NPM. None of the projects, though, offered a direct way to interact with the community, such as an online discussion forums or mailing list. We would be required to first make contact with the projects' leaders, and only then would we have a chance to approach potential respondents. Due to time constraints, we decided to look for respondents elsewhere.}
We then posted the survey on Reddit.\footnote{Reddit is a North American online discussion platform. Its discussion threads, often called subreddits, are sorted by subject. For our research, we posted information and links to the survey in two JavaScript subreddits.}. We explained our research purposes, and asked developers of any level of expertise to take the survey. To incentivize serious engagement, we proposed to raffle \$50 gift cards on Amazon products at the end of the survey. Within twelve hours we collected more than 150 answers, populating more than 70 replicas of the Latin Squares. We were able to collect significant data on time taken and discrepancies in answer correctness between confusing and non-confusing versions of the snippets. 
    
\rb{acho que podemos mover esse proximo paragrafo para a secao de resultados, ou de ameacas}
\adriano{acho que pode ir para resultados. nao considero uma ameaca pois eu e o Caio tratamos manualmente as entradas a serem removidas e garantimos a integridade dos latin squares}

Inconsistencies may arise when each square is being built. The main source of inconsistency we faced was when a user quit in the middle of the survey. When this happened, his row in the square was left incomplete. We considered all squares which contained incomplete rows to be invalid, and discarded them. Since we had a large enough number of samples, the squares we had to discard did not impact our results.
    
    
% \subsubsection{Pilot Survey}    

% To validate the web application we developed to conduct the survey, we ran an informal pilot survey whose main objectives were:
%     \begin{itemize}
%         \item To spot bugs in the application and in the data collection mechanism;
%         \item To gain feedback from respondents about the user experience of the application;
%         \item To formulate an estimate about how long the survey would, on average, take.
%     \end{itemize}
    
% We had fellow undergraduate students, work colleagues and friends take the pilot survey. We discovered that the aspect that needed most improvement was the user experience. Some users reported layout defects, and many reported that the landing page did not explain the survey well enough. We also spotted minor issues with our routines to create and populate the Latin Squares. After performing all the necessary changes, we were ready to conduct another experiment.

\rb{Comentei o survey com alunos}

% \subsubsection{A Survey With Undergraduate Students}
%     Our second attempt was due to courtesy of a professor of our department. During the semester of the writing of this text, he was lecturing a Data Structures course, which is usually taken during the second semester of the undergraduate course. The professor agreed to take the students to a laboratory during one of his classes, and all the students who attended that day took the survey.
    
%     No issues were reported with the web application, and all answers were appropriately collected. This time, all the Latin Squares were being populated correctly, making the data we collected eligible for use in the final analysis. However, we opted to dispose of all the data we gathered in this occasion. What motivated this decision were the two following aspects:
%     \begin{enumerate}
%         \item The students were not focusing hard enough. Most students were not very serious about the survey. While we watched them answer the questions, we could see many students peeking at each other's screens. We could also clearly see students not taking nearly enough time to reason about the code, and answering the questions without giving it proper thought.
%         \item We needed a more diverse spectrum of respondents. In the beginning of our research, we established that we would need at least 25 Latin Squares in order to be able to have significant results. By conducting this survey, we were able to populate 11 Latin Squares. Compounded by the problem reported above, we were concerned that our spectrum of respondents might end up skewed towards undergraduate students with little or no JavaScript programming experience.
%     \end{enumerate}
    
%     These two facts led us to conclude we needed to conduct a survey in which partakers engaged voluntarily, whilst also having an incentive to think hard enough about the problems they faced. To address these points, we prepared our final setting.
    
\subsection{Settings for the second study: Mining open source repositories}

To understand how often atoms of confusion appear in real settings, and thus answer our fourth research question (\emph{What is the frequency of occurrence of atoms of confusion in practice?}), we mined a set of GitHub open source repositories. To this end, we first collected the most popular GitHub repositories that are primarily written is JavaScript. We measured popularity using the project's stargazers, and selected only projects with stargazers\_count > 100. This metric, available on  GitHub API, represents the number of stars a project received from users of the platform. The same metric has been used in a number of previous studies as a proxy to estimate project's popularity~\cite{gyimesi2019bugsjs}.

After filtering out JavaScript project candidates, in the second step we built a curate dataset comprising the top 100 most popular repositories. Examples of projects in this dataset include \textsc{React}, \textsc{Node JS}, and \textsc{AngularJS}. Table~\ref{tab:projects-statistics} presents some statistics about the projects we consider in our research. The size of the projects range from small ones (883 liens of code) to complex systems with more than 10 MLOC. All projects in our dataset have at least \num{1161} forks and \num{23643} stars. We automated all the steps to filter, clone, and collect the statistics from the repositories using Python scripts.

In the third step we mined atoms of confusion from the repositories in our curate dataset. We written source code queries usng the CodeQL language~\cite{moor:gttse2007}, in order to filter the atoms of confusion. CodeQL is an object-oriented variant of the Datalog language~\cite{rodriguez2020efficient}, and currently supports researchers and practitioners to query the source code of systems written in different languages (such as Java and JavaScript). 
We also automate the process of running the queries and exporting the results to a format that simplifies our analysis and the reproduction of this study. In the final step we carry out basic descriptive statistics, in order to understand how often the atoms of confusion appear in practice. 

\begin{table*}[ht]
 \centering
 \begin{tabular}{rrrrrrr}
   \hline
             & Min. & 1st Qu. & Median & Mean & 3rd Qu. & Max. \\ \hline
 Lines of Code           & \num{883}  & \num{14939} & \num{39520} & \num{281088.46} & \num{145990.75} & \num{10149685} \\
 Num. of Contributors    & \num{6}   & \num{121} & \num{243} & \num{431.82} & \num{436.75} & \num{4042} \\
 Num. of Forks           & \num{1161}   & \num{3365} & \num{5711} & \num{8201.96} & \num{8438.25} & \num{68826} \\
 Num. of Stars        & \num{23643} & \num{27542.25} & \num{35687.5} & \num{45284.86} & \num{47963.25} & \num{310725} \\
    \hline
 \end{tabular}
 \caption{Some descriptive statistics about the projects used in the study}
 \label{tab:projects-statistics} 
 \end{table*}
 
\subsection{Settings for the third study: interview with practitioners}

To complement our initial survey, we performed semi-structured interviews with professional JavaScript developers, aiming to identify their perceptions regarding code snippets containing atoms of confusion. We also asked each participant if they knew of any other JavaScript-specific construct that they regarded as confusing. In this section we details the protocol we followed to conduct the interviews, and we also describe how their results were analysed.

% Nós realizamos entrevistas semi-estruturas com o objetivo de identificar a percepção dos desenvolvedores com experiência em JavaScript sobre algumas questões relacionadas à compreensão de código em JavaScript. Assim, nesta Seção nós descrevemos os procedimentos adotados para selecionar os participantes para as entrevistas e detalhamos como as entrevistas foram conduzidas. Além disso, detalhamos como os resultados das entrevistas foram analisados.

\subsubsection{Participants Selection}: The participants of the interviews were invited personally by the authors, and the pool of potential participants was based on our network of contacts. Our main criterion of selection was that all participants had to be working professionally with JavaScript. We invited a total of 17 developers, and 15 of these agreed to participate, and throughout a two week period, we conducted all the interviews.

% Os participantes das entrevistas foram selecionados a partir da nossa rede de contatos. Nós convidamos 17 pessoas e 2 não responderam ao nosso convite para participar das entrevistas. Durante 2 semanas, nós entrevistamos 15 desenvolvedores. Iniciamente nós conduzimos duas entrevistas como piloto. Após o piloto com dois desenvolvedores com mais de 03 anos de experiência, nós adicionamos uma nova pergunta na entrevista: Você conhece alguma construção em JavaScript que leve a um mal entendimento do código? Ao finalizar as entrevistas, nós perguntamos ao desenvolvedor se ele conhecia algum outro desenvolvedor com experiência em JavaScript que pudesse nos indicar para participar das entrevistas. Nós entrevistamos 02 desenvolvedores a partir da indicação dos entrevistados. A Tabela \ref{pinterview} apresenta o perfil dos 15 participantes das entrevistas.

\subsubsection{Interview Process}

We conducted semi-structured interviews using a web conferencing software. All the interviews were recorded with the consent of participants. On average, the interviews lasted 26.29 minutes, with the shortest one lasting 14.59 minutes, and the longest one 43.06 minutes. The main goal of the interviews was to have developers assess in real time whether two versions of code with the same behavior differed in readability, and, if so, which version they regarded as the easier one to understand. The interviews were conducted by two of the authors of this text, and a third one listened to all the recordings to cross-validate the collected data.

% Nós conduzimos entrevistas semi-estruturadas utilizando um software de conferência de audio/video. Todas as entrevistas foram gravadas com o consenso dos entrevistados. As entrevistas tiveram uma duração de aproximadamente 26.29 minutos. A entrevista com menor duração foi de 14.59 minutos e de maior duração foi de 43.06 minutos. O objetivo ao realizar as entrevistas era identificar quais as construções em JavaScript podem corresponder a confusões de código. As entrevistas foram realizadas por dois dos autores deste artigo.

The interviews had three main parts. In the first one, we asked the developers the following demographic information: name, email, gender, level of education, current job position, JavaScript experience and other programming languages they have worked with. Table \ref{pinterview} presents the profile of each participant.

\begin{table*}[htb!]
\centering
\begin{tabular}
{|p{0.4cm}|p{0.9cm}|p{2.0cm}|p{7.0cm}|p{1.9cm}|p{3.0cm}|}
\hline
ID & Gender & Level & Role & JS Experience & Other Languages \\ \hline  
P1 & Male & Graduate & Developer at Federal Court of Accounts - TCU Brazil & 9 years & Java, PHP, C, Go  
\\ \hline
P2 & Male & High School & Developer at Luma Health & 3.5 years & Python, Go, Dart, Lua, C++, C\#
\\ \hline
P3 & Male & Graduate & Developer at Superior Labour Court - TST Brazil & 4 years & Java, C
\\ \hline
P4 & Male & Undergraduate & Developer at Brazilian Association of Research and Industrial Innovation - EMBRAPII & 3 years & Python, C, C++, Java, Go
\\ \hline
P5 & Male & Master Student & Developer na Qubo Technology & 3 years & Python, SQL
\\ \hline
P6 & Male & Graduate & Software Architect at Superior Labour Court - TST Brazil & >15 years & Java, PHP, C, Python , Ruby, C\# and R
\\ \hline
P7 & Male & Graduate & Developer na Stefanini & 6 years & C, C++, Java, Assembly, Kotlin
\\ \hline
P8 & Male & PhD & Developer Regional Labour Court - TRT Goiás, Brazil & 2 years & Java, Python
\\ \hline
P9 & Male & Graduate & IT Analyst at Regional Bank of Brasilia - BRB Brazil & 5 years & PHP
\\ \hline
P10 & Male & Master Student & Business Analyst and Developer at National Confederation of Industry - CNI Brazil & 4 years & Java, C\# and Python
\\ \hline
P11 & Male & Master Student & Software Architect at the University of Brasilia's Informatics Centre & 4 years & Java, Erlang, C\#, Cobol
\\ \hline
P12 & Female & Master & Software Engineers at Novatics & 1 year & C
\\ \hline
P13 & Male & Master & Developer at the Military Police of the Federal District - PMDF Brazil & 13 years & Java, PHP
\\ \hline
P14 & Female & Master Student & Developer at Neoway & 3 years & C, Python, Ruby
\\ \hline
P15 & Female &  Graduate & Developer at SICOOB & 2 years & Java, PHP
\\ \hline
\end{tabular}
    \caption{Demographic information of the participants}
    \label{pinterview}
\end{table*}

The second part of the interview was more open, and our aim here was to allow the subjects to describe their JavaScript experience, as well as to allow them to make suggestions about any JavaScript's constructs they regarded as innately confusing, as this would allow us to include their suggestions in future work. The questions asked in this section were:
\begin{itemize}
    \item In general, do you implement new JavaScript components, or do you maintain components developed by other programmers?;
    \item In your opinion, what are the main pros and cons of the language?;
    \item Do you think JavaScript is a language that produces code that is hard to understand?
    \item Do you regard any particular construct of the language as especially confusing?
\end{itemize}

\rb{acho melhor mover os code snippets da entrevista para um apendice online. deixei comentado aqui.}
\adriano{como os snippets sao os mesmos do TCC, a ja temos o apendice, nao poderiamos colocar no apendice B deste texto? posso escrever os snippets no repositorio tambem. deixa-los no apendice pode tornar o paragrafo abaixo mais compreensivel/verificavel.}

In the third part of the interview, participants were shown the 20 code snippets that were used to build the aforementioned Latin Squares. The participants were told both sides of each pair had the same output, and were asked to evaluate which construct was easier to understand. The interviewers restrained themselves from introducing bias into the answers by not explaining that one of the sides of each pair contained the atom under investigation. Subjects were just presented the snippets and allowed to take the necessary time to decide on the most readable snippet.

% As entrevistas foram divididas em três etapas. Na primeira etapa da entrevista nós perguntamos aos desenvolvedores as seguintes informações: Nome; E-mail; Gender; Level; Role; Experience; Quais as outras linguagens de programação ele/ela conhecia. Na segunda etapa da entrevista,  perguntamos aos desenvolvedores: a) No geral você escreve novos componentes em JavaScript, ou você faz manutenção em componentes desenvolvidos por outra pessoa (explique o seu dia-a-dia); b) Na sua opinião, quais são os pontos fortes e fracos da linguagem JavaScript. c) Você acha que a linguagem JavaScript é uma linguagem que leva a um código difícil de ser compreendido? d) Tem alguma construção em JavaScript que você conhece que pode levar a uma dificuldade de entendimento. Na terceira etapa da entrevista nós apresentamos alguns exemplos de trechos de código em JavaScript, contendo os 10 tipos de construção apresentados na Table \ref{tab:atoms} com e sem átomos de confusão (Lado A e Lado B), conforme apresentado na Table \ref{snippets}. Nós perguntamos aos desenvolvedores: 1) Você acha que o código do lado B prejudica de alguma forma a compreensão do código? Os dois códigos apresentados nos dois lados de alguma maneira eram equivalentes. 

% \begin{longtable}
% {| p{.50\linewidth} | p{.50\linewidth} |} 
% % \begin{center}
% \caption*{Code Snippets presented on the interview}
% \label{snippets}
% % \begin{tabular}{ | p{8.2cm} | p{8.2cm} | } \hline
% \hline
% \multicolumn{2}{| c |}{Arithmetic as Logic}  \\ \hline
% Side A  &  Side B \\ \hline
% \begin{lstlisting}[language=JavaScript]
% let array_1 = [1,2,3];
% let array_2 = [2,3,2];
% if(array_1.length - array_2.length){
%   console.log(true);
% }
% else{
%   console.log(false);
% } \end{lstlisting}
% & 
% \begin{lstlisting}[language=JavaScript]
% let array_1 = [1,2,3];
% let array_2 = [2,3,2];
% if(array_1.length != array_2.length){
%   console.log(true);
% }
% else{
%   console.log(false);
% } \end{lstlisting} \\ \hline
% \multicolumn{2}{| c |}{Assignment as value}  \\ \hline
% Side A  &  Side B \\ \hline
% \begin{lstlisting}[language=JavaScript]
% function resetSchedulerState () {
%   let activatedChildren = {length: 10};
%   let index = length = activatedChildren.length = 0;
%   console.log(index);
% }
% resetSchedulerState(); \end{lstlisting}
% & 
% \begin{lstlisting}[language=JavaScript]
% function resetSchedulerState () {
%   let activatedChildren = {length: 10};
%   activatedChildren.length = 0;
%   let length = activatedChildren.length;
%   let index = length;
%   console.log(index);
% }
% resetSchedulerState(); \end{lstlisting} \\ \hline
% %\pagebreak 
% \hline
% \multicolumn{2}{| c |}{Automatic Semicolon Insertion}  \\ \hline
% Side A  &  Side B \\ \hline
% \begin{lstlisting}[language=JavaScript]
% function example(){
%   return
%       10
% }
% if(example() == 10){
%   console.log(true);
% }
% else{
%   console.log(false);
% } \end{lstlisting}
% & 
% \begin{lstlisting}[language=JavaScript]
% function example(){
%   return 10;
% }
% if(example() == 10){
%   console.log(true);
% }
% else{
%   console.log(false);
% } \end{lstlisting} \\ \hline
% \multicolumn{2}{| c |}{Comma Operator}  \\ \hline
% Side A  &  Side B \\ \hline
% \begin{lstlisting}[language=JavaScript]
% let V1 = 5, V2 = 10;
% V1 = (V2 = 1, 2);
% console.log(V1+V2);
% \end{lstlisting}
% & 
% \begin{lstlisting}[language=JavaScript]
% let V1 = 5, V2 = 10;
% V2 = 1;
% V1 = 2;
% console.log(V1+V2);
% \end{lstlisting} \\ \hline
% %\pagebreak
% \hline
% \multicolumn{2}{| c |}{Ternary Operator}  \\ \hline
% Side A  &  Side B \\ \hline
% \begin{lstlisting}[language=JavaScript]
% let config = {size: 3, isActive: false};
% const _config = config.isActive === true ? config : {size: 10};
% console.log(_config.size); \end{lstlisting}
% & 
% \begin{lstlisting}[language=JavaScript]
% let config = {size: 3, isActive: false}
% let _config;
% if(config.isActive === true){
%   _config = config;
% }
% else{
%   _config = {size: 10};
% }
% console.log(_config.size); \end{lstlisting} \\ \hline

% \multicolumn{2}{| c |}{Implicit Predicate}  \\ \hline
% Side A  &  Side B \\ \hline
% \begin{lstlisting}[language=JavaScript]
% let V1 = 10, V2 = 3;
% if (!(V1 % V2)){
%   console.log(true);
% }
% else{
%   console.log(false);
% } \end{lstlisting}
% & 
% \begin{lstlisting}[language=JavaScript]
% let V1 = 1, V2 = 2;
% if ((V2 - V1) == 0){
%   console.log(true);
% }
% else{
%   console.log(false);
% } \end{lstlisting} \\ \hline
% \multicolumn{2}{| c |}{Logic Control Flow}  \\ \hline
% Side A  &  Side B \\ \hline
% \begin{lstlisting}[language=JavaScript]
% let V1 = 3;
% let V2 = 5;
% let V3 = 0;
% while (V1 != V2 && ++V1) {
%   V3++;
% }
% console.log(V1 + V3); \end{lstlisting}
% & 
% \begin{lstlisting}[language=JavaScript]
% let V1 = 1;
% let V2 = 11;
% let V3 = 0;
% while (V1 != V2) {
%   V1++;
%   V3++;
% }
% console.log(V1+V3); \end{lstlisting} \\ \hline
% \multicolumn{2}{| c |}{Omitted Curly Braces \& Indentation}  \\ \hline
% Side A  &  Side B \\ \hline
% \begin{lstlisting}[language=JavaScript]
% let V1 = 1, V2 = 2
% if (V1 > V2)
% V2 = 1
% V1 = 2
% console.log(V2+V1) \end{lstlisting}
% & 
% \begin{lstlisting}[language=JavaScript]
% let V1 = 1, V2 = 2;
% if (V1 > V2) {
%   V2 = 1;
% }
% V1 = 2;
% console.log(V2+V1); \end{lstlisting} \\ \hline
% \multicolumn{2}{| c |}{Post Increment}  \\ \hline
% Side A  &  Side B \\ \hline
% \begin{lstlisting}[language=JavaScript]
% let V1 = 0;
% if (V1++ == 0) {
%   console.log(true);
% }
% else {
%   console.log(false);
% } \end{lstlisting}
% & 
% \begin{lstlisting}[language=JavaScript]
% let V1 = 0;
% if (V1 == 0) {
%   console.log(true);
% }
% else {
%   console.log(false);
% }
% V1 = V1 + 1; \end{lstlisting} \\ \hline
% \multicolumn{2}{| c |}{Pre Increment}  \\ \hline
% Side A  &  Side B \\ \hline
% \begin{lstlisting}[language=JavaScript]
% var index = 1;
% while (++index < 10) {
%   console.log(index);
%   break;
% } \end{lstlisting}
% & 
% \begin{lstlisting}[language=JavaScript]
% var index = 0;
% index = index + 1;
% while (index < 10) {
%   console.log(index);
%   index = index + 1;
%   break;
% } \end{lstlisting} \\ \hline
% % \end{tabular}
% % \end{center}
% % \end{table}
% \end{longtable}

\subsubsection{Interview Analysis}

Nós realizamos a análise das entrevistas em pares. Inicialmente um dos participantes de todas as entrevistas transcreveu os resultados. Posteriormente, um dos autores do trabalho ouviu os áudios das entrevistas e transcreveu os resultados. 

\textcolor{red}{Adriano teria algo mais para adicionar aqui na  análise das entrevistas?}

\todo[inline]{adriano:estou escrevendo uma tabela em que cada linha e um dos atomos, e colunas representam \% preferencia com atomo, \% preferencia sem atomo, \% indiferente. Mas acredito que esta tabela deve estar na section results, certo? O resto das informacoes deste paragrafo eu citei acima no Interview Process, entao acho que podemos abolir esta section. -- Edna: Concordo contigo Adriano}




\section{Results and Discussion}
\label{sec:results}

In this section we present the main findings of our research. We first present the results of each study (Section~\ref{sec:survey-resuts}, Sections~\ref{sec:msr-results}, and Section~\ref{sec:interview-results}). After that, in Section~\ref{sec:discussion} we consolidate our findings and present some implications of our study. 

\subsection{Results of the first study: a survey}
\label{sec:survey-resuts} 

Our first study aims to investigate the impact of atoms of confusion while developers try to understand JavaScript code. We estimate this impact considering two perspectives: \emph{misunderstanding rate} (number of wrong answers) and \emph{cognitive effort} (time necessary to provide an answer). As discussed in section~\ref{sec:survey-settings}, 140 participants answered the survey in full. Figure~\ref{fig:degree} and Figure~\ref{fig:xp} show the distributions of the subjects, according to their education level and years of programming experience, respectively.

\begin{figure}[htb]
      \centering
      \includegraphics[scale=.25]{images/degree.png}
      \caption{Participants' Education Level}\label{fig:degree}
  \end{figure}
  
  \begin{figure}[htb]
      \centering
      \includegraphics[scale=.25]{images/experience.png}
      \caption{Participants' Experience} \label{fig:xp}
  \end{figure}

It is possible to realize in Figure \ref{fig:xp} that more than a half of the participants hold either a BS degree or had taken some university course, meaning the participants have had some level of formal education on programming. No subject reported to have never attended university. Out audience comprises experience developers, and 93\% of the respondents have more than one year of experience with programming (see Figure~\ref{fig:xp}). Actually, 56\% of the participants have either between four and ten years of experience (37\%) or more than ten years of experience (19\%). Other 37\% of the respondents have between one and four years of experience. In this way, we can characterize the effect of atoms of confusion in JavaScript code taking into account the perceptions of both novice and experience developers. 

Each participant in this study evaluated {\color{red}10} code snippets, from which 5 were in their confusing versions, whilst the other 5 contained simplified versions of the code snippets (i.e., without the confusing constructs and idioms). The participants should provide the expected outcomes of the code snippets. We collected information about \emph{correctness} (whether the participant correctly predicted the program's output) and \emph{cognitive workload} (the time taken to answer the question). Our final dataset consists of 70 complete Latin Squares, which means that our assessment consider a total of 140 answers to each pair of code snippets.


Regarding \emph{correctness}, Table~\ref{results_correctness} and Figure~\ref{fig:boxplotcorrectness} summarize the results of the survey. Accordingly, the version of seven code snippets without atoms of confusion present at least a 15\% improvement in answer correctness. In particular, \emph{Comma Operator} atom presents the highest impact on misunderstanding. 
Curiously, frequently used constructs and idioms, such as \emph{Post Increment} and \emph{Omitted Curly Braces} (see Section~\ref{sec:msr-results}), also introduce high degrees of confusion. As the boxplot in Figure \ref{fig:boxplotcorrectness} shows, there is a considerable decrease in the average number of incorrect answers when the atoms were not present. Also, the sample of answers where there was no atoms had almost no dispersion, which is a sign that the non-confusing code is easier to evaluate correctly.

% \begin{table}[htbp]
% \caption{Difference in answer correctness between confusing and non-confusing pairs}
% \begin{center}
% \begin{small}
% \begin{tabularx}
% {{\linewidth}}{l p{1.5cm} p{1.1cm} p{1.1cm} p{1.2cm} }
% \textbf{Atom} & \textbf{\%Correct} & \textbf{\%Correct} \\
% &  \multicolumn{1}{l}{With AOC} \multicolumn{2}{l}{Without AOC}  & \Delta (\%)    \\
%  \hline
% Comma Operator & 40 & 93 & +132\%                 \\     
% Automatic Semicolon  Insertion & 46 & 97 & +110\% \\
% Post Increment & 69 & 91 & +31\%         \\
% Omitted Curly Braces & 67 & 83 & +23\%   \\ 
% Assignment as Value & 80 & 97 & +21\%    \\
% Implicit Predicate & 83 & 97 & +16\%     \\
% Logic as Control Flow & 59 & 68 & +15\%  \\
% Ternary Operator & 86 & 94 & +9\%        \\
% Pre-Increment & 71 & 76 & +7\%           \\
% Arithmetic as Logic & 91 & 90 & -1\%    \\
% \end{tabularx}
% \end{small}
% \end{center}
% \label{results_correctness}
% \end{table}

\begin{table}[htbp]
\caption{Average number of correct answers.}
\label{tab:difference-correctness}
\begin{tabular}{lccc} \toprule
 & \multicolumn{2}{c}{Average Number of Correct Answers} & \\ [0.1cm]
 Atom & Code with Atoms & Code without Atoms & $\Delta$(\%) \\ \midrule 
 Comma Operator          & 40 & 93  & +132 \\
 Post Increment          & 69 & 91  & + 31  \\
 Omitted Curly Braces    & 67 & 83  & +23 \\
 Assignment as Value     & 80 & 97  & +21 \\
 Implicit Predicate      & 83 & 97  & +16 \\
 Logic as Control Flow   & 59 & 68  & +15 \\
 Ternary Operator        & 86 & 94  & +9  \\
 Pre-Increment           & 71 & 76  & +7  \\
 Arithmetic as Logic     & 91 & 90  & +1  \\ \bottomrule
\end{tabular}
\end{table}


\begin{figure}[htb!]
\noindent
 \centering
 \includegraphics[scale=.20]{images/wrong.png}
 \caption{Number of wrong answers of each subject}
 \label{fig:boxplotcorrectness}
 \end{figure}

\rb{verificar a consistencia nas questoes de pesquisa mais especificas}
Regarding the first question we 
address in the survey (\emph{Do code snippets that contain atoms of confusion produce a higher error rate than snippets where the atom is removed?}), we found evidence that the atoms of confusion lead programmers to misunderstand JavaScript code. We also realized that just one atom whose correction has a non-significant improvement in the percentage of correct answers---we found an improvement of at least 15\% in the correct answers when removing the confusing code for seven atoms (out of ten atoms we consider in the survey). 

\rb{acho que valeria a pena colocar o teste de hipotese aqui}
\adriano{como fazemos/descrevemos o teste?}

Table \ref{tab:difference-time-taken} displays the results of the measurements of the average time taken to answer with and without atoms of confusion. Once again, the \emph{Comma Operator} was the construct whose removal had the greatest impact. The more expressive results in this measurement are the ones regarding \emph{Logic as Control Flow}, \emph{Implicit Predicate} and \emph{Omitted Curly Braces}, {\color{blue}as they are frequent in JavaScript libraries}. 

\rb{n\~{a}o sei se devemos antecipar o resultado do estudo de minera\c c\~{a}o 
nessa se\c c\~{a}o. talvez nao comprometa tanto.}

Figure \ref{fig:timetoanswer} shows a box plot of the difference in time taken the predict the output of average time to assess the code snippets. Although the dispersion is smaller for non-confusing code blocks, the median time did not vary much.


\begin{table}[htbp]
\caption{Average number of correct answers.}
\label{tab:difference-time-taken}
\begin{tabular}{lccc} \toprule
 & \multicolumn{2}{c}{Average Time in Seconds to Answer} & \\ [0.1cm]
 Atom & Code with Atoms & Code without Atoms & $\Delta$(\%) \\ \midrule 
Comma Operator        & 60 & 21  & -65  \\
Logic as Control Flow & 85 & 49  & -42  \\
Implicit Predicate    & 33 & 24  & -27  \\
Omitted Curly Braces  & 43 & 31  & -27  \\
Assignment as Value   & 53 & 49  & -7   \\
Ternary Operator      & 42 & 42  & 0    \\
Post Increment        & 27 & 27  & 0    \\
Arithmetic as Logic   & 29 & 36  & +24  \\
Pre-Increment         & 34 & 49  & +44  \\
         \bottomrule
    \end{tabular}
\end{table}

\begin{figure}[htb!]
      \noindent
      \centering
      \includegraphics[scale=.20]{images/timeeach.png}
      \caption{Box plot of time to answer each snippet}
      \label{fig:timetoanswer}
  \end{figure}
    
Recalling our second research question of this survey (\emph{Do code snippets with atoms of confusion require programmers to take longer to predict their output?}), the variance in the time to correctly predict the output of a snippet is significantly smaller. Moreover, since the number of answers is the same for each inspected atom, data from table \ref{results_time} shows that the time taken to answer questions without atoms of confusion is 20\% smaller. Not all atoms, though, took less time to predict the answer. In fact, for the \emph{Arithmetic as Logic} and the \emph{Pre-Increment} atoms, there was an increase of 24\% and 44\% in the time taken to answer questions wherein the atoms had been removed. For other two atoms, there is no difference whatsoever in the time taken. We only find in half of our atoms an increase in the time taken to answer. Therefore, even though we are not able to confirm hypotheses H.2 defined in Section 1, further inquiry would be necessary to discover the reason for some atoms taking even longer.

\rb{talvez tenhamos que descrever e testar as hipoteses. a descricao pode ser feita na secao de study settings, e o resultado dos testes pode aparecer aqui.}

% \rb{acho que podemos melhorar a apresenta\c c\~{a}o dessas tabelas, talvez usando booktabs.}


\subsection{Results of the second study: Interview}
\label{sec:interview-results}

In this section we present the results of two rounds of interviews with practitioners who agreed to participate in the interviews.In the first interview, a total of 15 practitioners were surveyed. In this phase, we collected information regarding programming experience, familiarity with JavaScript, and we also collected their opinions regarding nine candidates to atoms of confusion. This was done by presenting them pairs of code snippets whose behavior was identical, and ask them to classify each snippets readability.

In the second round of interviews, fourteen of the fifteen developers who participated in the first round were shown another 8 candidates to atom of confusion whose frequency we observed during the repository mining phase. The same procedure of showing code that behaves exactly the same and surveying for how easy to understand each snippet was applied in this phase. 

Supporting the results of the survey,Tables \ref{tab:interview-results1} and \ref{tab:interview-results2} show that for twelve out of the seventeen scenarios surveyed, the respondents prefer the version of the code without the atom of confusion, and in no case the \emph{neutral} ratio was higher than the option for the version without atoms of confusion. An example entry is contained in Appendix~\ref{sec:appendix-atoms}. 
For a full listing of the code snippets, visit the paper repository. 

\subsubsection*{Participants views of the atoms} Interesting, only for the \emph{ternary operator} scenario the participants prefer the version of the code with the atom of confusion, instead of the more cleaner version without the atom of confusion. Even in this case, some participants who opted for the \emph{left-hand-side} (with the atom of confusion) version still believed that the \emph{right-hand-side} (without the atom of confusion) version was more readable (see Figure~\ref{code:ternary}). 
The following quotes were extracted from three interviews:

\begin{figure*}

\noindent\begin{minipage}{.45\textwidth}
\begin{lstlisting}[language=JavaScript, caption=\emph{Left-hand side} (using the \emph{Ternary Operator} atom)]
let config = {size: 3, isActive: false};
const_config = config.isActive === true 
             ? config 
             : {size: 10};
console.log(_config.size);
\end{lstlisting}
\end{minipage}\hfill
\begin{minipage}{.45\textwidth}
\begin{lstlisting}[language=JavaScript, caption=\emph{Right-hand side} (without the atom)]
let config = {size: 3, isActive: false}
let _config;
if(config.isActive === true) {
  _config = config;
}
else{
   _config = {size: 10};
}
console.log(_config.size);
\end{lstlisting}
\end{minipage}
\caption{Example of pair of code snippets used in the interview. This case explores the use of the \emph{Ternary Operator atom of confusion}}
\label{code:ternary}
\end{figure*}

\begin{mq}
\emph{``I prefer to write [code using the \lhs version], but I think [the \rhs version]  is easier to read, especially for newer programmers".}
\end{mq}

\begin{mq}
\emph{``When I am programming, I write code with the ternary operator, [...], but, to be honest, I still think that the [the code using the \lhs version] is easier to understand"}.
\end{mq}

\begin{mq}
\emph{``I think [the \lhs version] is easier to understand, but [the \rhs version] is what I would write"}
\end{mq}

This particular atom of confusion also opens up the possibility for a derivative construct that JavaScript allows which can be rather confusing, and that is the nested ternaries construct, in which the right-hand side of the a ternary operator can be another ternary construct. While nested ternaries remove the number of lines that would be necessary to construct using nested if-then-else statements, they can become quite taxing, cognitively speaking, to understand. Therefore, nested ternaries are a choice of atoms to be analysed in future work.

\rb{Poder\'{i}amos quantificar a existencia de emph{nesting} com code queries?}

The \emph{Pre-Increment} atom of confusion also caused such divide in one of the interviewees, as they regarded the \emph{\rhs version} (without atom of confusion) as simpler to understand, but would still opt to write code that contained the atom. In contrast to such opinion, one of the participants found the version that contained the atom of confusion more elegant, but recognized it was less readable, and was willing to sacrifice elegance for readability.

When analyzing the \emph{Logic as Control Flow} atom of confusion, one of the interviewees gave an example of his own experience on why one should avoid writing code with such atom:

\begin{mq}
\emph{``This one is interesting, because I have written code that looks like the [\lhs version (with the atom of confusion)], and my colleagues complained that it was difficult to understand. So nowadays I prefer to write code using the [\rhs version]."}
\end{mq}

Two of the atoms had unanimous preference for the versions that did not contain the atoms: \emph{Comma Operator} and \emph{Omitted Curly Braces and Indentation}. Regarding the first one, we could often notice during the interviews that the \emph{\lhs} (with the atom of confusion) caused significant confusion among the participants. Two remarks about the \emph{Comma Operator} atom of confusion are listed below:

\begin{mq}
\emph{``I just learned that [the \lhs version of this code] is possible. I did not even know it worked"}
\end{mq}

\begin{mq}
\emph{``The code in [the \lhs version] is unlikely to be understood unless the programmer knows \clang or \cpplang"}
\end{mq}

As for the \emph{Omitted Curly Braces and Indentation} atom of confusion, which, as we shall see in Section~\ref{sec:msr-results}, is frequent in JS code bases, one of the interviewees mentioned that, although they understand why one would opt to not use braces for simple if-then-else statements, he still advised against it, on grounds that:

\begin{mq}
\emph{``[I prefer the \rhs version of the code \ldots] If I want to see well-written, easily understandable code, then I also have to do my job. Therefore I believe that, since I do not know who is on the other end maintaining this code, and it could be any person with any level of expertise, then I try to write readable, easy-to-understand code"}
\end{mq}

This view is central to our research on why such small, yet tricky, constructs should be avoided whenever possible, since maintenance is one of the longer and most costly parts of the software development life cycle, and one cannot make any assumptions about the other programmers' experience.

\rb{que conclus\~{a}o extra\'{i}mos com essa parte do estudo?} 


\subsection{Common Remarks From the Open-ended Questions} The final remarks that were drawn from the interviews are related to potentially confusing constructs that were suggested by the participants, as well as their perspective on JavaScript as a language, from which we can also uncover some other forms in which the language itself might contribute to writing confuse code.

One of the participants mentioned that the use of JavaScript's prototype-based inheritance can make it difficult to understand code that, particularly when involving  deep prototype chains. This is a core feature of the language, and most high-level tools and frameworks abstract it away. Although this was only mentioned by one practitioner, this is an important remark, as true understanding of JavaScript software necessarily involves understanding the concept of prototypes.

When asked about particular JavaScript constructs or patterns that can make code difficult to understand, three participants cited the callback pattern, which can lead to several levels of nested function calls, as extremely difficult to assimilate. One of the respondents regarded nested callbacks difficult even to write:

\begin{mq}
\emph{``Nested callbacks are very confusing. Even writing them can be confusing, let alone understanding them."}
\end{mq}

Two other developers implicitly touched upon the callback pattern, mentioning that it can be difficult to understand 'asynchronous' programming in JavaScript. Since what is really happening at a lower level is that the JavaScript engine creates a callback stack that is separate from the main execution stack, and that callback functions are only executed when the main execution stack is empty, the concepts of asynchronous events and callbacks are inseparable in the language, and any abstractions for callback functions, such as promises and async/await syntax only hide the pattern.

% citar uninitialized objects' types. floating point precision issues, 'this', type coercion

% \subsection{Common Misunderstandings About the Language}

A total of 40\% (n = 6) practitioners regarded the JavaScript language as quick and/or easy to learn, and 26.7\%(n = 4) mentioned that the abundance of frameworks, coupled with the abstractions they provide, make it so that one can develop quickly in the language. However, as mentioned by one of the interviewees, ``\emph{Developers who have started programming in JavaScript might miss out on important lower-level aspects, such as garbage collection, memory management, and processor management.}''. 

Moreover, such abstractions can make it so that one starts to program using the language without actually understanding some of its basic aspects. For instance, 26.7\%(n = 4) said that the language is not typed, 13.3\%(n = 2) believed the language runs exclusively on browsers, and two other developers stated that the language is natively asynchronous, all of which are incorrect statements. For the multi-purpose language that is one of the most, if not the most used one by practitioners
~\cite{}, a lack of understanding of the languages core features may be a problem worth investigating in future studies.

\rb{seria importante apresentar uma conclus\~{a}o em rela\c c\~{a}o a essa parte do estudo. Acredito que seja interessante discutir um pouco essa ``faca de dois gumes'' sobre a flexibilidade, mencionada abaixo. }

\adriano{Fiquei em duvida se vale citar os contras mais interessantes que os entrevistados falaram sobre JS: (1x) o jeito como ela lida com 'this', (que esta ligado a ela ser puramente orientada a objetos, sem classe); (2x)falta de precisao para tratamento de ponto flutuante, o que pode a tornar inadequada para tratamento de dados/ciencias puras; e a faca de dois gumes que e a flexibilidade da linguagem: muitos falavam isso como ponto positivo e negativo ao mesmo tempo. \todo[inline]{eu acho que podemos mencionar sim}}

\begin{table}[!htb]
    \centering
    \caption{Interviews Round 1 - Summary of participants' preferences for code snippets \textsc{with} and \textsc{without} atoms of confusion (\textsc{aoc}). Participants were also allowed to choose \textsc{neutral} when they thought both sides were equally readable.}
    \label{tab:interview-results1}
    \begin{tabular}{lrrr}\toprule
      & \multicolumn{3}{c}{\textsc{Preference (\%)}} \\
      \cmidrule(lr){2-4}
         \textsc{atom}           & \multicolumn{1}{c}{\textsc{with aoc}}
                                      &  \multicolumn{1}{c}{\textsc{without aoc}}
                                               & \multicolumn{1}{c}{\textsc{neutral}} \\ \midrule
         Arithmetic as Logic     & 0  & 93.33  & 6.67  \\
         Assignment as Value     & 20 & 60     & 20    \\
         Comma Operator          & 0  & 100    & 0     \\
         Ternary Operator        & 60 & 26.67  & 13.3  \\
         Implicit Predicate      & 20 & 73.33  & 6.67  \\
         Logic as Control Flow   & 20 & 60     & 20    \\
         Omitted Curly Braces    & 0  & 100    & 0     \\
         Post Increment          & 20 & 73.33  & 6.67  \\
         Pre Increment           & 40 & 46.67  & 13.33 \\ \midrule
         \textsc{overall}        & 20 & 70.37  & 9.63  \\
         \bottomrule
    \end{tabular}
\end{table}

\begin{table}[!htb]
    \centering
    \caption{Interviews Round 2 - Summary of participants' preferences for code snippets \textsc{with} and \textsc{without} atoms of confusion (\textsc{aoc}). Participants were also allowed to choose \textsc{neutral} when they thought both sides were equally readable.}
    \label{tab:interview-results2}
    \begin{tabular}{lrrr}\toprule
      & \multicolumn{3}{c}{\textsc{Preference (\%)}} \\
      \cmidrule(lr){2-4}
         \textsc{atom}           & \multicolumn{1}{c}{\textsc{with aoc}}
                                      &  \multicolumn{1}{c}{\textsc{without aoc}}
                                               & \multicolumn{1}{c}{\textsc{neutral}} \\ \midrule
         Property access         & 10  & 60  & 30  \\
         Object spread           & 50 & 50     & 0    \\
         Array spread            & 60  & 40    & 0     \\
         Arrow function          & 40 & 40     & 20  \\
         Array destructuring     & 40 & 40  & 20  \\
         Object destructuring    & 40 & 40     & 20    \\
         Type conversion         & 30  & 60    & 10     \\
         Change Literal encoding & 40 & 50  & 10  \\ \midrule
         \textsc{overall}        & 42.8 & 47.6  & 9.6  \\
         \bottomrule
    \end{tabular}
\end{table}

\subsection{Results of the third study: Mining open source repositories}
\label{sec:msr-results} 

We mined \minedprojects open source JavaScript repositories to understand how often atoms of confusion arise in real software. Similarly to a previous study~\cite{DBLP:conf/msr/GopsteinZFC18}, which investigates the prevalence of atoms of confusion in open source \clang and \cpplang projects, we found that atoms of confusion frequently arise in JavaScript open source systems. 

The 3 most frequently found atoms occur in at least 80\% of the projects.
Considering the extremes, atom Ternary Operator was found in 89.69\% of the repositories, while atom Comma Operator occurred in only 11.34\% of them (as seen in Figure~\ref{fig:rate}).

% Figure~\ref{fig:rate} summarizes this finding, showing that the occurrence of atoms of confusion in JavaScript systems range from 11.34\% (Comma Operator) to 89.69\% (Ternary Operator).


%Figure~\ref{fig:atoms-occurrence} shows that the occurrence of in JavaScript systems range from 

\begin{figure}
    \centering
    \includegraphics[width=\columnwidth]{images/rate-1.pdf}
    \caption{Incidence of atoms of confusion across the \minedprojects JavaScript repositories we mined}
    \label{fig:rate}
\end{figure}

\begin{figure}
    \centering
    \includegraphics[width=\columnwidth]{images/chart_ocurrence_kloc.png}
    \caption{The rate of each atom per thousand lines of code (KLOC) across all \minedprojects repositories}
    \label{fig:atoms-occurrence}
\end{figure}


Considering all JavaScript projects in our dataset, we found 
a total of \num{206317} atoms of confusion, though three atoms are responsible for 87.47\% of this total: Post Increment Expression, Omitted Curly Braces, and the Ternary Operator. The remaining atoms accounts for \num{25851} (12.53\%) of the total number of occurrences, and the Comma Operator and and Arithmetic as Control Flow are the atoms that arise less frequently (0.25\% in total with 373 and 164 occurrences, respectively).
%In our dataset of JavaScript projects, 
We found the Ternary Operator atom occurring more frequently than the Omitted Curly Braces, differently from what has been reported in a previous work~\cite{DBLP:conf/msr/GopsteinZFC18}. 

The results of our survey and interviews suggest that the Ternary Operator does not contribute significantly as a source of misunderstanding (increasing the number of wrong answers in 9\% of the cases, according to our survey). Nonetheless, the Post Increment Expression and Omitted Curly Braces atoms are listed in the top three sources of misunderstanding 
(see Table~\ref{tab:difference-correctness}). As such, fixing the the Post Increment Expression and Omitted Curly Braces might solve most of the unclear code caused by atoms of confusion. 

\rb{continuar desse ponto}


\subsection{Discussion}
\label{sec:discussion}

\section{Discussion}
\label{sec:discussion}

Our work leads to several implications.
First, we generalize the
work of Gopstein et al.~\cite{DBLP:conf/sigsoft/GopsteinIYDZYC17},
which investigates
the impact of atom candidates on
understanding \clang and \cpplang code. That is,
similarly to their work, the atom candidates
Comma Operator, Post/Pre Increment, Omitted Curly Braces,
Assignment as Value, Implicit Predicate, Logic as
Control Flow, and Ternary Operator make the 
JavaScript code hard to understand. Our results also 
refute the hypothesis that Arithmetic as Logic is an atom of
confusion (i.e., a source of misunderstanding).
In comparison to the previous work, our study
led to some differences in the effect size
of the atom candidates, in comparison to the original
work of Gopstein et al.~\cite{DBLP:conf/sigsoft/GopsteinIYDZYC17}.
Altogether, we answer our first research question
\emph{\rqa}:

\begin{mh}
  {\bf Answer to RQ1:} The first study (survey) give evidence that
  the atom candidates for \clang and \cpplang
  that also exist for JavaScript correspond to
  a source of misunderstanding in
  JavaScript code. 
\end{mh}

The results of the interview study complement the understanding
of atoms of confusion because the participants make
clear the existence of a trade-off between code comprehension
and other quality attributes. For instance, most of the participants
prefer the version of the code with the Ternary Operator,
even though they agree that its uses might contribute to the
misunderstanding of JavaScript code, particularly when novices are maintaining
the codebase. The participants of the interview study also
mentioned other possible sources of misunderstanding in JavaScript,
including the use of prototype-inheritance and nested call-backs (as discussed in Section~\ref{sec:interview-results}). Other JavaScript atom candidates include
Object Destructuring, Array Spread, Object Spread, and Type Conversion.
In summary, the results of the second study (interviews) allow
us to answer the second (\rqb) and third research questions
(\emph{\rqc}).

\begin{mh}
  {\bf Answer to RQ2:} The qualitative analysis of the
  interviews supports the results of the first study,
  indicating that atoms of confusion compromise
  source code understanding. 
\end{mh}

\begin{mh}
  {\bf Answer to RQ3:} The qualitative analysis of the
  interviews suggest that specific constructs of
  JavaScript might also correspond to atoms of
  confusion, including prototype inheritance and
  nested callbacks. 
\end{mh}

The result of the third study (mining open source
Java script repositories) give evidence that,
although atoms of confusion compromise program
comprehension, they frequently appear in open
source JavaScript projects. In particular,
{\color{red}six, out of nine atoms considered
  in our study,} appear in more than 50\% of
the projects. To sum up, the third study
allows us to answer the fourth research
question (\emph{\rqd}).


\begin{mh}
  {\bf Answer to RQ4:} The MSR study reveals that
  the atoms of confusion explored in our research
  appear frequently in practice. 
\end{mh}



%% We conducted a non-exact replication of the three
%% studies (survey, interview, and mining software repositories)
%% considering these more specific JavaScript atom candidates.
%% We confirmed that they truly correspond to sources of misunderstanding.
%% Due to lack of space,
%% we cannot present all the results here, and we postpone the presentation of these results as future work.




\section{Threats to Validity}
\label{threat}

%Conclusion validity is connected with how well it is possible to establish relationships between treatments and outcomes. Threats to conclusion validity often come from inappropriate use of statistics. 

\textbf{Conclusion validity.} 
%Threats to conclusion validity often come from inappropriate use of statistics. 
In the context of our survey study, we apply different non-parametric statistical tests appropriate for the cases where data was categorical (correctness, Chi-square test of independence) and continuous (time, Mann-Whitney U test). Furthermore, besides reporting p-values, we also report effect size measures appropriate for each scenario (odds-ratio and Cliff's delta) and apply a p-value correction technique to avoid the multiple testing problem. Finally, it could be argued that the size of the samples is insufficient to make conclusions for some of the atoms, a common problem in Empirical Software Engineering. We estimate the sample size for each one of the atom candidates, considering that each one had a different number of samples (Table~\ref{tab:difference-correctness}). To that end, we use the $\phi$ measure of effect size for each atom (based on the Chi-square statistic), the standard $\alpha$ coefficient of 0.05, and set the expected statistical power to 0.8, as usually employed in the literature~\cite{Ellis:2010:EGE}. We find out that the sample size is sufficient for all but one of the atoms where we had a statistically significant result for correctness, Implicit Predicate. 
% For the ones where we did not find statistically significant results for correctness, in fact the sample sizes
% we have analyzed are insufficient for such small effect sizes. This indicates that further studies on these
% atoms are required, due to the likelihood of type II errors.

%Construct validity is connected with how well the selected measures actually represent the concept of interest. 

\textbf{Construct validity.} We use correctness as a proxy for program comprehension and predicted outcomes of small programs as a measure of correctness. As discussed elsewhere~\cite{Oliveira:2020:ECR}, this approach is a test of the developers' ability to trace programs. Although this is a common approach in studies about atoms of confusion~\cite{TheEyesDoNotLie,Langhout:2021:ACJ,DBLP:conf/sigsoft/GopsteinIYDZYC17}, other measures of correctness could have been employed, potentially yielding different results. As a complement to correctness in the survey, we have measured the time it took for the participants to correctly predict the outcome of the code snippets (either with or without atom candidates). Furthermore, we have  asked the interviewees about their preferences when comparing confusing and clean versions of the programs.

%A potential problem with our method is that there little incentive for subjects to think thoroughly about the questions. We observed a lack of engagement when we ran the survey with undergraduate students during a pilot study. Although our final subjects were voluntarily partaking in the survey, we could not be sure that, after some time taking the survey, respondents would become tired and stop thinking clearly about the code.


%Internal validity is concerned with how well the study isolates the variables of interest and accounts for confounding factors. 

\textbf{Internal validity.} Since the survey was conducted online with unknown participants, we have no way of confirming their levels of education and experience. We mitigate this threat by, in the data analysis, explicitly accounting for programmer experience and using an experimental design that allows us to isolate the impact of the treatment from factors such as experience and formal education level. Also, we did not have a way to prevent respondents from cheating, such as running the code on an interpreter, or consulting other people. We partially mitigate this threat by presenting images with source code, instead of text. This creates an obstacle for participants to run the code while taking the survey.

%External validity is linked to whether it is possible to extrapolate the results of the study. 

\textbf{External validity.} Our results suggest that the selected idioms and code constructs may lead to confusion for small code snippets, but it is not clear if that result extrapolates to other scenarios. Even though it is likely that in larger code bases the confusion induced by these constructs may be even greater, the existence of additional context may mitigate this effect. Another possible threat to the generalizability of our results, in particular for the survey and interviews, lies in the fact that the analyzed atoms may rarely occur in real JavaScript code bases. To mitigate that threat, we have analyzed 72 popular open source JavaScript repositories and found out that most of them are common, occurring at least once per 1,000 lines of code. Only Arithmetic as Logic and Comma Operator occur less frequently than once per 10,000 lines of code. 
% \castor{A forma como os átomos foram detectados é uma ameaça que não discutimos.}


%% Finally, in one of the atoms, namely the Omitted Curly Braces,
%% we intentionally removed indentation from the original code,
%% which is highly unusual, given that many programmers use automatic
%% formatting in their code editors. This can introduce some level
%% of artificiality to this atom's question. Nonetheless,
%% we discovered that JavaScript does allow the programmer
%% to omit the curly braces after \textit{if} statements,
%% and insert multiple statements in the following line.
%% This fact itself might constitute
%% a source of confusion, which we leave to analyse
%% in our future endeavors. 

\section{Conclusions}
\label{conclusion}

This paper reports the results of a mixed-method research effort that allowed us to better understand the impact of atoms of confusion in JavaScript code. First, we conducted a survey with 140 JavaScript developers, asking them to predict the output of some code snippets (with and without atoms). We provide evidence that five atom candidates significantly impact program comprehension activities and can be considered atoms of confusion.
%% : Comma Operator, Automatic Semicolon Insertion, Post Increment, Assignment as Value, and Implicit Predicate. 
%% We also interviewed 15 professional developers. In general, they considered code without atom candidates easier to understand and highlighted other JavaScript constructs and idioms that might also introduce misunderstanding. 
%Last, but not least,
%Finally, we conducted a mining software repositories study. We found four atom candidates frequently used in JavaScript code, i.e., Implicit Predicate, Post Increment, Ternary Operator, and Omitted Curly Braces.  %The first two can be considered atoms of confusion. 
Our efforts have two implications for practice. First, we present evidence that not all atoms of confusion validated to statically typed languages (such as \clang, \cpplang, or Java) led to a statistically significant impact on program understanding for JavaScript (a dynamically typed language). Besides that, our findings might be used to alert developers to avoid writing JavaScript code with certain atoms of confusion (e.g., Comma Operator, Automatic Semicolon Insertion, Post Increment/Decrement, Assignment as Value, and Implicit Predicate). Finally, our results might help tool developers to create program transformation tools for removing atoms that frequently appear in software. 
%They might firstly focus, for instance, on the atoms we found to be more confusing and at the same time more common in practice (e.g., Omitted Curly Braces and Post Increment).
%Promising candidates may be the atoms we found to be most confusing and, at the same time, common in practice (e.g., Implicit Predicate and Post Increment).
%As mentioned, the results of the interviews
Our research also pointed out additional JavaScript constructs and idioms 
that might introduce misunderstanding. These include nested callbacks, % and prototype-based inheritance (which might not have a counterpart version), but also idioms such
%as 
\emph{property access}, and \emph{arrow functions}. As future work, we intend to
reproduce our survey to validate whether or not these additional
\emph{atom candidates}
%, which the interviewees pointed out, 
truly affect
the understanding of JavaScript code. 


% \section{Atoms of Confusion Considered in our Research}\label{sec:appendix-atoms} 

%\begin{atom}{Arithmetic as Logic}
%\emph{Use of the result of an arithmetic expression as control logic.}

%\begin{lstlisting}[language=JavaScript]
%if(array_1.length - array_2.length){
  console.log(true);
%}
%\end{lstlisting}
%\end{atom}

%\todo[inline]{detalhar os demais atomos seguindo esse padr\~{a}o}

\begin{longtable}{| p{.50\linewidth} | p{.50\linewidth} |} 
% \begin{center}
\caption*{Code Snippets presented on the survey}
\begin{tabular}{ | p{8.2cm} | p{8.2cm} | } \hline
\multicolumn{2}{| c |}{Arithmetic as Logic}  \\ \hline
Obfuscated ID: 11  Obfuscated Answer: false  &  Transformed ID: 1  Transformed Answer: false \\ \hline
\begin{lstlisting}[language=JavaScript]
let array_1 = [1,2,3];
let array_2 = [2,3,2];
if(array_1.length - array_2.length){
  console.log(true);
}
else{
  console.log(false);
} \end{lstlisting}
& 
\begin{lstlisting}[language=JavaScript]
let array_1 = [1,2,3];
let array_2 = [2,3,2];
if(array_1.length != array_2.length){
  console.log(true);
}
else{
  console.log(false);
} \end{lstlisting} \\ \hline


\multicolumn{2}{| c |}{Assignment as value}  \\ \hline
Obfuscated ID: 12  Obfuscated Answer: 0  &  Transformed ID: 2  Transformed Answer: 0 \\ \hline
\begin{lstlisting}[language=JavaScript]
function resetSchedulerState () {
  let activatedChildren = {length: 10};
  let index = length = activatedChildren.length = 0;
  console.log(index);
}
resetSchedulerState(); \end{lstlisting}
& 
\begin{lstlisting}[language=JavaScript]
function resetSchedulerState () {
  let activatedChildren = {length: 10};
  activatedChildren.length = 0;
  let length = activatedChildren.length;
  let index = length;
  console.log(index);
}
resetSchedulerState(); \end{lstlisting} \\ \hline


\pagebreak 
\hline
\multicolumn{2}{| c |}{Automatic Semicolon Insertion}  \\ \hline
Obfuscated ID: 13  Obfuscated Answer: false  &  Transformed ID: 3  Transformed Answer: true \\ \hline
\begin{lstlisting}[language=JavaScript]
function example(){
  return
      10
}
if(example() == 10){
  console.log(true);
}
else{
  console.log(false);
} \end{lstlisting}
& 
\begin{lstlisting}[language=JavaScript]
function example(){
  return 10;
}
if(example() == 10){
  console.log(true);
}
else{
  console.log(false);
} \end{lstlisting} \\ \hline


\multicolumn{2}{| c |}{Comma Operator}  \\ \hline
Obfuscated ID: 14 Obfuscated Answer: 13  &Transformed ID: 4 Transformed Answer: 3 \\ \hline
\begin{lstlisting}[language=JavaScript]
let V1 = 5, V2 = 10;
V1 = (V2 = 1, 2);
console.log(V1+V2);
\end{lstlisting}
& 
\begin{lstlisting}[language=JavaScript]
let V1 = 5, V2 = 10;
V2 = 1;
V1 = 2;
console.log(V1+V2);
\end{lstlisting} \\ \hline

\pagebreak
\hline
\multicolumn{2}{| c |}{Ternary Operator}  \\ \hline
Obfuscated ID: 15 Obfuscated Answer: 10  &Transformed ID: 5 Transformed Answer: 10 \\ \hline
\begin{lstlisting}[language=JavaScript]
let config = {size: 3, isActive: false};
const _config = config.isActive === true ? config : {size: 10};
console.log(_config.size); \end{lstlisting}
& 
\begin{lstlisting}[language=JavaScript]
let config = {size: 3, isActive: false}
let _config;
if(config.isActive === true){
  _config = config;
}
else{
  _config = {size: 10};
}
console.log(_config.size); \end{lstlisting} \\ \hline




\multicolumn{2}{| c |}{Implicit Predicate}  \\ \hline
Obfuscated ID: 16 Obfuscated Answer: false&Transformed ID: 6 Transformed Answer: false \\ \hline
\begin{lstlisting}[language=JavaScript]
let V1 = 10, V2 = 3;
if (!(V1 % V2)){
  console.log(true);
}
else{
  console.log(false);
} \end{lstlisting}
& 
\begin{lstlisting}[language=JavaScript]
let V1 = 1, V2 = 2;
if ((V2 - V1) == 0){
  console.log(true);
}
else{
  console.log(false);
} \end{lstlisting} \\ \hline





\multicolumn{2}{| c |}{Logic Control Flow}  \\ \hline
Obfuscated ID: 17 Obfuscated Answer: 7&Transformed ID: 7 Transformed Answer: 21 \\ \hline
\begin{lstlisting}[language=JavaScript]
let V1 = 3;
let V2 = 5;
let V3 = 0;
while (V1 != V2 && ++V1) {
  V3++;
}
console.log(V1 + V3); \end{lstlisting}
& 
\begin{lstlisting}[language=JavaScript]
let V1 = 1;
let V2 = 11;
let V3 = 0;
while (V1 != V2) {
  V1++;
  V3++;
}
console.log(V1+V3); \end{lstlisting} \\ \hline



\multicolumn{2}{| c |}{Omitted Curly Braces \& Indentation}  \\ \hline
Obfuscated ID: 18 Obfuscated Answer: 4  &Transformed ID: 8 Transformed Answer: 4 \\ \hline
\begin{lstlisting}[language=JavaScript]
let V1 = 1, V2 = 2
if (V1 > V2)
V2 = 1
V1 = 2
console.log(V2+V1) \end{lstlisting}
& 
\begin{lstlisting}[language=JavaScript]
let V1 = 1, V2 = 2;
if (V1 > V2) {
  V2 = 1;
}
V1 = 2;
console.log(V2+V1); \end{lstlisting} \\ \hline



\multicolumn{2}{| c |}{Post Increment}  \\ \hline
Obfuscated ID: 19 Obfuscated Answer: true&Transformed ID: 9 Transformed Answer: true\\ \hline
\begin{lstlisting}[language=JavaScript]
let V1 = 0;
if (V1++ == 0) {
  console.log(true);
}
else {
  console.log(false);
} \end{lstlisting}
& 
\begin{lstlisting}[language=JavaScript]
let V1 = 0;
if (V1 == 0) {
  console.log(true);
}
else {
  console.log(false);
}
V1 = V1 + 1; \end{lstlisting} \\ \hline



\multicolumn{2}{| c |}{Pre Increment}  \\ \hline
Obfuscated ID: 20  Obfuscated Answer: 2  &  Transformed ID: 10  Transformed Answer: 1 \\ \hline
\begin{lstlisting}[language=JavaScript]
var index = 1;
while (++index < 10) {
  console.log(index);
  break;
} \end{lstlisting}
& 
\begin{lstlisting}[language=JavaScript]
var index = 0;
index = index + 1;
while (index < 10) {
  console.log(index);
  index = index + 1;
  break;
} \end{lstlisting} \\ \hline



\end{tabular}
% \end{center}
% \end{table}
\end{longtable}


% \section*{Acknowledgment}

\balance
\bibliographystyle{IEEEtran}
\bibliography{reference}

\vspace{12pt}
\end{document}
