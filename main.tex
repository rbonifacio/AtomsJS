\documentclass[preprint,12pt]{elsarticle}


% The preceding line is only needed to identify funding in the first footnote. If that is unneeded, please comment it out.


%\usepackage{natbib}
%\usepackage[english]{babel}
\usepackage{enumerate}
\usepackage{balance}
\usepackage{adjustbox}
\usepackage{rotating}
\usepackage{caption}
\usepackage[normalem]{ulem}
\useunder{\uline}{\ul}{}
%\usepackage{cite}
\usepackage{amsmath,amssymb,amsfonts}
\usepackage{algorithmic}
%\usepackage[backend=bibtex]{biblatex}
\usepackage{graphicx}
%\usepackage{supertabular}
\usepackage{textcomp}
\usepackage{xcolor}
\usepackage{colortbl}
\def\BibTeX{{\rm B\kern-.05em{\sc i\kern-.025em b}\kern-.08em
    T\kern-.1667em\lower.7ex\hbox{E}\kern-.125emX}}
\usepackage{listings}
\usepackage{color}
\usepackage{scalefnt}
\usepackage{multirow}
\usepackage{pifont}
\usepackage{xspace}
\usepackage{todonotes}
\usepackage{booktabs}
\usepackage{longtable}
%\usepackage{tabularx} 
\usepackage{multicol}
\usepackage{listings}
\usepackage{xspace}
\usepackage{booktabs}
\usepackage[group-separator = {,}, group-minimum-digits = 4]{siunitx}
\usepackage{amsmath,amssymb,amsfonts}
\usepackage{algorithmic}
\usepackage{graphicx}
\usepackage{graphics}
\usepackage{textcomp}
\usepackage{siunitx}
\usepackage{xcolor}
\usepackage{csquotes}
\usepackage{array}
\usepackage{listings}
\usepackage{color}
\usepackage{scalefnt}
\usepackage{multirow}
\usepackage{pifont}
\usepackage{xurl}
% \usepackage{appendix}
\usepackage[framemethod=TikZ]{mdframed}
\definecolor{oldlace}{rgb}{0.99, 0.96, 0.9}
\newmdenv [ %
 skipabove=\topsep,
 skipbelow=\topsep,
 leftmargin       = 2,
 rightmargin      = 2,
 splittopskip     = \topskip]{mh}

\newmdenv [ %
 skipabove=\topsep,
 skipbelow=\topsep,
 roundcorner = 5pt,
 leftmargin = 2,
 rightmargin = 2,
 % backgroundcolor = oldlace,
 innertopmargin = 3,
 splittopskip = 3]{mq}

\definecolor{javared}{rgb}{0.6,0,0} % for strings
\definecolor{javagreen}{rgb}{0.25,0.5,0.35} % comments
\definecolor{javapurple}{rgb}{0.5,0,0.35} % keywords
\definecolor{javadocblue}{rgb}{0.25,0.35,0.75} % javadoc
\lstset{
  basicstyle=\footnotesize\tt,        % the size of the fonts that are used for the code
  breakatwhitespace=false,         % sets if automatic breaks should only happen at whitespace
  breaklines=true,                 % sets automatic line breaking
  captionpos=b,                    % sets the caption-position to bottom
  extendedchars=true,              % lets you use non-ASCII characters; for 8-bits encodings only, does not work with UTF-8
  frame=single,                    % adds a frame around the code
  language=Java,                 % the language of the code
  keywordstyle=\bf,
  showspaces=false,                % show spaces everywhere adding particular underscores; it overrides 'showstringspaces'
  showstringspaces=false,          % underline spaces within strings only
  showtabs=false,                  % show tabs within strings adding particular underscores
  tabsize=2,                       % sets default tabsize to 2 spaces
  keywordstyle=\color{javapurple}\bfseries,
  stringstyle=\color{javared},
  commentstyle=\color{javagreen},
  morecomment=[s][\color{javadocblue}]{/**}{*/},
%  numbers=left,
%  numberstyle=\tiny\color{black},
  stepnumber=1,
  keywords={let},
}    
\AtBeginDocument{%
  \providecommand\BibTeX{{%
    \normalfont B\kern-0.5em{\scshape i\kern-0.25em b}\kern-0.8em\TeX}}}

\newcommand{\clang}{C\nolinebreak\xspace}

\newcommand{\cpplang}{C\nolinebreak\hspace{-.05em}\raisebox{.6ex}{\tiny\bf +}\nolinebreak\hspace{-.10em}\raisebox{.6ex}{\tiny\bf +}\xspace}

% TODO: Completar
\newcommand{\minedprojects}{72\xspace}


\newcommand{\rb}[1]{\todo[inline]{{\bf (rb note) }#1}}
\newcommand{\diego}[1]{\todo[inline,color=green]{{\bf (diego note) }#1}}
\newcommand{\adriano}[1]{\todo[inline,color=yellow]{{\bf (adriano note) }#1}}
\newcommand{\castor}[1]{\todo[inline,color=pink]{{\bf (castor note) }#1}}

\newcommand{\rqa}{What is the impact of the analyzed code patterns % atoms of confusion 
 on the comprehension of JavaScript code?}

\newcommand{\rqb}{Do JavaScript developers identify atoms of confusion as contributing to program misunderstanding?}

\newcommand{\rqc}{What are the particular JavaScript idioms and language constructs that might cause source code misunderstanding?}

\newcommand{\rqd}{What is the frequency of occurrence of atoms of confusion in practice (i.e., in open-source JavaScript projects)?}


\newcommand{\revised}[1]{\textcolor{blue}{#1}}


\lstset{language=C}
\lstset{
 morekeywords={printf}
}

\newcommand{\lhs}{left-hand side\xspace}
\newcommand{\rhs}{right-hand side\xspace}

\lstdefinelanguage{JavaScript}[]{Java}{
  morekeywords={var, let, console.log},
  moredelim=[is][\textcolor{darkgray}]{\%\%}{\%\%},
  moredelim=[il][\textcolor{darkgray}]{§§}
}

\lstdefinelanguage{CodeQL}[]{Java}{
  morekeywords={select, from, where, not, and}
}
\newcommand{\na}{10\xspace}

\begin{document}

\begin{frontmatter}
\newenvironment{atom}[1]
  {\mdfsetup{
    frametitle={\colorbox{white}{\space#1\space}},
    innertopmargin=10pt,
    frametitleaboveskip=-\ht\strutbox,
    frametitlealignment=\center
    }
  \begin{mdframed}%
  }
  {\end{mdframed}}

\title{An Investigation of Confusing Code Patterns in JavaScript}

\author[1]{Adriano Torres}
\author[1]{Caio Oliveira}
\author[1]{M\'{a}rcio Okimoto}
\author[2]{Diego Marc\'{i}lio} 
\author[3]{Pedro Queiroga}
\author[4,5]{Fernando Castor}
\author[1]{Rodrigo Bonif\'{a}cio}
\author[1]{Edna Dias Canedo}
\author[5]{M\'{a}rcio Ribeiro}
\author[6]{Eduardo Monteiro}

\address[1]{Computer Science Department, University of Bras\'{i}lia, Brazil}
\address[2]{Universit\'{a} Della Svizzera Italiana, Switzerland}
\address[3]{Informatics Center, University Federal of Pernambuco, Brazil}
\address[4]{Information and Computing Sciences Department, Utrecht University, The Netherlands}
\address[5]{Institute of Computing, Federal University of Alagoas, Brazil}
\address[6]{Statistics Department, University of Bras\'{i}lia, Brazil}

\begin{abstract}
  Evolving legacy code is a challenging task, particularly when the code has been poorly written or uses confuse idioms and language constructs, which might increase maintenance efforts and impose a significant cognitive load on developers. For this reason, researchers have investigated possible sources of confusion in codebases, including the impact of small code patterns (hereafter atoms of confusion) that contribute to misunderstanding the source code written in statically typed languages such as \clang, \cpplang, and Java. In this work, we investigate whether atoms of confusion identified in statically typed languages also confuse developers of a dynamically typed language. We use JavaScript as a representative example of dynamic programming languages and collect evidence from a mixed-method research effort: a survey, a set of interviews with practitioners, and an activity of mining open source JavaScript repositories (MSR). Our survey and interviews confirm that atoms of confusion lead to code that is hard to understand in JavaScript. Considering ten atom candidates, developers correctly predict the outcome of at least 15\% more cases in code snippets where atom candidates are not present, compared to alternative versions that include the atom candidates. For five of these atoms, the difference is statistically significant (accounting for p-value correction). Effect size is large for four atoms and medium for one of them. In addition, our MSR effort reveals that atom candidates are frequent and used intensively in 72 popular open-source JavaScript systems. Four atom candidates appear in 90\% of the analyzed projects, and two of them occur more than once for every 100 lines of code in the dataset. Altogether, our findings might help practitioners: (1) better understand the implications of atoms of confusion on understanding JavaScript code, (2) avoid writing code that is unnecessarily difficult to maintain, and (3) design program transformation tools that remove potential sources of misunderstanding in JavaScript code.
\end{abstract}

\begin{keyword}
  code readability, program comprehension, program understanding, atoms of confusion, JavaScript code
\end{keyword}


\end{frontmatter}

%\begin{IEEEkeywords}
%code readability, program comprehension, program understanding, atoms of confusion, JavaScript code
%datasets, neural networks, gaze detection, text tagging
%\end{IEEEkeywords}

\input{introduction}
\input{background}
\section{Research Methods}
\label{method}

The main goal of this research is to investigate the impact of atoms of confusion on JavaScript code comprehension. As such, in this paper we answer the following research questions: 

\begin{enumerate}[(RQ1)]
\item \rqa 
\item \rqb
% \item \rqc  
\item \rqd
\end{enumerate}

By answering these research questions, we can either generalize or refute the findings about atoms of confusion already discussed in the literature (goal of research questions (RQ1) and (RQ2)). 
In addition, answering the third question allows us to enrich existing catalogs about atoms of confusion and discuss how often they occur in practice. Answering these research questions also lays the foundations for the implementation of tools that can automatically transform code into cleaner versions.
%---though we postpone these results to future research work.
We conduct a mixed-methods study to answer these questions. 
It includes two independently designed and conducted experiments: a repeated measures/within subjects study (Section~\ref{sec:meth:replication}) and one using a Latin square, between subjects, counterbalancing experimental design to control for subject experience (Section~\ref{sec:meth:survey}). By using two different experiments, designed by different researchers, with different designs, and conducted with different participants, we hope to rigorously verify the impact of atoms of confusion in code comprehension activities. \revised{Conducting multiple studies, with different designs and samples, is a typical approach in other areas of science to obtain more solid evidence and also to contest evidence obtained by single study, for example, Weggemans et al.~\cite{Weggemans:2001:DCE}. We discuss this matter further in Section~\ref{sec:whytwo}}.

These two experiments explore the impact of atom candidates on understanding JavaScript code (goal of research question RQ1). They investigate whether or not programs that contain atom candidates tend to produce more misunderstanding for programmers trying to predict their outcome.% We leverage two independently-developed and conducted experiments with different designs to more  
We also conduct a set of interviews with developers (Section~\ref{sec:method:interview}) to contrast the quantitative results of the experiments with their preferences and opinions (thus addressing RQ2). Finally, we perform a repository mining study (Section~\ref{sec:method:msr}) that investigates the prevalence of some of these atom candidates in large scale, professionally-developed JavaScript software.


\section{Repeated Measures Study}\label{sec:s01}

\revised{The results of the repository mining study show that some of the atom candidates we aim to study occur very frequently. Taking that as a starting point, we conduct two experiments and an interview study to gauge their potential to cause confusion. Furthermore, since we are focusing on JavaScript code, we also analyze a number of atom candidates that are specific to this language in one of the experiments.}

The first experiment we perform uses a repeated measures, or within-subject, design~\cite{Keselman:2001:ARM}. This experiment is inspired by the work of Gopstein et al.~\cite{DBLP:conf/sigsoft/GopsteinIYDZYC17}, but targeting JavaScript instead of the C language, and including some methodological differences. It covers all the atom candidates analyzed in the work of Gopstein et al. that can be transposed to JavaScript and adds candidates that are specific to this language.
%Overall, we select 24 atom candidates in the repeated measures study. 

\subsection{Experimental design} 

In this experiment, the control variable consists of a tiny program containing a single atom candidate and the treatment is a functionally-equivalent version that does not contain the atom. The main dependent variable is whether subjects are able to correctly determine the output of the programs. The other dependent variable is the time required to correctly vs. incorrectly determine the outputs, independently of the presence of atoms. We use the same null hypothesis as Gopstein et al.: \textit{``code from both control and treatment groups can be hand-evaluated with equal accuracy''} and the alternate hypothesis that \textit{``the existence of an atom of confusion causes more errors than other code in hand-evaluated outputs.''}. For each atom candidate, we built six programs, three versions including the atom candidate and three versions not including it. 
%Each subject is exposed to four programs for every atom candidate, two including an atom and two not including an atom.  We have also built six programs for each atom candidate, three with and three without the atom. Considering the elements that can vary in a replication study, defined by Juristo and Gomez~\cite{Juristo2012}, in this study we employ the same experimental protocol and construct operationalizations. We have different population properties, experimenters, and a different site. In addition, since we focus on a different language, some atoms that exist in C cannot by construction exist in JavaScript programs. Conversely, JavaScript exhibits atom candidates that do not exist in C. 

% The design of the repeated measures study considers two treatments: the presence or absence of atom candidates within the programs.
To identify atom candidates, we start out by selecting every atom candidate from Gopstein's study that can also happen in JavaScript. That excludes atoms related to pointers and preprocessor directives. Furthermore, we conduct an informal search of \revised{JavaScript questions on StackOverflow about constructs that are not typical of other languages, such as array and object spread, solutions to Code Golf~\footnote{https://codegolf.stackexchange.com/} problems, and code style guides\footnote{AirBNB JavaScript Style Guide: https://github.com/airbnb/javascript}\footnote{Google JavaScript Style Guide: https://google.github.io/styleguide/jsguide.html} to identify new atom candidates. We leverage previous work~\cite{castor2018} on the identification of atom candidates to guide us in this informal search.} Overall, the repeated measures study analyzes 24 atoms, 15 from the original study and 9 that are new (see Tables~\ref{tab:atomsBoth} and ~\ref{tab:atomsRepeated}. 

For each atom candidate, we write six short programs, three including and three not including the atom candidate. The former are the control for the study, or the \textbf{obfuscated} versions of the programs. The alternative, atom-free, versions of the programs are the treatment, or \textbf{clean}, versions. Each subject is exposed to one obfuscated and one clean version of each atom, totalling 48 programs, and should determine their outputs. As mentioned before, we measure answer correctness and the time to answer questions correctly vs. incorrectly. We control for learning effects~\cite{Neely:1991:SPE} in three manners. First, by having multiple obfuscated and clean versions for each atom and presenting only one of each per subject. Second, by presenting the programs in a random order. Third, by presenting obfuscated and clean versions corresponding to the same atom candidate with at least 11 other programs in between, in  accordance to the original experimental protocol~\cite{DBLP:conf/sigsoft/GopsteinIYDZYC17}. 


{\bf Study instrument.} The experiment is conducted by means of a questionnaire made available as a web application. As part of this effort, we carried out an informal pilot whose main objectives were (i) to spot bugs in the application and in the data collection mechanism; (ii) to gain feedback from respondents about the user experience of the application; and (iii) to formulate an estimate about how long answering the questionnaire would take on average. Based on the feedback of the participants of the pilot, we present only one obfuscated and one clean version of each atom candidate, to reduce to total time of the study and potentially increase participation.% Initially, our plan was to present two of each. 

We organize the study in three sections. In the first one we present instructions and also a check button whose checking means users agree that all collected data will be used anonymously and solely for research purposes. The instructions explain how the study works and asks them to dedicate their attention to it. We stress to participants the importance of not using any aids during the experiment, such as online or console interpreters. 

The second section presents the programs, one at a time. For each one, there is a text box where the answer should be written and a ``Next'' button. If the subject leaves the text box empty and presses ``Next'', this is treated as a wrong answer. To reduce the likelihood of a subject attempting to execute the programs, the web application verifies whether the subject tries to change tabs or windows and presents a pop-up message if that is the case. Furthermore, it presents the programs as images, instead of text, to demotivate respondents from resorting to external resources by copying and pasting the code into an interpreter. Upon submitting an answer for a particular program, the subject is automatically led to a similar page, containing the next program.

We do not provide feedback about the time subjects take to answer each question. We also do not tell them whether their answers are correct or not. This aims at avoiding introducing bias for future respondents. Since we posted information about the study on social media platforms, possible threats could have arisen if we gave respondents instant feedback.


{\bf Study audience.} We posted the link to the questionnaire on the authors' social media platforms, on Brazilian CS departments' mailing lists, and on two programming subreddits. We explained our research purposes and asked developers to take part in the study only if they have some familiarity with JavaScript. We collected 70 responses. %The average survey respondent has a mean 26 months of experience in JavaScript, with a median of 12.\castor{We should not say this here unless we are able to do the same for the Latin square study. }


\subsection{Results}\label{sec:repeated}

In this experiment, we investigate whether 24 atom candidates impact readability. We examine how the performance of the participants differ when predicting the outputs of the clean and obfuscated versions of small code snippets (Section~\ref{sec:repeated:correct}). Based on those results we establish which candidates can actually be considered atoms of confusion. In addition, similarly to Gopstein et al.~\cite{DBLP:conf/sigsoft/GopsteinIYDZYC17}, we verify whether there is a trade-off between correctness and time irrespective of the presence of atoms; we check if participants who correctly predict the outputs of the programs tend to do so more slowly than the ones that make incorrect predictions (Section~\ref{sec:repeated:time}). 

This study had 70 participants. Out of these, 67 have received university-level training in programming, 4 hold a PhD degree, 7 hold a master's degree, and 11 hold a bachelor's degree. 
The median programming experience of the participants is four years (mean 6.5 years) and the median experience with JavaScript programming is one year (mean 2.5 years). 
% General programming: median: 48, 1st quartile: 22,5, 3rd quartile: 91,5
% JavaScript programming: median: 12, 1st quartile: 4, 3rd: 24

%-----------------------------------------------------------------------
\subsubsection{Correctness Analysis}\label{sec:repeated:correct}

{\bf Exploratory Data Analysis.}
For each of the 24 atom candidates, the participants predicted the outputs of two versions, one clean (control) and one obfuscated (treatment), as per the definition of a repeated measures design~\cite{Hunter-Experimenters}. Thus, each participant examined 48 code snippets.  We also measured the time required by the participants to provide their answers. For most cases, participants made more mistakes when attempting to predict the outcomes of the obfuscated versions. 

Table~\ref{tab:diff-correctness-repeated} presents the number of correct predictions for the clean and obfuscated versions of each atom candidate, as well as the percentage difference. In only three cases the participants predicted results correctly more often for the obfuscated versions, Arithmetic as Logic, Logic as Control Flow, and Property Access. The former two were empirically validated as atoms of confusion in the study of Gopstein and colleagues~\cite{DBLP:conf/sigsoft/GopsteinIYDZYC17}. For seven of the atom candidates, the number of correct predictions for the clean versions is more than twice the corresponding number for the obfuscated versions.  Some of these atoms are common programming patterns in JavaScript programs, such as pre- and post-increments and object destructuring. The difference was even greater for Type Conversion (+440\%) and Change of Literal Encoding (+300\%). 

\begin{table}[t!]
\caption{Summary of the correctness analysis.}
\label{tab:diff-correctness-repeated}
\centering{
  \begin{scriptsize}
\begin{tabular}{lrrr} \toprule
  Atom & Obfuscated & Clean & $\Delta$(\%)  \\ \midrule
Type Conversion           &            5 &      27 &                 +440 \\
Change of Literal Encoding &           9 &      36 &                 +300 \\
Comma Operator            &            8 &      29 &                 +262 \\
Object Destructuring      &           15 &      42 &                 +180 \\
Assignment As Value       &           18 &      43 &                 +139 \\
Pre-Increment             &           13 &      28 &                 +115 \\
Post-Increment            &            8 &      17 &                 +112 \\
Array Destructuring       &           15 &      28 &                 +87 \\
Lack of Indentation no Braces     &           30 &      50 &                 +67 \\
Omitted Curly Braces      &           35 &      51 &                 +46 \\
Object Spread             &           21 &      30 &                 +43 \\
Array Spread              &           22 &      31 &                 +41 \\
Automatic Semicolon Insertion &       15 &      21 &                 +40 \\
Infix Operator Precedence &           47 &      53 &                 +13 \\
Arrow Function            &           54 &      58 &                  +7 \\
Implicit Predicate        &           41 &      44 &                  +7 \\
Lack of Indentation With Braces   &           53 &      55 &                  +4 \\
Conditional Operator      &           60 &      62 &                  +3 \\
Constant Variables        &           59 &      61 &                  +3 \\
Dead Unreachable Repeated &           63 &      64 &                  +2 \\
Repurposed Variables      &           23 &      23 &                   0 \\
Arithmetic as Logic       &           52 &      47 &                 -10 \\
Logic as Control Flow     &           33 &      28 &                 -15 \\
Property Access           &           51 &      42 &                 -18 \\
\bottomrule  
\end{tabular}
\end{scriptsize}
}
\end{table}


{\bf Statistical analysis.}
Since this study uses a within-subject design, the data is paired and dependent. Our goal is to verify whether there is a significant difference in the performance of the same participants when examining clean and obfuscated versions of programs. The analyzed data is also dichotomous, since each participant either correctly predicts the output of a program or does not. Considering this scenario, we employ McNemar's test~\cite{McNemar:1947:NSE}, which is aimed at \textit{``judging the significance of the difference between correlated proportions''}.

The ``McNemar test'' column of Table~\ref{tab:repeated:misunderstanding} indicates the p-values for the tests. These p-values are compared against an alpha of 0.0021, after applying Bonferroni correction (0.05/24). One of the atom candidates, Constant Variables, is omitted from the table because, due to the very low number of mistakes, it is not possible to perform the test. The table shows that eight of the 24 analyzed atom candidates can be considered atoms of confusion based on the results of our study, Object Destructuring, Type Conversion, Change of Literal Encoding, Comma Operator, Lack of Indentation No Braces, Assignment as Value, Pre-Increment, and Ommitted Curly Braces. Among these eight, only Object Destructuring is JavaScript-specific. Six of the other atoms have been previously reported for the C language~\cite{DBLP:conf/sigsoft/GopsteinIYDZYC17} and six for Java~\cite{Langhout:2021:ACJ}.
% Lack of Indentation No Braces is not specific to JavaScript and has also been studied by Langhout and Aniche~\cite{Langhout:2021:ACJ}. 
For some candidates that have been previously identified as atoms, there was not a statistically significant difference. This is the case for Repurposed Variables, Logic as Control Flow, Implicit Predicate, and Conditional Operator. We examine this in Section~\ref{sec:discussion}.

We leverage the odds ratio (OR) as a measure of effect size, to quantify the meaningfulness~\cite{Ellis:2010:EGE} of these results. The odds ratio is an appropriate measure of effect size for scenarios where McNemar's test is applicable~\cite{Mangiafico:2016:SAE}. For example, the odds ratio for Type Conversion is 12. This suggests that the odds of a participant correctly predicting the output of a clean version while incorrectly predicting the output of the corresponding obfuscated version is 12 times higher than the odds of correctly predicting the output of the obfuscated version and incorrectly predicting the output of the correct one. 

According to the thresholds established by Chen and colleagues~\cite{Chen:2010:HBB} (OR = 1.68 small, 3.47 medium, and 6.71 large), for most of the atoms where there is a statistically significant difference the odds ratio can be considered high. The value of the odds ratio for Object Destructuring is $\infty$ because no participant correctly guessed the output for the obfuscated version while incorrectly guessing the output for the clean one, which results in a denominator of 0. Omitted Curly Braces exhibits an OR of 6.3, which is borderline between medium and high~\cite{Chen:2010:HBB}. Array Destructuring and Post-Increment are two atom candidates that deserve a special note. The difference in the performance of participants examining clean and obfuscated versions of these atoms was not statistically significant, after correction. Notwithstanding, effect sizes were large and medium, respectively. This is a result that hints at practical relevance in spite of the p-values~\cite{Ellis:2010:EGE}. Further investigation is required for these two atom candidates.

\begin{table}[th!]
\caption{Hypotheses Testing (correctness). Asterisks ($^{*}$) indicate a statistically significant difference. Using Bonferroni correction, a p-value is considered statistically significant if it is lower than 0.0021.}
 \centering
 {\scriptsize 
 \begin{tabular}{lrr}
   \toprule
Atom candidate & McNemar test &      Odds Ratio \\
\midrule
Object Destructuring      &   \textbf{$<$ 0.0001*} & $\infty$ \\
Type Conversion           &   \textbf{$<$ 0.0001*} & 12.00 \\
Change of Literal Encoding   &   \textbf{$<$ 0.0001*} & 28.00 \\
Comma Operator            &   \textbf{$<$ 0.0001*} & 11.50 \\
Lack of Indentation No Braces     &   \textbf{$<$ 0.0001*} &  11.0 \\
Assignment As Value       &   \textbf{$<$ 0.0001*} &  9.30 \\
Pre-Increment              &  \textbf{0.0003*} &    16.00 \\
Omitted Curly Braces      &  \textbf{0.0009*} &  6.30 \\
Array Destructuring       &  0.0023 &     7.50 \\
Post-Increment              &  0.0225 &     5.50 \\
Array Spread              &  0.0490 &    3.30 \\
Object Spread             &  0.0636 &     2.80 \\
Property Access           &  0.0636 &     2.80 \\
Infix Operator Precedence &  0.1796 &     2.50 \\
Arrow Function            &  0.2891 &     3.00 \\
Arithmetic As Logic       &  0.3323 &  1.80 \\
Automatic Semicolon Insertion  &  0.3915 &  1.40 \\
Logic As Control Flow     &  0.4421 &  1.50 \\
%Constant Variables        &  0.5000 &     0.0 \\
Implicit Predicate        &  0.6900 &  1.30 \\
Lack of Indentation With Braces   &  0.7266 &  1.70 \\
Conditional Operator      &  0.7905 &  1.30 \\
Dead Unreachable Repeated &  1.0000 &     1.50 \\
Repurposed Variables      &   1.0000 &     1.00 \\
\bottomrule
\end{tabular}
}
\label{tab:repeated:misunderstanding}
\end{table}



%-----------------------------------------------------------------------
\subsubsection{Time Analysis}\label{sec:repeated:time}

{\bf Exploratory Data Analysis.}
In this section we investigate whether there is a trade-off between correctness and time required to predict program outputs, similarly to the analysis performed by Gopstein et al.~\cite{DBLP:conf/sigsoft/GopsteinIYDZYC17}. For this analysis, we consider only whether participants correctly predict the outputs of programs or not, independently of the presence or absence of atoms, and the time required to do so. The null hypothesis we investigate is that there is no significant difference in time between correct and incorrect predictions. Intuitively, we would expect correct responses to take longer, on the basis that participants would spend more time carefully analyzing the programs before providing an answer. 

Table~\ref{tab:repeated:time} presents the mean and median times required by the participants to provide answers. The data is based on the 1648 correct and 1572 incorrect answers collected in the study. Contrary to our intuition, the mean time required by participants to provide an incorrect answer is 25\% greater than the time required to provide a correct one and the median time is 20\% greater. The standard deviations suggest that there is great variation in the times required by different participants, for different code snippets. Notwithstanding, the standard deviation for incorrect answers is 62.58\% greater when compared to the same statistic for correct answers. 

\begin{table}[tb]
  \caption{Time in seconds required by the participants to correctly and incorrectly predict the outcomes of the programs.}\label{tab:repeated:time}
  \centering
  {\scriptsize 
  \begin{tabular}{lrrr}
    \toprule  
             & Mean  & Median & Std. Dev. \\
    \midrule
  Correct   & 35.44 & 24.36  & 38.43     \\
  Incorrect & 44.50 & 29.29  & 62.48    \\
  \bottomrule
  \end{tabular}
}
  \end{table}

{\bf Statistical Analysis.}
We employ the non-parametric Mann-Whitney U test, also known as Wilcoxon's rank-sum test, to verify the aforementioned null hypothesis, i.e., whether the times for correct and incorrect answers can be considered identical. That test produces a p-value of \textbf{0.00000046}, which shows that the difference is statistically significant and we can reject the null hypothesis. To gauge the magnitude of that difference we employ Cliff's Delta as a measure of effect size. This produces a result of \textbf{0.1027}, which suggests that, although participants tend to provide correct predictions more quickly, they do that only slightly so. 

\section{Latin Square Study}\label{sec:s02}

This study uses a different experimental design that makes it possible to control for the experience of the subjects. It employs a subset of the atom candidates used in the repeated measures study. More specifically, this study investigates ten of the atoms investigated in the repeated measures study: 
%Since JavaScript and \clang have some constructs in common (Section~\ref{back}), we first 
%select a set of 9+1 atom candidates: 
nine previously-validated atoms of confusion for the \clang language~\cite{DBLP:conf/sigsoft/GopsteinIYDZYC17} that also exist in JavaScript programs plus one atom candidate that is specific to the JavaScript language (Automatic Semicolon Insertion).  
%Our supplementary material discusses and shows examples of the atoms we consider in our research.\castor{We should probably do this here or just refer to the table that I intend to insert for the replication study.}

\subsection{Experimental design} 

The design of the Latin square study blocks two variables (subject experience and the programs) and considers two treatments: the presence or absence of atom candidates within the programs. To achieve such a design goal, controlling the effect of experience and individual programs, we resort to the \textit{Latin Square Design}~\cite{Hunter-Experimenters}. Using this design we create a 2 x 2 matrix in which each row represents a subject and each column indicates the set of programs. The design of each square (a replica) is such that no treatment is repeated in the same row or column. For example, considering that we have a set of 10
code samples $s_0, s_1, ..., s_{10}$, if a given subject (P1) is asked to predict the output of the code samples $s_0, s_1, ..., s_5$ that contain atom candidates, then, when answering questions about clean programs, they will only be presented with clean versions of the code samples $s_6, s_7,..., s_{10}$. Furthermore, a given subject (P2), who constitutes the second row of our example square, will be asked questions about the clean versions for $s_1, s_2, ..., s_5$, and will answer questions about obfuscated programs for $s_6, s_7,..., s_{10}$. By doing that, we guarantee that all versions of the programs are contained within each square, and that each configuration occurs only once within a square. Figure \ref{fig:latinsquare} offers a visual representation of the concept.

  \begin{figure}[htb!]
      \noindent
      \centering
      \includegraphics[scale=.50]{images/latin-square.pdf}
      \caption{Latin square design. Each ``square'' corresponds to 
      a replica in our study. Each replica comprises two participants (square rows, e.g., P1 and P2) 
      and two sets of programs (CS-1 and CS-2). We randomly apply the 
      treatments (atom or non-atom code) to the cells of the squares.} 
      \label{fig:latinsquare}
  \end{figure}

For each of the 10 selected atom candidates, we write one short program containing the atom. As in the repeated measures study, we call them the obfuscated versions of the programs. We also write corresponding short programs without the atom candidates and call them the clean versions of the programs. Overall, the experiment employed 20 programs, 10 obfuscated and 10 clean. In order to reduce the cognitive effort, each subject was asked to predict the output of 10 programs, five obfuscated and five clean. The order in which they are presented is randomized. By doing this, we seek to minimize the chances of subjects being aware that the current listing they are analyzing contains (or not) atoms of confusion. That is, each subject should indicate what would be the outcomes of the programs, some of them having atoms of confusion (while other programs do not). Since participants are not exposed to obfuscated and clean versions corresponding to the same atom candidate, learning effect is not possible. We measure answer correctness and the total time each participant requires to participate in the experiment.

{\bf Study instrument.} 
We implemented our experiment as a questionnaire in a web application and 
%carried out a pilot. 
%As part of this effort, we 
carried out an informal pilot.% whose main objectives were (i) to spot bugs in the application and in the data collection mechanism; (ii) to gain feedback from respondents about the user experience of the application; and (iii) to formulate an estimate about how long answering the survey would take on average.
Undergraduate students and professional colleagues took part in the pilot. Some users reported layout defects, and many reported that the landing page did not explain the study well enough. We also spotted minor issues with our routines to create and populate the latin squares. 

We organize the questionnaire in three sections. The first section aims to characterize the subjects, asking their age, education level, and programming experience. We also include a check button, whose checking means users agree that all collected data will be used solely for research purposes. In the second section, we present to the participants a small set of instructions, where we explain how the study works and ask them to dedicate their attention to it. We stress to participants the importance of not using any aids during the experiment, such as online or console interpreters. For each question page, we kept track of whether or not the subjects switch windows. 

The last section of the questionnaire presents a sequence of ten questions, each containing a program. For each question, there is a text box where the answer should be written. There is also an ``I do not know'' button, which, when clicked, leads the subject to the next question. In our setting, ``I do not know'' is treated as a wrong answer. The programs are also presented as images copied from a text editor. Upon submitting their answer for a particular question, the subjects are automatically led to a similar page, containing another program. Similarly to the repeated measures study, we do not provide feedback about time and correctness to the subjects.


% \rb{nao acho esse paragrafo necessario} 
% \adriano{o de baixo ne? eu tambem acho}

% As we mentioned before, we first wrote the code listings in a text editor, and took pictures of it. In the case of an atom of confusion that was exclusive to JavaScript, which we called \textit{Automatic Semicolon Insertion} (see Appendix~\ref{sec:appendix-atoms}), it was necessary to remove the syntax highlighter. Even though semicolons at the end of statements are optional to programmers in JavaScript, the interpreter automatically inserts them into the code. Our text editor was incorrectly highlighting a line break after a return statement, even thought it was valid JavaScript syntax. We had thus to turn the highlighter off to take the picture of this atom. 

{\bf Study audience.}
We posted the invitation to take part in the study on a JavaScript Reddit channel.\footnote{https://www.reddit.com/r/javascript/} We explained our research purposes and asked developers of any level of expertise to take part. Within twelve hours, we collected more than 150 answers, populating more than 70 replicas of the Latin Squares. We collected significant data on time taken and discrepancies in answer correctness between obfuscated and clean versions of the programs. Some inconsistencies arose while building the squares, for instance, when a user quit in the middle of the questionnaire. We discarded from our study all squares that contained incomplete rows.

%Since we had a large enough number of samples, the squares we had to discard did not impact our analysis.

\subsection{Results}\label{sec:latin}

In the latin-square study, we estimate the impact of atoms considering the same two perspectives of the previous experiment: \emph{correctness} (number of wrong answers)
and \emph{time} (how long to provide a correct answer). In particular, in the time analysis, differently from the repeated  measures study, we compare the time required by participants to submit a correct response when analyzing code snippets with and without atom candidates. This study is based on the responses of 140 participants (a total of 70 replicas). All participants had taken at least some university course or hold a bachelor degree or equivalent. In addition, 21 participants hold a master's degree and three a doctorate degree. Considering the programming experience of our respondents, 19\% have more than ten years of programming experience, 37\% have between four and ten years of experience, 37\% have between one and four years of experience, and 7\% have less than one year of programming experience.  
Accordingly, we characterize the effect of atoms of confusion in JavaScript code taking into account the perceptions of both novice and experienced developers. 

\subsubsection{Correctness Analysis}
% \castor{We need to be uniform about terminology. Are we going with ``misunderstanding rate analysis''? The problem is that it is not clear to me to which rate this refers.}

{\bf Exploratory Data Analysis.}
As mentioned in Section~\ref{sec:meth:survey}, each participant in this experiment evaluated ten code snippets, from which five were in their obfuscated versions, whilst the other five contained clean versions of the code snippets (i.e., without the atom candidates). As in the repeated measures study, the participants should provide the expected outcomes of the code snippets. Differently from that study, in this one each participant only predicts the outcome of one version of each code snippet, either clean or obfuscated. As discussed, we collect information about \emph{correctness} (whether the participant correctly predicted the program's output) and \emph{time}. 
Table~\ref{tab:difference-correctness} and Figure~\ref{fig:boxplotcorrectness} summarize the results of the correctness experiment.

Considering Table~\ref{tab:difference-correctness}, the clean versions of six code snippets present at least a 15\% improvement in answer correctness when compared with the obfuscated versions. In particular, the presence of the \emph{Comma Operator} atom exhibits the highest impact on misunderstanding. 
Frequently used constructs and idioms, such as \emph{Post Increment} and \emph{Omitted Curly Braces} (see Section~\ref{sec:msr-results}), also result in many mistakes. %introduce high degrees of confusion.
The boxplot of Figure~\ref{fig:boxplotcorrectness} shows a non-negligible decrease in the average number of incorrect answers when observing the clean versions of the code snippets. Also, the sample of responses with no atoms had almost no dispersion, which supports the argument that the clean versions of the code snippets are easier to evaluate correctly. 


% \begin{table}[htbp]
% \caption{Difference in answer correctness between confusing and non-confusing pairs}
% \begin{center}
% \begin{small}
% \begin{tabularx}
% {{\linewidth}}{l p{1.5cm} p{1.1cm} p{1.1cm} p{1.2cm} }
% \textbf{Atom} & \textbf{\%Correct} & \textbf{\%Correct} \\
% &  \multicolumn{1}{l}{With AOC} \multicolumn{2}{l}{Without AOC}  & \Delta (\%)    \\
%  \hline
% Comma Operator & 40 & 93 & +132\%                 \\     
% Automatic Semicolon  Insertion & 46 & 97 & +110\% \\
% Post Increment & 69 & 91 & +31\%         \\
% Omitted Curly Braces & 67 & 83 & +23\%   \\ 
% Assignment as Value & 80 & 97 & +21\%    \\
% Implicit Predicate & 83 & 97 & +16\%     \\
% Logic as Control Flow & 59 & 68 & +15\%  \\
% Ternary Operator & 86 & 94 & +9\%        \\
% Pre-Increment & 71 & 76 & +7\%           \\
% Arithmetic as Logic & 91 & 90 & -1\%    \\
% \end{tabularx}
% \end{small}
% \end{center}
% \label{results_correctness}
% \end{table}

\begin{table}[htbp]
\caption{Summary of the correctness analysis}
\label{tab:difference-correctness}
\centering{
  {\scriptsize
  \begin{scriptsize}
\begin{tabular}{lrrr} \toprule
  Atom & Obfuscated & Clean & $\Delta$(\%)  \\ \midrule
  Comma Operator                &  28 &  65 & +132   \\ 
  Automatic Semicolon Insertion &  32 &  68 & +112   \\ 
  Post-Increment                &  48 &  64 & +33    \\ 
  Omitted Curly Braces          &  47 &  58 &  +23   \\
  Assignment as Value           &  56 &  68 &  +21   \\ 
  Implicit Predicate            &  58 &  68 &  +17   \\ 
  Logic as Control Flow         &  41 &  48 &  +17   \\ 
  Conditional Operator              &  60 &  66 &  +10   \\ 
  Pre-Increment                 &  50 &  53 &  +6    \\ 
  Arithmetic as Logic           &  64 &  63 &  -2    \\ \bottomrule
  
\end{tabular}
\end{scriptsize}
}}
\end{table}

 %% Comma Operator          & 40 & 93  & +132 \\
 %% Post Increment          & 69 & 91  & + 31  \\
 %% Omitted Curly Braces    & 67 & 83  & +23 \\
 %% Assignment as Value     & 80 & 97  & +21 \\
 %% Implicit Predicate      & 83 & 97  & +16 \\
 %% Logic as Control Flow   & 59 & 68  & +15 \\
 %% Ternary Operator        & 86 & 94  & +9  \\
 %% Pre-Increment           & 71 & 76  & +7  \\
 %% Arithmetic as Logic     & 91 & 90  & +1  \\ \bottomrule


\begin{figure}[b!]
\noindent
 \centering
 \includegraphics[width=0.6\columnwidth]{images/wrong-answers-plot-1.pdf}
 \caption{Number of wrong answers of each subject.}
 \label{fig:boxplotcorrectness}
 \end{figure}% \castor{Fiz de tudo para ajeitar a figura porque o texto embaixo deveria ser "Clean versions" e "Confusing versions" mas não sei porque não funcionou. Até mudei a figura. Não sei o que está havendo.}
 
% \rb{verificar a consistencia nas questoes de pesquisa mais especificas}

{\bf Statistical analysis.}
We first use the \emph{Pearson's Chi-squared test}
to investigate if there is a statistically significant difference in the frequency of correct and incorrect answers---due to the versions of the code snippets (obfuscated and clean code). The p-values for these tests are reported in the ``Chi-square test'' column of Table~\ref{tab:hypothesis-testing}. For five atom candidates (Comma Operator, Automatic Semicolon Insertion, Post-Increment, Assignment as Value, Implicit Predicate), the results indicate that the obfuscated versions of the code snippets have a negative impact on code understanding (p-value $< 0.05$). This result holds even after applying the Benjamini-Hochberg correction with a false discovery rate of 5\%. 

We measure the effect size of the clean version of the code snippets into the answers' correctness using the \emph{Odds Ratio} (OR). The results are reported in the ``Odds Ratio'' column of Table~\ref{tab:hypothesis-testing}. 
In the table, we also report the confidence interval (CI) for the OR in the ``CI'' column. Although many of the intervals are wide, for six of the atom candidates the lower bound of the confidence interval is greater than or equal to 1. This indicates that, at a 95\% confidence level, a developer is likely to commit less mistakes when using the clean versions. 
%Considering only the atoms for which there was a significant difference in correctness for confusing and clean versions and u
%Using the thresholds established by Chen and colleagues~\cite{Chen:2010:HBB} (OR = 1.68 small, 3.47 medium, and 6.71 large), 
For four of the atoms, Comma Operator, Automatic Semicolon Insertion, Assignment as Value, and Implicit Predicate, effect size can be considered large. For Implicit Predicate, it is medium. 
%We found a negligible effect size for the atom candidates Arithmetic as Logic, Logic as Control Flow, and Pre-Increment. Nonetheless, for the remaining candidates, the effect size is significant. 
For instance, when comparing clean and obfuscated code snippets pertaining to the Comma Operator atom candidate, we observed an \emph{Odds Ratio} of \num{19.02}. This means that the odds of a correct answer are \num{19.02} times higher when interpreting the clean version---in comparison with the corresponding obfuscated version of the code snippet. Furthermore, with a 95\% confidence level, the odds ratio is between 6.6 and 68.22, i.e., although the error margin is wide, it is strongly in favor of the clean version. If there is a significant likelihood of a developer committing a mistake when analyzing a clean version, the lower bound of the CI should be a number between 0 and 1 (since the OR is a ratio). This is the case for the four atom candidates in the lower part of the table. 
%In particular, in the last row (Arithmetic as Logic), the OR between 0 and 1 indicates that a participant was actually more likely to misunderstand the clean version. 

Finally, we also use a \emph{Binomial Generalized Logistic Regression} analysis to investigate if either \emph{participant education} or \emph{participant experience} impacts correctness. The findings suggest that \emph{participant experience} impacts the results related to correctness (p-value $<$ 0.001 and $\chi^2 = 24.18$).
Figure~\ref{fig:correctness-over-experience} summarizes the number of correct and incorrect answers over the four experience groups. Even though the presence of atoms of confusion reduces the number of correct answers for all \emph{experience groups}, this effect is not uniform. For instance, novice developers (under one year of experience) provide a higher number of wrong answers for the obfuscated version of the code (57.8\%), while developers with more than four years of experience provide more than 70\% of correct answers even when evaluating the obfuscated version of the code snippets. It is also important to note that, independently of the \emph{experience group}, developers tend to provide more than 84\% of correct answers while predicting the outputs for the clean versions of the code. We replicate the \emph{Pearson's Chi-squared test} for each experience group and find that the statistical difference is less significant for those developers with more than ten years of experience (p-value = 0.007 and $\chi^2$ = 7.08). Differently from \emph{participant experience}, the factor \emph{participant education} does not significantly change the correctness of the answers (p-value = 0.236 and $\chi^2$ = 5.54).

% \castor{I'd like to have the numbers for the results above. What are the p-values, regression coefficients, etc.?}

\begin{figure}
  \centering
  \includegraphics[scale=0.5]{images/correctness-by-experience}
  \caption{Impact of atoms of confusion on the correctness (over the four participant experience groups)}
  \label{fig:correctness-over-experience}
\end{figure}


%% \begin{mh}
%%   The results of our exploratory data analysis and   hypothesis testing suggest that the atom candidates Comma Operator, Automatic Semicolon Insertion, Post Increment, Omitted Curly Braces, Assignment as Value, and Implicit Predicate %, and Ternary Operator
%%   introduce some degree of misunderstanding in JavaScript code. %Based on the results of this study, 
%%   Therefore, they can be deemed atoms of confusion for JavaScript.
%% \end{mh}

%% \begin{mh}
%% {\color{red}Comparing the results of different studies,
%%   the impact of atoms candidates on source code
%%   (mis)understanding differ according to the programming
%%   language.}
%% \end{mh}
%% Regarding the first question we 
%% address in the survey (\emph{Do code snippets that contain atoms of confusion produce a higher error rate than snippets where the atom is removed?}), we found evidence that the atoms of confusion lead programmers to misunderstand JavaScript code. We also realized that just one atom whose correction has a non-significant improvement in the percentage of correct answers---we found an improvement of at least 15\% in the correct answers when removing the confusing code for seven atoms (out of ten atoms we consider in the survey). 

\subsubsection{Time Analysis}

{\bf Exploratory Data Analysis.}
Table \ref{tab:difference-time-taken} shows the average time necessary for the participants to give a correct answer about the expected outcomes of a code snippet, considering both obfuscated and clean versions. We do not consider cases where the participants made mistakes. Seven atom candidates required more time from participants to predict a correct response. For these seven atom candidates, developers take at least 5.90\% less time on average to find a correct answer when considering the clean version of a code snippet. In the extreme case (atom candidate Comma Operator), the participants take 76.23\% less
time on average to find the correct answer for the clean version of the code snippets. 
Not all atom candidates, though, require less time for the participants to predict the answer. In fact, for  Arithmetic as Logic and Pre-Increment, the time to give a correct answer for the clean versions was   
29.06\% and 38.19\% smaller than for the obfuscated versions---we did not confirm these atom candidates as atoms of confusion in the previous section.

\begin{table}[htbp]
\caption{Time in seconds to submit a correct answer}
\centering{
  {\scriptsize
  \begin{scriptsize}
\begin{tabular}{lrrr} \toprule
Atom & Obfuscated  & Clean & $\Delta$(\%) \\ \midrule
Comma Operator & 87.67 & 20.84 & -76.23 \\ 
Automatic Semicolon Insertion & 46.08 & 22.04 & -52.17 \\ 
Post-Increment & 28.70 & 25.67 & -10.56 \\ 
Omitted Curly Braces & 48.85 & 30.00 & -38.58 \\ 
Assignment as Value & 52.47 & 48.95 & -6.71 \\ 
Implicit Predicate & 36.24 & 24.01 & -33.75 \\ 
Logic as Control Flow & 108.94 & 51.07 & -53.12 \\ 
Conditional Operator & 41.80 & 39.34 & -5.90 \\ 
Pre-Increment & 30.71 & 42.45 & 38.19 \\  
Arithmetic as Logic & 28.82 & 37.20 & 29.06 \\ \bottomrule
\end{tabular}
\label{tab:difference-time-taken}
\end{scriptsize}
}}
\end{table}

 % latex table generated in R 4.0.4 by xtable 1.8-4 package
 % Thu Apr 22 07:58:21 2021
 \begin{table*}[ht]
\caption{Hypotheses Testing (misunderstanding and time). Asterisks ($^{*}$) indicate a statistically significant difference.}
 \centering
 {\scriptsize
 \begin{tabular}{lrrr|rr}
   \toprule
Atom  & Chi-square & Odds          & Confidence & Mann-Whitey& Cliff's\\ 
 &  test &Ratio          & Interval &  U test & Delta \\  \midrule
Comma Operator & \textbf{$<$ 0.0001*} & 19.02 & (6.60, 68.22) & \textbf{$<$ 0.0001*} & -0.53 \\ 
Automatic Semicolon Insertion & \textbf{$<$ 0.0001*} & 39.33 & (9.21, 356.62) & 0.0980 & -0.16 \\ 
Post-Increment & \textbf{0.0015*} & 4.83 & (1.73, 15.74) & 0.1201 & 0.15 \\ 
Omitted Curly Braces & 0.0510 & 2.35 & (1.00, 5.77) & 0.9253 & -0.01 \\ 
Assignment as Value & \textbf{0.0035*} & 8.39 & (1.81, 79.12) & 0.8268 & 0.02 \\ 
Implicit Predicate & \textbf{0.0112*} & 6.95 & (1.46, 66.56) & \textbf{0.0029*} & -0.29 \\ 
Logic as Control Flow & 0.2920 & 1.54 & (0.73, 3.28) & \textbf{$<$ 0.0001*} & -0.39 \\ 
Conditional Operator & 0.1590 & 2.73 & (0.74, 12.57) & 0.2407 & -0.12 \\ 
Pre-Increment & 0.7015 & 1.25 & (0.55, 2.85) & \textbf{0.0062*} & 0.27 \\ 
Arithmetic as Logic & 1 & 0.84 & (0.22, 3.12) & \textbf{0.0172*} & 0.23 \\ 
    \bottomrule
 \end{tabular}}
 \label{tab:hypothesis-testing}
 \end{table*}


{\bf Statistical analysis}
We use the \emph{Mann-Whitney U test} to
investigate the null hypothesis that developers 
spend the same amount of time to correctly
predict the outcome of clean and obfuscated versions of a code snippet.
%, regardless of evaluating the confusing or clean version of the code.
The results of Table~\ref{tab:hypothesis-testing}
suggest that we should refute the null hypotheses for five atom candidates (Comma Operator, Implicit Predicate, Logic as Control Flow, Pre-Increment, and Arithmetic as Logic), after applying the Benjamini-Hochberg correction with a false discovery rate of 5\%. For the first three, the analysis suggests that the participants need more time to predict the outcome of the code snippet in the obfuscated code version. For the atom candidates Arithmetic as Logic and Pre-Increment, the participants take less time to predict the outcome of the code snippets in the obfuscated version. We also computed the effect size using Cliff's Delta statistic. In the table, negative effect sizes suggest that it took less time to correctly predict the output of clean versions of the snippets.  Considering the thresholds established by Romano et al.~\cite{Romano:2006:ASO} (0.147 $<$ $|$d$|$ $<$ 0.33 small, $|$d$|$ $<$ 0.474 medium, otherwise large), we found a large effect for the Comma Operator atom candidate; a medium effect size for Logic as Control Flow, and small effect sizes for Automatic Semicolon Insertion, Implicit Predicate, Arithmetic as Logic, Post-, and Pre-Increment. 


%% \begin{mh}
%%   The results suggest that developers tend to spend
%%   more time to predict the correct answer in the presence of the atom candidates Comma Operator, Logic as Control Flow, and Implicit Predicate. Conversely, the code snippets with the atom candidates Arithmetic as Logic and Pre-Increment demand less time to predict the correct answers. 
%% \end{mh}


%%  Comma Operator        & 60 & 21  & -65  \\
%% Logic as Control Flow & 85 & 49  & -42  \\
%% Implicit Predicate    & 33 & 24  & -27  \\
%% Omitted Curly Braces  & 43 & 31  & -27  \\
%% Assignment as Value   & 53 & 49  & -7   \\
%% Ternary Operator      & 42 & 42  & 0    \\
%% Post Increment        & 27 & 27  & 0    \\
%% Arithmetic as Logic   & 29 & 36  & +24  \\
%% Pre-Increment         & 34 & 49  & +44  \\


% \rb{acho que podemos melhorar a apresenta\c c\~{a}o dessas tabelas, talvez usando booktabs.}



\section{Interview Study}\label{sec:s03}

To complement the two experiments, we perform semi-structured interviews with professional JavaScript developers, aiming to identify their perceptions regarding programs containing atoms of confusion (research question RQ2). We also ask each participant if they know of any other JavaScript-specific construct or idiom they think is likely to make the code hard to understand. % (research question RQ3). 
In this section we detail the protocol we followed to conduct the interviews and to analyze the results.

% Nós realizamos entrevistas semi-estruturas com o objetivo de identificar a percepção dos desenvolvedores com experiência em JavaScript sobre algumas questões relacionadas à compreensão de código em JavaScript. Assim, nesta Seção nós descrevemos os procedimentos adotados para selecionar os participantes para as entrevistas e detalhamos como as entrevistas foram conduzidas. Além disso, detalhamos como os resultados das entrevistas foram analisados.

\subsection{Participant Selection} We invite the participants of the interviews using a snowballing technique. That is, starting from our network of contacts, we invite an initial set of candidates to take part in our experiments. From this initial list, we ask for an indication of additional candidates. Our main selection criterion is that all participants should have been working with JavaScript in their daily professional activities. We invited a total of 17 developers, and 15 of them agreed to participate.   
%We conducted the interviews for a period of two weeks.


\subsection{Interview Process} We conduct semi-structured interviews using web conferencing software. We record all the interviews with the consent of the participants. On average, the interviews last 26.29 minutes, with the shortest one lasting 14.59 minutes and the longest one 43.06 minutes. Two of the researchers conduct the interviews, and a third one listens to all the recordings to cross-validate the collected data. The interviews have three main parts. In the first one, we ask the developers the following demographic information: name, email, gender, level of education, current job position, JavaScript experience in years, and other programming languages they have worked with. Table \ref{pinterview} summarizes this demographic information.

\begin{table}[htb!]
  
  \centering
  \caption{Demographic information of the participants}
\begin{scriptsize}  
\begin{tabular}{clcl}
\toprule
ID & Education & JS Experience & Other Languages \\ \midrule 
P1 & BSc Degree & 9 years & Java, PHP, C, Go  
\\ 
P2 & HS Degree & 3 years & Python, Go, Dart, Lua, C++, C\#
\\ 
P3 & BSc Degree & 4 years & Java, C
\\ 
P4 & Undergraduate & 3 years & Python, C, C++, Java, Go
\\ 
P5 & Master Student & 3 years & Python, SQL
\\ 
P6 & Bsc Degree & 15  years  & Java, PHP, C, Python , Ruby, C\#
\\ 
P7 & BSc Degree  & 6 years & C, C++, Java, Assembly, Kotlin
\\ 
P8 & PhD Degree & 2 years & Java, Python
\\ 
P9 & BSc Degree  & 5 years & PHP
\\ 
P10 & Master Student & 4 years & Java, C\# and Python
\\ 
P11 & Master Student & 4 years & Java, Erlang, C\#, Cobol
\\ 
P12 & Master Degree & 1 year & C
\\ 
P13 & Master Degree & 13 years & Java, PHP
\\ 
P14 & Master Student & 3 years & C, Python, Ruby
\\ 
P15 &  BSc degree  & 2 years & Java, PHP
\\ \bottomrule
\end{tabular}
\end{scriptsize}
    \label{pinterview}
\end{table}

In the second part of the interview our aim is to allow the subjects to describe their JavaScript experience, as well as to allow them to reveal any JavaScript constructs they regard as innately confusing. This allows us to identify potential atom candidates that are more specific to the JavaScript language. Examples of questions we explore in this section include: \emph{Does JavaScript favor developers to produce code that is hard to understand?} and \emph{Do you regard any particular construct or idiom of the language as especially confusing?}

In the third part of the interview, participants are shown pairs of programs that were used in the experiments, where each pair contains a clean and an obfuscated version of the same atom candidate. For this part we only use atoms that appear in both experiments. The participants 
are asked to evaluate which version of the code is easier to understand. To avoid introducing bias in the answers, the interviewers do not explain that one of the versions in each pair contains the atom under investigation. Subjects are just presented the pairs of programs and allowed to take the necessary time to decide on the most readable one.

\subsection{Interview Analysis}

We first transcribe each interview and then examine the broad distribution of the answers. Our goal is to build an initial understanding of the participants' perceptions with regards to the challenges to understand JavaScript code in general and JavaScript code with atoms of confusion in particular. We follow-up with an open-coding procedure, highlighting the main themes and quoting the answers of the participants. We present these results in Section~\ref{sec:interview-results}. 


\subsection{Results}\label{sec:interview-results}

In this section we present the results of the interviews with the practitioners. We contrast the findings of the experiments of Sections~\ref{sec:repeated} and~\ref{sec:latin} with the opinion of software developers about their preferences regarding the obfuscated or clean versions of the code snippets. For the interview study, we only considered atoms that were analyzed in both the repeated measures (Section~\ref{sec:repeated}) and the latin square (Section~\ref{sec:latin}) studies. 

A total of 15 practitioners took part of the interviews.
We collected information regarding programming
experience, familiarity with JavaScript, and their opinion about the \na atom candidates we explored in the both experiments. We first presented them pairs of clean and obfuscated code snippets and then asked them to discuss their preferences towards one version. We did not indicate in any way whether any version was assumed to be confusing or not. 

\begin{table}[!htb]
    \centering
    {\scriptsize
    \caption{Summary of participants' preferences for code snippets \emph{with} and
      \emph{without} atom candidates. The participants were only presented the code snippets, without any indication about whether one was confusing or not. 
      Participants were also allowed to choose Neutral when they thought both sides were equally readable.}\label{tab:interview-results1}
    \begin{tabular}{lrrr}\toprule
      & \multicolumn{3}{c}{\textsc{Preference (\%)}} \\
      \cmidrule(lr){2-4}
         Atom           & \multicolumn{1}{c}{Obfuscated}
                                      &  \multicolumn{1}{c}{Clean}
                                               & \multicolumn{1}{c}{Neutral} \\ \midrule
         Comma Operator                  & 0  & 100    & 0     \\
         Automatic Semicolon Insertion  & 0  & 80     & 0     \\
         Post-Increment                  & 20 & 73.33  & 6.67  \\
         Omitted Curly Braces            & 0  & 100    & 0     \\
         Assignment as Value             & 20 & 60     & 20    \\
         Implicit Predicate              & 20 & 73.33  & 6.67  \\
         Logic as Control Flow           & 20 & 60     & 20    \\
         Conditional Operator                & 60 & 26.67  & 13.3  \\
         Pre-Increment                   & 40 & 46.67  & 13.33 \\ 
         Arithmetic as Logic             & 0  & 93.33  & 6.67  \\ \midrule
         \textsc{overall}                & 18 & 71.33  & 8.64  \\
         \bottomrule
    \end{tabular}
    }

\end{table}

\subsubsection{The Participants' perceptions of the atom candidates} 

Supporting the results of Sections~\ref{sec:repeated} and~\ref{sec:latin}, Table~\ref{tab:interview-results1}
 shows that for eight out of the
10 scenarios surveyed, the majority of the respondents prefer the version of the
code without the atom candidate.
In no case the \emph{neutral}
ratio was higher than the option for the clean version.
%% An example entry is contained in Appendix~\ref{}. 
%% For a full listing of the code snippets, visit the paper repository. 
Only for the Conditional Operator atom candidate the participants preferred the obfuscated version, instead of the clean one. Figure~\ref{code:ternary} shows the code snippets for the corresponding obfuscated and clean versions. For this atom candidate, some participants who preferred the left-hand side version (obfuscated) still believed that the right-hand side version (clean) was more readable. The following quotes were extracted from the transcripts with
% three interviewees:
two interviewees:

\begin{figure*}

\noindent\begin{minipage}{.45\textwidth}
\begin{lstlisting}[language=JavaScript, caption=\emph{Left-hand side} (using the \emph{Conditional Operator} atom).]
let config = {size: 3, isActive: false};
const_config = config.isActive === true 
             ? config 
             : {size: 10};
console.log(_config.size);
\end{lstlisting}
\end{minipage}\hfill
\begin{minipage}{.45\textwidth}
\begin{lstlisting}[language=JavaScript, caption=\emph{Right-hand side} (without the atom).]
let config = {size: 3, isActive: false}
let _config;
if(config.isActive === true) {
  _config = config;
}
else{
   _config = {size: 10};
}
console.log(_config.size);
\end{lstlisting}
\end{minipage}
\caption{Example of a pair of code snippets used in the interview pertaining to the \emph{Conditional Operator} atom candidate}
\label{code:ternary}
\vspace{-0.2cm}
\end{figure*}

\begin{mq}
\emph{``I prefer to write [code using the \lhs version], but I think [the \rhs version] is easier to read, especially for newer programmers''}.
\end{mq}

\begin{mq}
\emph{``When I am programming, I write code with the conditional operator, [...], but, to be honest, I still think that 
[the code using the \rhs version] 
is easier to understand''}.
\end{mq}

%% \begin{mq}
%% \emph{``I think [the \lhs version] is easier to understand, but [the \rhs version] is what I would write"}
%% \end{mq}

%% The atom candidate Ternary Operator also opens up
%% the possibility for a derivative construct that JavaScript
%% allows which can be rather confusing, and that is the nested
%% ternaries construct, in which the right-hand side of the a
%% ternary operator can be another ternary construct.
%% While nested ternaries remove the number of lines that
%% would be necessary to construct using nested if-then-else statements,
%% they can become quite taxing to understand.
%% Nested ternaries are a choice of atoms to be analyzed in future work.

The Pre-Increment atom candidate also caused a
conflict in one of the interviewees, who
regarded the clean version as simpler to understand, but would still opt to write
code with the atom. In contrast, %to such opinion,
one of the participants found the version with the atom candidate more elegant, but recognized it was less readable, and was willing to sacrifice elegance for readability. In the repeated measures study, we found a significant difference in favor of the clean version, with a large effect size in its favor. In the latin square study, no difference could be observed in terms of correctness. However, there was a significant difference in terms of the time required to predict the output correctly in favor of the obfuscated version. 
%% \rb{pela discuss\~{a}o anterior, parece-me que alguns
%%   desenvolvedores reconhecem que esses candidatos
%%   a \'{a}tomos introduzem certa dificuldade de compreens\~{a}o;
%%   mas mesmo assim optam por usar a constru\c c\~{a}o correspondente.
%%   Acho que cabe um box aqui com essa discuss\~{a}o sobre isso.}

When analyzing the Logic as Control Flow atom candidate, one of the interviewees gave an example of personal experience that might motivate one to avoid writing code using this construct:

\begin{mq}
  \emph{``This one is interesting, because I have written code that looks like the left-hand side [obfuscated version], and my colleagues complained that it was difficult to understand. Nowadays I prefer to write code using the version on the right-hand side [clean version]''}.
\end{mq}
\noindent
None of the two studies found a significant difference in correctness between clean and obfuscated versions of snippets related to this atom candidate. However, in the latin square study the participants analyzing clean versions of the snippets required significantly less time. 

For two of the atom candidates, the interviewees were unanimous in their preference for the clean version: Comma Operator and Omitted Curly Braces. Not coincidentally, in both experiments there were significant differences in correctness in favor of the clean versions for both atoms. Regarding the first one, we could often notice during the interviews that the \emph{\lhs} (with the atom of confusion) caused significant confusion among the participants. One remark about the \emph{Comma Operator} atom of confusion is listed below:

%% \begin{mq}
%% \emph{``I just learned that [the \lhs version of this code] is possible. I did not even know it worked''}
%% \end{mq}

\begin{mq}
\emph{``The code in [the \lhs version] is unlikely to be understood unless the programmer knows \clang or \cpplang''}.
\end{mq}

As for the atom candidate Omitted Curly Braces, one of the interviewees mentioned that, although they understand why one would opt not to use braces for simple if-then-else statements, they still advised against it, on grounds that:

\begin{mq}
\emph{``[I prefer the \rhs version of the code \ldots]
If I want to see well-written, easily understandable code, then I also have to do my job. Therefore I believe that, since I do not know who is on the other end maintaining this code, and it could be any person with any level of expertise, then I try to write readable, easy-to-understand code''}.
\end{mq}

%This is an interesting perspective, because such a perhaps simple decision,  might hinder novice developers in the task of maintaining code; and one cannot make any assumption about the programmers' experience of who is going to maintain a code in a long run.

%This is an interesting perspective because such a simple decision might hinder novice developers in maintaining code; one cannot assume the programmers' experience of who will maintain a code in the long run.



\subsubsection{Confusion in JavaScript Code} 

% The final remarks that were drawn from the interviews are related to potentially confusing constructs that were suggested by the participants, as well as their perspective on JavaScript as a language, from which we can also uncover some other forms in which the language itself might contribute to writing confusing code.

The final remarks drawn from the interviews are related to potentially confusing constructs suggested by the participants and their perspectives on JavaScript as a language. From the latter, we can also uncover some other forms in which the language itself might contribute to writing confusing code.

One of the participants mentioned that the use of JavaScript's prototype-based inheritance can make it difficult to understand code, particularly when involving deep prototype chains.
%This is a core feature of the language, and most high-level tools and frameworks abstract it away.
%Although this was only mentioned by one practitioner, this is an important remark, as true understanding of JavaScript software necessarily involves understanding the concept of prototypes.
% When asked about particular JavaScript constructs or patterns that can make code difficult to understand, three participants cited the callback pattern, which can lead to several levels of nested function calls, as extremely difficult to assimilate.
When asked about particular JavaScript constructs or patterns that can make code difficult to understand, three participants cited the callback pattern---potentially leading to several levels of nested function calls---as extremely difficult to assimilate. One of the respondents stated:

\begin{mq}
\emph{``Nested callbacks are very confusing. Even writing them can be confusing, let alone understanding them.''}
\end{mq}

Two other developers implicitly touched upon the callback pattern, mentioning that it can be difficult to understand \emph{asynchronous} programming in JavaScript.
%Since what is really happening at a lower level is that the JavaScript engine creates a callback stack that is separate from the main execution stack, and that callback functions are only executed when the main execution stack is empty, the concepts of asynchronous events and callbacks are inseparable in the language, and any abstractions for callback functions, such as promises and async/await syntax only hide the pattern.
We also found other idioms that are JavaScript atom candidates, including
\emph{property access} via array subscription (\lstinline[language=javascript]{V1['P1']} instead of \lstinline[language=javascript]{V1.P1}) and \emph{arrow functions} (see listings~\ref{arr1} and~\ref{arr2}). The former was investigated in the repeated measures study but we did not find a significant difference between clean and obfuscated versions. As future work, we aim at investigating whether or not these atom candidates are more likely to introduce confusion in JavaScript code.

\begin{figure}
\begin{small}
\begin{lstlisting}[language=JavaScript,caption=Example of arrow function.,label=arr1]
let inc = (x) => x + 1;
\end{lstlisting}
\begin{lstlisting}[language=JavaScript,caption=Alternative version without arrow function.,label=arr2]
let inc = function(x) {
  return x + 1
}
\end{lstlisting}
\vspace*{-0.6cm}
\end{small}
\end{figure}

\section{Prevalence Study}\label{sec:s04}

%To understand how often the analyzed atom candidates appear in open source JavaScript projects, and thus answer our third research question (\emph{What is the frequency of occurrence of atoms of confusion in practice?}), we mine a set of GitHub open source repositories and compute the prevalence of atom candidates.

We mine a set of GitHub open source repositories and compute how often these atom candidates occur. We do this with the goal of quantifying the prevalence of the investigated atom candidates. \revised{If they rarely occur in real-world systems, this would suggest that they cannot be a significant source of confusion for developers. This study addresses our first research question (\emph{What is the frequency of occurrence of atom candidates in practice?}).}

\subsection{Study Setting}

We first collect the most popular GitHub repositories that are primarily written in JavaScript. We measure popularity using the project's stargazers. This metric, available through the GitHub API, represents the number of stars a project received from users of the platform. The same metric has been used in a number of previous studies as a proxy to estimate projects' popularity~\cite{gyimesi2019bugsjs,canedo:esem2020}. We then select the top 100 most popular repositories and remove projects that do not reach the first quartile of the distribution of lines of code (excluding from our dataset small projects). 

After filtering out JavaScript project candidates, we build a curated dataset comprising a total of \minedprojects repositories. Examples of projects in this dataset include \textsc{React}, \textsc{Node JS}, and \textsc{AngularJS}. Table~\ref{tab:projects-statistics} presents some statistics about the projects we consider in our research. The size of the projects range from small ones (5,543 lines of code) to complex systems with more than 1 MLOC. All projects in our dataset have at least \num[group-separator = {,}]{1244} forks and at least \num[group-separator = {,}]{23672} stars. We automate all the steps to filter, clone, and collect the statistics from the repositories using Python scripts.

We mine the occurrence of atom candidates from the repositories in our curated dataset using source code queries that we specified using the Semmle's CodeQL language~\cite{moor:gttse2007}. CodeQL is an object-oriented variant of the Datalog language~\cite{rodriguez2020efficient}, and currently supports researchers and practitioners to query the source code of systems written in different languages (such as \cpplang, Java, and JavaScript). All CodeQL queries deal with a specific atom candidate, and collect the exact location in the source code where the atoms appear. Figure~\ref{lst:codeql} shows an example of a CodeQL query that mine occurrences of the conditional operator.

Although most of our queries are simple, we have to circumvent a few corner cases to avoid false positives. For instance, we do not consider atom candidates the pre-increment occurrences that appear within
\texttt{for statement} definitions.  We automate the process of running the queries and exporting the results to a format that simplifies our analysis (and also the reproduction of this study). Finally, we compute some descriptive statistics to measure the prevalence of atoms of confusion in practice. 

\begin{table}[ht]
  \centering
   \caption{Some descriptive statistics about the projects used in the MSR study}
%  \begin{scriptsize}
 \begin{tabular}{rrrrr}
   \toprule
                       & Min.             & Median        & Mean             & Max. \\ \midrule
 Lines of Code         & \num[group-separator = {,}]{5543}       & \num[group-separator = {,}]{36161.5} & \num[group-separator = {,}]{111432.33}  & \num[group-separator = {,}]{1278405} \\
 Num. of Forks         & \num[group-separator = {,}]{1244}       & \num[group-separator = {,}]{6078}    & \num[group-separator = {,}]{8906.24}    & \num[group-separator = {,}]{68849} \\
 Num. of Contributors  & \num[group-separator = {,}]{6}          & \num[group-separator = {,}]{285.5}   & \num[group-separator = {,}]{533.44}     & \num[group-separator = {,}]{4047} \\
 Num. of Stars         & \num[group-separator = {,}]{23672}      & \num[group-separator = {,}]{34990}   & \num[group-separator = {,}]{46919.51}   & \num[group-separator = {,}]{310935} \\
 
    \bottomrule
 \end{tabular}
 %\end{scriptsize}
 \label{tab:projects-statistics} 
\end{table}

\begin{figure}
\begin{lstlisting}[language=CodeQL]
import javascript

from ConditionalExpr e
where not e.getTopLevel().isMinified() 
select e, e.getLocation().getStartLine() as Location,
       e.getLocation().getFile()
\end{lstlisting}
\caption{CodeQL query that searches for Conditional Expressions.}
\label{lst:codeql}
\end{figure}

%% \begin{lstlisting}[language=CodeQL]
%% import javascript

%% from ControlStmt c
%% where not c.getTopLevel().isMinified() and
%%  (c instanceof IfStmt and not c.(IfStmt).getCondition().mayHaveBooleanValue(true)) or
%%  (c instanceof LoopStmt and  not c.(LoopStmt).getTest().mayHaveBooleanValue(true) ) or 
%%  (c instanceof WithStmt and not  c.(WithStmt).getExpr().mayHaveBooleanValue(true))  
%% select c, c.getLocation().getStartLine() as Location,c.getLocation().getFile()
%% \end{lstlisting}  

\subsection{Results}
\label{sec:msr-results} 

In this section we present the results of the  software repository
mining effort, whose goal is to reveal how prevalent atom candidates are in open source JavaScript projects. 
We mined \minedprojects open source JavaScript repositories to understand how
often atoms arise in real software. This study focused on atoms that were investigated in
previous research~\cite{DBLP:conf/sigsoft/GopsteinIYDZYC17} and that we explored in the remaining
studies of this research (detailed in Section~\ref{sec:s01}, Section~\ref{sec:s02}, and Section~\ref{sec:s03}).  


Similarly to previous studies~\cite{DBLP:journals/ese/MedeirosLAAKRG19,DBLP:conf/msr/GopsteinZFC18} which investigate
the prevalence of atoms of confusion in open source \clang and \cpplang projects, we found that atoms of confusion
frequently arise in JavaScript open source systems. For instance, the six most frequently found atoms
occur in at least 80\% of the projects. Considering the extremes, atom candidates Conditional Operator and
Implicit Predicate were found in all repositories we mined (see examples in Figure~\ref{lst:conditional-operator-sample}
and Figure~\ref{lst:implicit-predicate-sample}), while Comma Operator occurred in only 15\% of them (as seen in Table~\ref{tab:occurrences-summary}).
Figure~\ref{lst:comma-operator-sample} shows an example of the use of the Comma Operator in the \texttt{styled-components}
project.

\begin{figure}[htb]
  \begin{lstlisting}[language=JavaScript]
function prependModifierMarker (symbol, name, dynamic) {
  return dynamic
  ? ("_p(" + name + ",\"" + symbol + "\")")
  : symbol + name // mark the event as captured
}

function concat (a, b) {
  return a ? b ? (a + ' ' + b) : a : (b || '')
}
  \end{lstlisting}
  \caption{Examples of the Conditional Operator atom present in the \texttt{vue} project (urls: shorturl.at/bJPS4 and shorturl.at/tJOS8)}
   \label{lst:conditional-operator-sample}
\end{figure}

\begin{figure}[htb]
  \begin{lstlisting}[language=JavaScript]
React.useEffect(() => {
  if (props.config.selectedLines) { // Implicit Predicate
    dispatch({
      type: 'MULTILINE',
      selectedLines: props.config.selectedLines,
    })
  }
}, [props.config.selectedLines])
  \end{lstlisting}
  \caption{Example of the Implicit Predicate atom present in the \texttt{carbon} project (url: shorturl.at/aloqU)}
  \label{lst:implicit-predicate-sample}
\end{figure}

\begin{figure}[htb]
  \begin{lstlisting}[language=JavaScript]
var props = ((_props = {}),
 (_props[SC_ATTR] = stringifyNames(names)),
 (_props[SC_VERSION_ATTR] = '4.3.1'),
 _props);
  \end{lstlisting}
  \caption{Example of the Comma Operator atom present in the \texttt{styled-components} project (url: shorturl.at/sCKM0)}
  \label{lst:comma-operator-sample}
\end{figure}

% Figure~\ref{fig:rate} summarizes this finding, showing that the occurrence of atoms of confusion in JavaScript systems range from 11.34\% (Comma Operator) to 89.69\% (Ternary Operator).


%Figure~\ref{fig:atoms-occurrence} shows that the occurrence of in JavaScript systems range from 

% latex table generated in R 4.0.3 by xtable 1.8-4 package
% Fri Nov 13 11:28:56 2020
%%\begin{table}[!htb]
%%\centering
%% \caption{Summary of atoms occurrences on our dataset}
%% 
%%\setlength\tabcolsep{2pt} % default value: 6pt
%%\label{tab:occurrences-summary}
%%\begin{tabular}{lrcc}
%%
%%  \hline
%%Atom & Projects & Occurrences/KLOC & Category \\ 
%% \hline
%%Ternary Operator & 100\% & 10.16 &  highly used \\ 
%%Omitted Curly Braces & 91.67\% & 6.61 &  highly used \\ 
%%Post Increment & 90.28\% & 5.62 &  highly used \\ 
%%Pre-Increment & 83.33\% & 1.04 & commonly used \\ 
%%Assignment as Value & 81.94\% & 1.02 & commonly used \\ 
%%Logic as Control Flow & 56.94\% & 0.95 & commonly used \\ 
%%Arithmetic as Logic & 23.61\% & 0.02 & little used \\ 
%%Comma Operator & 15.28\% & 0.03 & little used \\ 
%%   \hline
%%\end{tabular}
%%\end{table}

% latex table generated in R 4.0.3 by xtable 1.8-4 package
% Tue Apr 13 09:34:35 2021
\begin{table}[ht]
  \centering
\caption{Summary of atom candidate occurrences in our dataset.}
\label{tab:occurrences-summary}
\setlength\tabcolsep{2pt} % default value: 6pt
\begin{tabular}{lrr}%c}
  \toprule
Atom & Projects & Occurr./KLOC \\%& Frequency \\ 
  \midrule
Implicit Predicate & 100\% & 19.89 \\%&  intensively used \\ 
  Conditional Operator & 100\% & 10.16 \\%&  intensively used \\ 
  Omitted Curly Braces & 91.67\% & 6.61 \\%&  intensively used \\ 
  Post Increment & 90.28\% & 5.62 \\%&  intensively used \\ 
  Pre Increment & 83.33\% & 1.04 \\%& commonly used \\ 
  Assignment as Value & 81.94\% & 1.02 \\%& commonly used \\ 
  Logic as Control Flow & 56.94\% & 0.95 \\%& commonly used \\ 
  Automatic Semicolon Insertion & 33.33\% & 0.13 \\%& commonly used \\ 
  Arithmetic as Logic & 23.61\% & 0.02 \\%& rarely used \\ 
  Comma Operator & 15.28\% & 0.03 \\%& rarely used \\
   \bottomrule
\end{tabular}
  
\end{table}


Considering all JavaScript projects in our dataset, we found a total of \num[group-separator = {,}]{364873} atom candidates, though four atoms are responsible
for 92.97\% of this total: Implicit Predicate, Post Increment, Conditional Operator, and Omitted Curly Braces. 
%The first two, based on the results of Section~\ref{sec:repeated} and Section~\ref{sec:latin}, can be considered atoms of confusion. 
The remaining atom
candidates comprise \num[group-separator = {,}]{25621} occurrences combined, 7.03\% of the total number of occurrences.
The Comma Operator and Arithmetic as Logic atoms are the ones that arise less
frequently (0.001\% in total with 232 and 165 occurrences, respectively). 

% Regarding frequency, we proceeded with a classification of \emph{atom occurrence}
% based on thresholds used in a previous
% study~\cite{DBLP:journals/ese/MedeirosLAAKRG19}. 
% Atoms that occur 1000 times are classified as \textit{rarely used} while atoms with more than \num[group-separator = {,}]{10000} occurrences are classified as \textit{highly used}; \textit{commonly used} atoms occur in between.
We calculate the frequency of each atom per KLOC.
The four atoms mentioned in the previous paragraph,
Implicit Predicate, Post Increment, Conditional Operator, and Omitted Curly Braces, occur frequently in the analyzed projects. All of them have more than 5 occurrences per KLOC on average. In particular, two atom candidates occur very frequently, Implicit Predicate (19.89 occurrences per KLOC) and Conditional Operator (10.16 occurrences per KLOC). On the other hand, Arithmetic as Logic and Comma Operator occur less than once for every 50 KLOC. 
\revised{Given the frequency of these code patterns in open-source JavaScript programs, it is worth it to investigate their impact on understanding JavaScript code. This is the goal of the studies we present in the following sections.}
%In our dataset of JavaScript projects, 
%We found the Ternary Operator atom occurring more frequently than the Omitted Curly Braces, differently from what has been reported in a previous work~\cite{DBLP:conf/msr/GopsteinZFC18}. %




%%%%%%%%%%%%%%%%%%%%%% MAYBE the following can go to a discussion section. 
%%%%%%%%%%%%%%%%%%%%%% 
%The first two, based on the results of Section~\ref{sec:repeated} and Section~\ref{sec:latin}, can be considered atoms of confusion. 

%The results of our experiments and interviews suggest that Conditional Operator does not contribute significantly as a source of misunderstanding (increasing the number of wrong answers in 9\% of the cases, according to the Latin-square study). Nonetheless, the Post-Increment Expression and Automatic Semicolon Insertion atoms are listed in the top four sources of misunderstanding  (Table~\ref{tab:difference-correctness}), and also frequently appear in open source systems. As such, refactoring existing systems to avoid Post-Increment and Automatic Semicolon Insertion might improve the readability of large amounts of code in JavaScript projects. 




 

\section{Discussion}
\label{sec:discussion}

Our work leads to several implications.
First, we generalize the
work of Gopstein et al.~\cite{DBLP:conf/sigsoft/GopsteinIYDZYC17},
which investigates
the impact of atom candidates on
understanding \clang and \cpplang code. That is,
similarly to their work, the atom candidates
Comma Operator, Post/Pre Increment, Omitted Curly Braces,
Assignment as Value, Implicit Predicate, Logic as
Control Flow, and Ternary Operator make the 
JavaScript code hard to understand. Our results also 
refute the hypothesis that Arithmetic as Logic is an atom of
confusion (i.e., a source of misunderstanding).
In comparison to the previous work, our study
led to some differences in the effect size
of the atom candidates, in comparison to the original
work of Gopstein et al.~\cite{DBLP:conf/sigsoft/GopsteinIYDZYC17}.
Altogether, we answer our first research question
\emph{\rqa}:

\begin{mh}
  {\bf Answer to RQ1:} The first study (survey) give evidence that
  the atom candidates for \clang and \cpplang
  that also exist for JavaScript correspond to
  a source of misunderstanding in
  JavaScript code. 
\end{mh}

The results of the interview study complement the understanding
of atoms of confusion because the participants make
clear the existence of a trade-off between code comprehension
and other quality attributes. For instance, most of the participants
prefer the version of the code with the Ternary Operator,
even though they agree that its uses might contribute to the
misunderstanding of JavaScript code, particularly when novices are maintaining
the codebase. The participants of the interview study also
mentioned other possible sources of misunderstanding in JavaScript,
including the use of prototype-inheritance and nested call-backs (as discussed in Section~\ref{sec:interview-results}). Other JavaScript atom candidates include
Object Destructuring, Array Spread, Object Spread, and Type Conversion.
In summary, the results of the second study (interviews) allow
us to answer the second (\rqb) and third research questions
(\emph{\rqc}).

\begin{mh}
  {\bf Answer to RQ2:} The qualitative analysis of the
  interviews supports the results of the first study,
  indicating that atoms of confusion compromise
  source code understanding. 
\end{mh}

\begin{mh}
  {\bf Answer to RQ3:} The qualitative analysis of the
  interviews suggest that specific constructs of
  JavaScript might also correspond to atoms of
  confusion, including prototype inheritance and
  nested callbacks. 
\end{mh}

The result of the third study (mining open source
Java script repositories) give evidence that,
although atoms of confusion compromise program
comprehension, they frequently appear in open
source JavaScript projects. In particular,
{\color{red}six, out of nine atoms considered
  in our study,} appear in more than 50\% of
the projects. To sum up, the third study
allows us to answer the fourth research
question (\emph{\rqd}).


\begin{mh}
  {\bf Answer to RQ4:} The MSR study reveals that
  the atoms of confusion explored in our research
  appear frequently in practice. 
\end{mh}



%% We conducted a non-exact replication of the three
%% studies (survey, interview, and mining software repositories)
%% considering these more specific JavaScript atom candidates.
%% We confirmed that they truly correspond to sources of misunderstanding.
%% Due to lack of space,
%% we cannot present all the results here, and we postpone the presentation of these results as future work.




\section{Threats to Validity}
\label{threat}

In this section we discuss some of the main threats to the validity of this work. 

%Conclusion validity is connected with how well it is possible to establish relationships between treatments and outcomes. Threats to conclusion validity often come from inappropriate use of statistics. 

\textbf{Conclusion validity.} 
%Threats to conclusion validity often come from inappropriate use of statistics. 
In the context of our survey study, we apply different non-parametric statistical tests appropriate for the cases where data was categorical (correctness, Chi-square test of independence) and continuous (time, Mann-Whitney U test). Furthermore, besides reporting p-values, we also report effect size measures appropriate for each scenario (odds-ratio and Cliff's delta) and apply a p-value correction technique to avoid the multiple testing problem. Finally, it could be argued that the size of the samples is insufficient to make conclusions for some of the atoms, a common problem in Empirical Software Engineering. We estimate the sample size for each one of the atom candidates, considering that each one had a different number of samples (Table~\ref{tab:difference-correctness}). To that end, we use the $\phi$ measure of effect size for each atom (based on the Chi-square statistic), the standard $\alpha$ coefficient of 0.05, and set the expected statistical power to 0.8, as usually employed in the literature~\cite{Ellis:2010:EGE}. We find out that the sample size is sufficient for all but one of the atoms where we had a statistically significant result for correctness, Implicit Predicate. 
% For the ones where we did not find statistically significant results for correctness, in fact the sample sizes
% we have analyzed are insufficient for such small effect sizes. This indicates that further studies on these
% atoms are required, due to the likelihood of type II errors.

%Construct validity is connected with how well the selected measures actually represent the concept of interest. 

\textbf{Construct validity.} We use correctness as a proxy for program comprehension and predicted outcomes of small programs as a measure of correctness. As discussed elsewhere~\cite{Oliveira:2020:ECR}, this approach is a test of the developers' ability to trace programs. Although this is a common approach in studies about atoms of confusion~\cite{TheEyesDoNotLie,Langhout:2021:ACJ,DBLP:conf/sigsoft/GopsteinIYDZYC17}, other measures of correctness could have been employed, potentially yielding different results. As a complement to correctness in the survey, we have measured the time it took for the participants to correctly predict the outcome of the code snippets (either with or without atom candidates). Furthermore, we have  asked the interviewees about their preferences when comparing confusing and clean versions of the programs.

%A potential problem with our method is that there little incentive for subjects to think thoroughly about the questions. We observed a lack of engagement when we ran the survey with undergraduate students during a pilot study. Although our final subjects were voluntarily partaking in the survey, we could not be sure that, after some time taking the survey, respondents would become tired and stop thinking clearly about the code.


%Internal validity is concerned with how well the study isolates the variables of interest and accounts for confounding factors. 

\textbf{Internal validity.} Since the survey was conducted online with unknown participants, we have no way of confirming their levels of education and experience. We mitigate this threat by, in the data analysis, explicitly accounting for programmer experience and using an experimental design that allows us to isolate the impact of the treatment from factors such as experience and formal education level. Also, we did not have a way to prevent respondents from cheating, such as running the code on an interpreter, or consulting other people. We partially mitigate this threat by presenting images with source code, instead of text. This creates an obstacle for participants to run the code while taking the survey.

%External validity is linked to whether it is possible to extrapolate the results of the study. 

\textbf{External validity.} Our results suggest that the selected idioms and code constructs may lead to confusion for small code snippets, but it is not clear if that result extrapolates to other scenarios. Even though it is likely that in larger code bases the confusion induced by these constructs may be even greater, the existence of additional context may mitigate this effect. Another possible threat to the generalizability of our results, in particular for the survey and interviews, lies in the fact that the analyzed atoms may rarely occur in real JavaScript code bases. To mitigate that threat, we have analyzed 72 popular open source JavaScript repositories and found out that most of them are common, occurring at least once per 1,000 lines of code. Only Arithmetic as Logic and Comma Operator occur less frequently than once per 10,000 lines of code.


\revised{We manually devised sets of code snippets to use in our experiments. These code snippets should have been designed considering the concept of ``minimality''~\cite{DBLP:conf/sigsoft/GopsteinIYDZYC17}
  to avoid other sources of misunderstanding. After running the experiments, we observed that this concept was not well addressed in a few cases. In particular, the code snippet contained the atom candidate Logic as Control Flow in our second experiment (based on the Latin Square design) has additional atom candidates. This could have inflated the number of wrong answers in the obfuscated version of this code pattern. However, the experiment results show the contrary, and the findings of the second experiment do not confirm Logic as Control Flow as an atom of confusion. Further analysis should be
  conducted to understand the effect of multiple atoms in the same code snippet on program comprehension tasks.}


\revised{\textbf{Precision of our CodeQL queries.}
  We relied on existing CodeQL JavaScript predicates to implement most of our source code queries. However,
  we still had to implement our own predicates for three specific queries: the queries that search for the atoms Automatic Semicolon Insertion, Comma Operator, and Omitted Curly Braces.
  Figure~\ref{lst:codeql-automatic} shows examples of predicates we have implemented. Our test cases and manual assessments revealed that our queries correctly identified the
  atoms in the source code. Unfortunately, we observed that a predefined CodeQL predicate (named mayHaveBooleanValue) we used to mine the Implicit Predicate atom might have
  produced many false positives. That is, there are situations where this predicate wrongly states that an expression does not have a boolean value.
  This issue might have inflated the number of Implicit Predicate occurrences we reported in Section~\ref{sec:s04}}.

\begin{figure}[htb]
\begin{lstlisting}[language=CodeQL, basicstyle=\scriptsize]
import javascript

predicate hasTokenAfter(ReturnStmt r) {
  r.getLastToken().getNextToken().getValue() != "}"
}


predicate isNextTokenVariable(ReturnStmt r) {
  r.getLastToken().getNextToken() instanceof IdentifierToken and 
  r.getLastToken().getNextToken().getNextToken().getValue() != "="
}


predicate isNextTokenLiteral(ReturnStmt r) {
  ( r.getLastToken().getNextToken() instanceof StringLiteralToken or
    r.getLastToken().getNextToken() instanceof NumericLiteralToken ) and 
  r.getLastToken().getNextToken().getNextToken().getValue() != "="
 }

from ReturnStmt r
where not r.getTopLevel().isMinified() and
 r.isSubjectToSemicolonInsertion() and 
 r.getNumChildStmt() = 0 and
 r.getNumChildExpr() = 0 and
 not r.toString().matches("%;") and
 hasTokenAfter(r) 
select r, r.getLocation().getStartLine(), r.getLocation().getFile()
\end{lstlisting}
  \caption{CodeQL query that searches for the atom Automatic Semicolon Insertion. Note that, for this query,
    we had to implement three CodeQL predicates.}
  \label{lst:codeql-automatic}
\end{figure}


%% \textbf{A note on independent studies.} The Repeated Measures and the Latin Squares studies were designed
%% and conducted independently from each other. Only after the conduction of both studies that we
%% decided to join efforts and report all results in here, in a singlepaper. Someone might argue that both studies are
%% too independent, in such a way that we could hardly combine their results. Perhaps if we had join 
%% the research efforts before, we could had review some design decisions so that we could seamly
%% integrate the results of both studies.


% \castor{A forma como os átomos foram detectados é uma ameaça que não discutimos.}


%% Finally, in one of the atoms, namely the Omitted Curly Braces,
%% we intentionally removed indentation from the original code,
%% which is highly unusual, given that many programmers use automatic
%% formatting in their code editors. This can introduce some level
%% of artificiality to this atom's question. Nonetheless,
%% we discovered that JavaScript does allow the programmer
%% to omit the curly braces after \textit{if} statements,
%% and insert multiple statements in the following line.
%% This fact itself might constitute
%% a source of confusion, which we leave to analyse
%% in our future endeavors. 

\input{conclusion}

% \section*{Acknowledgment}
{\small
\bibliographystyle{elsarticle-num-names} 
\bibliography{reference}
}
\end{document}
