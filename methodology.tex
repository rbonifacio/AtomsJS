\section{Research Methods}
\label{method}

The main goal of this research is to investigate the impact of atoms of confusion on JavaScript code comprehension. As such, in this paper we answer the following research questions: 

\begin{enumerate}[(RQ1)]
\item \rqa 
\item \rqb
% \item \rqc  
\item \rqd
\end{enumerate}

By answering these research questions, we can either generalize or refute the findings about atoms of confusion already discussed in the literature (goal of research questions (RQ1) and (RQ2)). 
In addition, answering the third question allows us to enrich existing catalogs about atoms of confusion and discuss how often they occur in practice. Answering these research questions also lays the foundations for the implementation of tools that can automatically transform code into cleaner versions.
%---though we postpone these results to future research work.
We conduct a mixed-methods study to answer these questions. 
It includes two independently designed and conducted experiments: a repeated measures/within subjects study (Section~\ref{sec:meth:replication}) and one using a Latin square, between subjects, counterbalancing experimental design to control for subject experience (Section~\ref{sec:meth:survey}). By using two different experiments, designed by different researchers, with different designs, and conducted with different participants, we hope to rigorously verify the impact of atoms of confusion in code comprehension activities. \revised{Conducting multiple studies, with different designs and samples, is a typical approach in other areas of science to obtain more solid evidence and also to contest evidence obtained by single study, for example, Weggemans et al.~\cite{Weggemans:2001:DCE}. We discuss this matter further in Section~\ref{sec:whytwo}}.

These two experiments explore the impact of atom candidates on understanding JavaScript code (goal of research question RQ1). They investigate whether or not programs that contain atom candidates tend to produce more misunderstanding for programmers trying to predict their outcome.% We leverage two independently-developed and conducted experiments with different designs to more  
We also conduct a set of interviews with developers (Section~\ref{sec:method:interview}) to contrast the quantitative results of the experiments with their preferences and opinions (thus addressing RQ2). Finally, we perform a repository mining study (Section~\ref{sec:method:msr}) that investigates the prevalence of some of these atom candidates in large scale, professionally-developed JavaScript software.

